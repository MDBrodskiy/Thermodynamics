%%%%%%%%%%%%%%%%%%%%%%%%%%%%%%%%%%%%%%%%%%%%%%%%%%%%%%%%%%%%%%%%%%%%%%%%%%%%%%%%%%%%%%%%%%%%%%%%%%%%%%%%%%%%%%%%%%%%%%%%%%%%%%%%%%%%%%%%%%%%%%%%%%%%%%%%%%%%%%%%%%%
% Written By Michael Brodskiy
% Class: Thermodynamics & Statistical Mechanics
% Professor: A. Stepanyants
%%%%%%%%%%%%%%%%%%%%%%%%%%%%%%%%%%%%%%%%%%%%%%%%%%%%%%%%%%%%%%%%%%%%%%%%%%%%%%%%%%%%%%%%%%%%%%%%%%%%%%%%%%%%%%%%%%%%%%%%%%%%%%%%%%%%%%%%%%%%%%%%%%%%%%%%%%%%%%%%%%%

\documentclass[12pt]{article} 
\usepackage{alphalph}
\usepackage[utf8]{inputenc}
\usepackage[russian,english]{babel}
\usepackage{titling}
\usepackage{amsmath}
\usepackage{graphicx}
\usepackage{enumitem}
\usepackage{amssymb}
\usepackage[super]{nth}
\usepackage{everysel}
\usepackage{ragged2e}
\usepackage{geometry}
\usepackage{multicol}
\usepackage{fancyhdr}
\usepackage{cancel}
\usepackage{siunitx}
\usepackage{physics}
\usepackage{tikz}
\usepackage{mathdots}
\usepackage{yhmath}
\usepackage{cancel}
\usepackage{color}
\usepackage{array}
\usepackage{multirow}
\usepackage{gensymb}
\usepackage{tabularx}
\usepackage{extarrows}
\usepackage{booktabs}
\usepackage{lastpage}
\usetikzlibrary{fadings}
\usetikzlibrary{patterns}
\usetikzlibrary{shadows.blur}
\usetikzlibrary{shapes}

\geometry{top=1.0in,bottom=1.0in,left=1.0in,right=1.0in}
\newcommand{\subtitle}[1]{%
  \posttitle{%
    \par\end{center}
    \begin{center}\large#1\end{center}
    \vskip0.5em}%

}
\usepackage{hyperref}
\hypersetup{
colorlinks=true,
linkcolor=blue,
filecolor=magenta,      
urlcolor=blue,
citecolor=blue,
}


\title{Homework 3}
\date{October 7, 2023}
\author{Michael Brodskiy\\ \small Professor: A. Stepanyants}

\begin{document}

\maketitle

\begin{enumerate}

  \item

    We can find the sum of all photons to be:

    $$\sum\langle s_n\rangle=\sum\frac{1}{e^{\frac{\hbar\omega_n}{\tau}}-1}$$

    We know that in a cavity of volume $V$, the edge length is $L$, which means $V=L^3$. From here, we know $\omega_n=\frac{n\pi c}{L}$. Now, if we were to assume that $n_x$, $n_y$, and $n_z$ are in a positive octant, we can get:

    $$\sum_n\frac{1}{e^{\frac{\hbar n\pi c}{L\tau}}-1}=\frac{1}{4}\int_0^\infty\left( \frac{4\pi n^2}{e^{\frac{\hbar n\pi c}{L\tau}}-1} \right)\,dn$$
    $$=\pi\int_0^\infty\left( \frac{4\pi n^2}{e^{\frac{\hbar n\pi c}{L\tau}}-1} \right)\,dn$$

    If we perform a $u$-substitution, we obtain:

    $$u=\frac{\hbar n\pi c}{L\tau}$$
    $$du=\frac{\hbar \pi c}{L\tau}dn$$

    Substituting this in, we get:

    $$\pi\left(  \frac{L\tau}{\hbar \pi c}\right)^3\int_0^\infty\frac{u^2}{e^u-1}\,du$$

    Using a numerical solver, we find:

    $$\int_0^\infty\frac{u^2}{e^u-1}\,du=2.404$$

    Thus, we get:

    $$N=\frac{L^3\tau^3}{\pi^2\hbar^3c^3}(2.404)$$

    We can simplify the $L^3$ to get:

    $$\boxed{N=\frac{V\tau^3}{\pi^2\hbar^3c^3}(2.404)}$$

    Finding $\frac{\sigma}{N}$, we find:

    $$\frac{\sigma}{N}=\left( \frac{4\pi^2V\tau^3}{45\hbar^3c^3} \right)\left( \frac{\pi^2\hbar^3c^3}{V\tau^3(2.404)} \right)$$
    $$=\frac{4\pi^4}{45}$$
    $$=3.6017$$

  \item

    \begin{enumerate}

      \item 

        To find the surface energy production rate, we need to simply multiply the density by the surface area. This gives us:

        $$4\pi R^2\cdot C_s=4\pi(1.49\cdot10^{11})^2\cdot.136$$
        $$=3.794\cdot10^{22}$$

        We then need to convert the radius, which was in meters, to centimeters:

        $$\text{Energy rate } = 3.794\cdot10^{22}\cdot\left( 10^2 \right)^2$$
        $$\text{Energy rate } = 3.794\cdot10^{26}\left[ \frac{\si{\joule}}{\si{\second}} \right]$$
        $$\boxed{\text{Energy rate } \approx 4\cdot10^{26}\left[ \si{\watt}\right]}$$

      \item 

        We first find the energy emission per unit are of the sun itself, using the radius provided:

        $$\frac{4\cdot10^{26}}{4\pi(7\cdot10^{10})^2}=6496.1\left[ \frac{\si{\watt}}{\si{\centi\meter\squared}} \right]$$

        We know this value is equal to the Stefan-Boltzmann constant multiplied by the temperature to the power of four. From here, we can solve:

        $$\frac{6496.1}{5.67\cdot10^{-12}}=T^4$$
        $$T^4=1.1457\cdot10^{15}[\si{\kelvin}^4]$$
        $$T=5,817.9[\si{\kelvin}]$$
        $$\boxed{T\approx 6,000[\si{\kelvin}]}$$

    \end{enumerate}

    \setcounter{enumi}{3}

  \item

    \begin{enumerate}

      \item 

        First we start off by finding the atomic mass difference when converting hydrogen to helium:

        $$4(1.0078)-4.0026=.0286[\si{\amu}]$$

        We then convert this value to kilograms:

        $$.0286\cdot\left( \frac{1.6726\cdot10^{26}}{1.00727647} \right)=4.79\cdot10^{-29}[\si{\kilo\gram}]$$

        We then plug this into Einstein's equation to find the energy:

        $$4.79\cdot10^{-29}\cdot(3\cdot10^8)^2=4.27\cdot10^{-12}[\si{\joule}]$$

        We then multiply this by the mass of the sun, divided by the quantity of $10\%$ of the hydrogen to find the available energy:

        $$\boxed{E=(.1)(2\cdot10^{30})\left( \frac{1}{4\cdot 1.0078} \right)\left( \frac{1.00727647}{1.6726\cdot10^{-27}} \right)(4.27\cdot10^{-12})=1.277\cdot10^{44}[\si{\joule}]}$$

      \item 

        We need to divide the amount from part (a), and divide it by the energy radiation rate:

        $$\frac{1.277\cdot10^{44}}{4\cdot10^{26}}=3.1925\cdot10^{17}[\si{\second}]$$

        We then convert seconds to years to find:

        $$\boxed{2.875\cdot10^{18}\cdot(3600)^{-1}\cdot(24)^{-1}\cdot(365)^{-1}=1.012\cdot10^{10}[\text{yr}]}$$

    \end{enumerate}

  \item

    Setting up a ratio, we can find that:

    $$R^2_{sun}T^4_{sun}=R^2_{earth}T^4_{earth}$$

    This gives us:

    $$T_{earth}=T_{sun}\sqrt{\frac{R_{sun}}{R_{earth}}}$$

    Plugging in our values, we get:

    $$T_{earth}=5800\sqrt{\frac{7\cdot10^{10}}{1.5\cdot10^{13}}}$$
    $$\boxed{T_{earth}=396.22[\si{\kelvin}]}$$

  \item

    \begin{enumerate}

      \item 

        We know that in a ``$j$-th'' mode, the total energy can be described as:

        $$U=\sum_j \varepsilon_j$$

        For a harmonic, we can find that $\varepsilon_j=s_j\hbar\omega_j$:

        $$U=\sum_j s_j\hbar\omega_j$$

        Then, we know:

        $$P=-\frac{\partial U}{\partial V}$$

        Combining the two, we see:

        $$\boxed{P=-\sum_js_j\hbar\frac{\partial\omega_j}{\partial V}}$$

      \item 

        Since we know:

        $$\omega_j=\frac{j\pi c}{L}\text{ and }V=L^3$$

        We can combine the two to find:

        $$\omega_j=\frac{j\pi c}{\sqrt[3]{V}}$$

        Then, we differentiate:

        $$\frac{\partial \omega_j}{\partial V}=-\frac{1}{3}j\pi c V^{-\frac{4}{3}}$$

        This can be simplified to find:

        $$\boxed{\frac{\partial \omega_j}{\partial V}=-\frac{\omega_j}{3V}}$$

      \item 

        Combining (a) and (b), we find:

        $$P=\sum_j \frac{s_j\hbar\omega_j}{3V}$$

        Which becomes:

        $$\boxed{P=\frac{U}{3V}}$$

      \item 

        We may use the formula:

        $$P=\frac{Nk_BT}{V}$$

        This would yield:

        $$P=\underbrace{\left( 6.022\cdot 10^{23} \right)\left( 10^2 \right)^3}_{\text{molecules per volume}}\underbrace{\left( 1.381\cdot10^{-23} \right)}_{\text{Boltzmann constant}}\underbrace{\left( 2\cdot10^7 \right)}_{\text{temperature}}=1.6633\cdot10^{14}\left[ \si{\pascal} \right]$$

        Now, using the formula from (c)

        $$P=\frac{3U}{V}$$
        $$P=\frac{\pi^2\tau^4}{(3)(15)\hbar^3c^3}$$

        To convert to temperatures, we use the multiply both sides by the Boltzmann constant:

        $$\frac{\pi^2k_B^4T^4}{45\hbar^3c^3}=\frac{Nk_BT}{V}$$

        Simplifying, we find:

        $$T=\sqrt[3]{\frac{45\hbar^3c^3N}{\pi^2k_B^3V}}$$
        $$\boxed{T=3.2\cdot10^7[\si{\kelvin}]}$$

    \end{enumerate}

\end{enumerate}

\end{document}


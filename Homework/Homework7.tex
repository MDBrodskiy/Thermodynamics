%%%%%%%%%%%%%%%%%%%%%%%%%%%%%%%%%%%%%%%%%%%%%%%%%%%%%%%%%%%%%%%%%%%%%%%%%%%%%%%%%%%%%%%%%%%%%%%%%%%%%%%%%%%%%%%%%%%%%%%%%%%%%%%%%%%%%%%%%%%%%%%%%%%%%%%%%%%%%%%%%%%
% Written By Michael Brodskiy
% Class: Thermodynamics & Statistical Mechanics
% Professor: A. Stepanyants
%%%%%%%%%%%%%%%%%%%%%%%%%%%%%%%%%%%%%%%%%%%%%%%%%%%%%%%%%%%%%%%%%%%%%%%%%%%%%%%%%%%%%%%%%%%%%%%%%%%%%%%%%%%%%%%%%%%%%%%%%%%%%%%%%%%%%%%%%%%%%%%%%%%%%%%%%%%%%%%%%%%

\documentclass[12pt]{article} 
\usepackage{alphalph}
\usepackage[utf8]{inputenc}
\usepackage[russian,english]{babel}
\usepackage{titling}
\usepackage{amsmath}
\usepackage{graphicx}
\usepackage{enumitem}
\usepackage{amssymb}
\usepackage[super]{nth}
\usepackage{everysel}
\usepackage{ragged2e}
\usepackage{geometry}
\usepackage{multicol}
\usepackage{fancyhdr}
\usepackage{cancel}
\usepackage{siunitx}
\usepackage{physics}
\usepackage{tikz}
\usepackage{mathdots}
\usepackage{yhmath}
\usepackage{cancel}
\usepackage{color}
\usepackage{array}
\usepackage{multirow}
\usepackage{gensymb}
\usepackage{tabularx}
\usepackage{extarrows}
\usepackage{booktabs}
\usepackage{lastpage}
\usetikzlibrary{fadings}
\usetikzlibrary{patterns}
\usetikzlibrary{shadows.blur}
\usetikzlibrary{shapes}

\geometry{top=1.0in,bottom=1.0in,left=1.0in,right=1.0in}
\newcommand{\subtitle}[1]{%
  \posttitle{%
    \par\end{center}
    \begin{center}\large#1\end{center}
    \vskip0.5em}%

}
\usepackage{hyperref}
\hypersetup{
colorlinks=true,
linkcolor=blue,
filecolor=magenta,      
urlcolor=blue,
citecolor=blue,
}


\title{Homework 7}
\date{November 15, 2023}
\author{Michael Brodskiy\\ \small Professor: A. Stepanyants}

\begin{document}

\maketitle

\begin{enumerate}

  \item

    \begin{enumerate}

      \item 

        First and foremost, we know:

        $$Q_h=Q_l+W$$
        $$W=Q_l-Q_h$$

        Per reversible conditions:

        $$\sigma_h=\sigma_l$$
        $$\frac{Q_h}{\tau_h}=\frac{Q_l}{\tau_l}$$
        $$Q_h=\frac{\tau_hQ_l}{\tau_l}$$

        Thus, we can combine to write:

        $$W=Q_h-\frac{\tau_l}{\tau_h}Q_h$$
        $$W=Q_h\left( 1-\frac{\tau_l}{\tau_h}\right)$$
        $$W=Q_h\left( \frac{\tau_h-\tau_l}{\tau_h}\right)$$

        And finally:

        $$\boxed{\frac{W}{Q_h}=\frac{\tau_h-\tau_l}{\tau_h}}$$

        If the heat pump is not reversible, we know the efficiency is:

        $$\frac{W}{Q_h}<\frac{\tau_h-\tau_l}{\tau_h}$$

      \item 

        We can write:

        $$W=(\tau_{hh}-\tau_l)\left( \frac{Q_{hh}}{\tau_{hh}}-\frac{Q_l}{\tau_l} \right)$$

        We know that the work generated is used as $Q_h$ in the heat pump:

        $$Q_h=(\tau_{hh}-\tau_l)\left( \frac{Q_{hh}}{\tau_{hh}}-\frac{Q_l}{\tau_l} \right)$$

        We can then rearrange to find the desired ratio:

        $$\frac{Q_h}{\tau_{hh}-\tau_l}=\frac{Q_{hh}\tau_l-Q_l\tau_{hh}}{\tau_{hh}\tau_l}$$
        $$\frac{\tau_{hh}\tau_l}{\tau_{hh}-\tau_l}=\frac{Q_{hh}\tau_l-Q_l\tau_{hh}}{Q_h}$$

        We know $Q_l=\frac{\tau_l}{\tau_h}Q_h$, which gives us:

        $$\frac{\tau_{hh}\tau_l}{\tau_{hh}-\tau_l}=\frac{Q_{hh}\tau_l}{Q_h}-\frac{\tau_l\tau_{hh}}{\tau_h}$$
        $$\boxed{\frac{Q_{hh}}{Q_h}=\frac{\tau_{hh}}{\tau_{hh}-\tau_l}+\frac{\tau_{hh}}{\tau_h}}$$

        Using the given values, we obtain:

        $$\frac{Q_{hh}}{Q_h}=\frac{600}{600-270}+2$$
        $$\boxed{\frac{Q_{hh}}{Q_h}=3.82}$$

      \item 

        For this problem, we essentially need to draw a ``double Carnot engine,'' with the output work of one ($W$) being the $Q_h$ of the second:

        \begin{figure}[H]
          \centering
          \tikzset{every picture/.style={line width=0.75pt}} %set default line width to 0.75pt        

\begin{tikzpicture}[x=0.75pt,y=0.75pt,yscale=-1,xscale=1]
%uncomment if require: \path (0,444); %set diagram left start at 0, and has height of 444

%Shape: Rectangle [id:dp10170211024317433] 
\draw   (147,21) -- (492,21) -- (492,61) -- (147,61) -- cycle ;
%Shape: Rectangle [id:dp4544811596066871] 
\draw   (147,360) -- (492,360) -- (492,400) -- (147,400) -- cycle ;
%Shape: Rectangle [id:dp3842791172068818] 
\draw   (185.75,76.88) -- (453.25,76.88) -- (453.25,201.96) -- (185.75,201.96) -- cycle ;
%Shape: Rectangle [id:dp8874129257814114] 
\draw   (246.38,219.04) -- (392.63,219.04) -- (392.63,344.12) -- (246.38,344.12) -- cycle ;
%Curve Lines [id:da12079388453509687] 
\draw    (531,112) .. controls (504.27,146.65) and (483.42,94.07) .. (454.87,135.71) ;
\draw [shift={(454,137)}, rotate = 303.39] [color={rgb, 255:red, 0; green, 0; blue, 0 }  ][line width=0.75]    (10.93,-3.29) .. controls (6.95,-1.4) and (3.31,-0.3) .. (0,0) .. controls (3.31,0.3) and (6.95,1.4) .. (10.93,3.29)   ;
%Straight Lines [id:da13771774514472246] 
\draw    (319.5,20) -- (319.5,137.42) ;
\draw [shift={(319.5,139.42)}, rotate = 270] [color={rgb, 255:red, 0; green, 0; blue, 0 }  ][line width=0.75]    (10.93,-3.29) .. controls (6.95,-1.4) and (3.31,-0.3) .. (0,0) .. controls (3.31,0.3) and (6.95,1.4) .. (10.93,3.29)   ;
%Straight Lines [id:da42095144155513076] 
\draw    (319.5,139.42) -- (319.5,340.58) ;
\draw [shift={(319.5,342.58)}, rotate = 270] [color={rgb, 255:red, 0; green, 0; blue, 0 }  ][line width=0.75]    (10.93,-3.29) .. controls (6.95,-1.4) and (3.31,-0.3) .. (0,0) .. controls (3.31,0.3) and (6.95,1.4) .. (10.93,3.29)   ;
%Straight Lines [id:da693914432473546] 
\draw    (415.92,159.42) -- (415.92,396.84) ;
\draw [shift={(415.92,398.84)}, rotate = 270] [color={rgb, 255:red, 0; green, 0; blue, 0 }  ][line width=0.75]    (10.93,-3.29) .. controls (6.95,-1.4) and (3.31,-0.3) .. (0,0) .. controls (3.31,0.3) and (6.95,1.4) .. (10.93,3.29)   ;
%Straight Lines [id:da18101276066233973] 
\draw    (397.92,139.42) -- (319.5,139.42) ;
%Straight Lines [id:da8566525074884959] 
\draw    (216,20.58) -- (216,75) ;
\draw [shift={(216,77)}, rotate = 270] [color={rgb, 255:red, 0; green, 0; blue, 0 }  ][line width=0.75]    (10.93,-3.29) .. controls (6.95,-1.4) and (3.31,-0.3) .. (0,0) .. controls (3.31,0.3) and (6.95,1.4) .. (10.93,3.29)   ;
%Straight Lines [id:da2421143339083962] 
\draw    (272,343.29) -- (272,397.71) ;
\draw [shift={(272,399.71)}, rotate = 270] [color={rgb, 255:red, 0; green, 0; blue, 0 }  ][line width=0.75]    (10.93,-3.29) .. controls (6.95,-1.4) and (3.31,-0.3) .. (0,0) .. controls (3.31,0.3) and (6.95,1.4) .. (10.93,3.29)   ;
%Straight Lines [id:da6170237872934388] 
\draw    (319.5,342.58) -- (319.5,396) ;
\draw [shift={(319.5,398)}, rotate = 270] [color={rgb, 255:red, 0; green, 0; blue, 0 }  ][line width=0.75]    (10.93,-3.29) .. controls (6.95,-1.4) and (3.31,-0.3) .. (0,0) .. controls (3.31,0.3) and (6.95,1.4) .. (10.93,3.29)   ;
%Straight Lines [id:da3109235268228321] 
\draw    (262,20.58) -- (262,75) ;
\draw [shift={(262,77)}, rotate = 270] [color={rgb, 255:red, 0; green, 0; blue, 0 }  ][line width=0.75]    (10.93,-3.29) .. controls (6.95,-1.4) and (3.31,-0.3) .. (0,0) .. controls (3.31,0.3) and (6.95,1.4) .. (10.93,3.29)   ;
%Shape: Arc [id:dp5226403073336301] 
\draw  [draw opacity=0] (396.94,139.42) .. controls (396.94,139.42) and (396.94,139.42) .. (396.94,139.42) .. controls (396.94,139.42) and (396.94,139.42) .. (396.94,139.42) .. controls (407.42,139.42) and (415.92,148.28) .. (415.92,159.21) -- (396.94,159.21) -- cycle ; \draw   (396.94,139.42) .. controls (396.94,139.42) and (396.94,139.42) .. (396.94,139.42) .. controls (396.94,139.42) and (396.94,139.42) .. (396.94,139.42) .. controls (407.42,139.42) and (415.92,148.28) .. (415.92,159.21) ;  
%Curve Lines [id:da04778616308795569] 
\draw    (169,307) .. controls (195.73,272.35) and (216.58,324.93) .. (245.13,283.29) ;
\draw [shift={(246,282)}, rotate = 123.39] [color={rgb, 255:red, 0; green, 0; blue, 0 }  ][line width=0.75]    (10.93,-3.29) .. controls (6.95,-1.4) and (3.31,-0.3) .. (0,0) .. controls (3.31,0.3) and (6.95,1.4) .. (10.93,3.29)   ;
%Straight Lines [id:da8681702403313141] 
\draw    (216,201) -- (217.98,396) ;
\draw [shift={(218,398)}, rotate = 269.42] [color={rgb, 255:red, 0; green, 0; blue, 0 }  ][line width=0.75]    (10.93,-3.29) .. controls (6.95,-1.4) and (3.31,-0.3) .. (0,0) .. controls (3.31,0.3) and (6.95,1.4) .. (10.93,3.29)   ;

% Text Node
\draw (213.75,24.28) node [anchor=north east] [inner sep=0.75pt]    {$\sigma _{h}$};
% Text Node
\draw (264,23.98) node [anchor=north west][inner sep=0.75pt]    {$\tau _{h}{}_{h}$};
% Text Node
\draw (216.75,395.9) node [anchor=south east] [inner sep=0.75pt]    {$\sigma _{l} \ $};
% Text Node
\draw (321.5,136.02) node [anchor=south west] [inner sep=0.75pt]    {$\ Q_{hh}$};
% Text Node
\draw (417.92,395.44) node [anchor=south west] [inner sep=0.75pt]    {$\ Q_{l}$};
% Text Node
\draw (321,198.71) node [anchor=south west] [inner sep=0.75pt]    {$W\rightarrow Q_{h}$};
% Text Node
\draw (270,396.31) node [anchor=south east] [inner sep=0.75pt]    {$\sigma _{l} \ $};
% Text Node
\draw (317.5,278.18) node [anchor=south east] [inner sep=0.75pt]    {$Q_{h} -Q_{l} \ $};
% Text Node
\draw (317.5,394.6) node [anchor=south east] [inner sep=0.75pt]    {$\tau _{l} \ $};
% Text Node
\draw (248.38,222.44) node [anchor=north west][inner sep=0.75pt]    {$\Delta \sigma =0$};
% Text Node
\draw (531,108.6) node [anchor=south] [inner sep=0.75pt]    {$\text{Carnot Engine}$};
% Text Node
\draw (169,310) node [anchor=north] [inner sep=0.75pt]   [align=left] {Heat Pump};


\end{tikzpicture}

          \caption{Carnot Engine Feeding a Heat Pump}
          \label{fig:1}
        \end{figure}

    \end{enumerate}

  \item

    \begin{enumerate}

      \item 

      \item 

    \end{enumerate}

  \item

    \begin{enumerate}

      \item 
        
      \item 
        
      \item 
        
      \item 
        
    \end{enumerate}

    \setcounter{enumi}{4}

  \item

    First and foremost, we may write:

    $$W=(\tau_h-\tau_l)\sigma_l$$

    Given the assumption that the process is reversible, we may write:

    $$\sigma_l=\frac{Q_l}{\tau_l}$$

    This gives us:

    $$W=\left( \frac{\tau_h}{\tau_l}-1 \right)Q_l$$

    We can apply the given parameters to find:

    $$W=\left( \frac{500}{20}-1 \right)1500$$
    $$\boxed{W=36[\si{\giga\watt}]}$$

    Given improvements, we can write:

    $$W=\left( \frac{600}{20}-1 \right)1500$$
    $$\boxed{W=43.5[\si{\giga\watt}]}$$

    Clearly, there is nearly a 20\% increase in the ability to produce energy

  \item

    \begin{enumerate}

      \item 

        First and foremost, we may write:

        $$W=\left( 1-\frac{\tau_l}{\tau_h} \right)Q_h=\left( \frac{\tau_h}{\tau_l}-1 \right)Q_l$$

        Given that:

        $$\frac{dW}{dt}=P$$

        We may obtain:

        $$P=\left( \frac{\tau_h}{\tau_l}-1 \right)\frac{dQ_l}{dt}$$

        From the problem, we may substitute:

        $$P=A\left( \frac{\tau_h}{\tau_l}-1 \right)(\tau_h-\tau_l)$$

        Multiplying both sides by $\tau_l$ and dividing by $A$, we get:

        $$\frac{P\tau_l}{A}=(\tau_h-\tau_l)^2$$
        $$\frac{P\tau_l}{A}=\tau_h^2-2\tau_h\tau_l+\tau_l^2$$

        We can then move all terms to one side:

        $$\frac{P\tau_l}{A}+2\tau_h\tau_l-\tau_l^2-\tau_h^2=0$$
        $$\left(\frac{P}{A}+2\tau_h\right)\tau_l-\tau_l^2-\tau_h^2=0$$

        Employing the quadratic equation, we obtain:

        $$\tau_l^2-\left( \frac{P}{A}+2\tau_h \right)\tau_l+\tau_h=0\to \left\{\begin{array}{l} a=1\\b=-\left( \frac{P}{A}+2\tau_h \right)\\c=\tau_h^2\end{array}$$
        $$\tau_l=\frac{1}{2}\left( \frac{P}{A}+2\tau_h-\sqrt{\left( \frac{P}{A}+2\tau_h \right)^2-4\tau_h^2}\right)\text{\footnote{Note: we can forego the solution with a positive determinant, as that would make $\tau_l>\tau_h$}}$$

        Distributing, we finally obtain:

        $$\tau_l=\frac{P}{2A}+\tau_h-\left[ \left( \frac{P}{2A}+\tau_h \right)^2-\tau_h^2 \right]^{\frac{1}{2}}$$

        The Boltzmann factor cancels out, leaving us with:

        $$\boxed{T_l=\left(T_h+\frac{P}{2A}\right)-\left[ \left( T_h+\frac{P}{2A} \right)^2-T_h^2 \right]^{\frac{1}{2}}}$$

      \item 

        We can rearrange one of the formulas obtained in (a) to write:

        $$A=\frac{PT_l}{\left( T_h-T_l \right)^2}$$

        This gives us:

        $$A=\frac{(2\cdot10^3)(290)}{(20)^2}$$
        $$A=\frac{(580\cdot10^3)}{400}$$
        $$\boxed{A=1450\left[ \frac{\si{\watt}}{\si{\kelvin}} \right]}$$

    \end{enumerate}

  \item

    We know from (1) that:

    $$\frac{W}{Q_l}=\left( \frac{\tau_h}{\tau_l}-1 \right)$$

    And also that:

    $$W+Q_l=Q_h$$

    We can express the work done by the light bulb as $W=Q_l$. The total work of the two processes needs to equal zero, which allows us to write:

    $$\left( \frac{\tau_h}{\tau_l}-1 \right)Q_l-Q_l=0$$

    This simplifies to:

    $$\tau_h=2\tau_l$$

    Room temperature implies $27[\si{\celsius}]\to300[\si{\kelvin}]$

    Thus, the Carnot refrigerator, at peak efficiency, may cool to:

    $$\boxed{\tau_l=\frac{300}{2}=150[\si{\kelvin}]}$$

    Thus, the refrigerator should be able to cool below room temperature.

\end{enumerate}

\end{document}


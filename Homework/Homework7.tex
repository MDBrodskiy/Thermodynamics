%%%%%%%%%%%%%%%%%%%%%%%%%%%%%%%%%%%%%%%%%%%%%%%%%%%%%%%%%%%%%%%%%%%%%%%%%%%%%%%%%%%%%%%%%%%%%%%%%%%%%%%%%%%%%%%%%%%%%%%%%%%%%%%%%%%%%%%%%%%%%%%%%%%%%%%%%%%%%%%%%%%
% Written By Michael Brodskiy
% Class: Thermodynamics & Statistical Mechanics
% Professor: A. Stepanyants
%%%%%%%%%%%%%%%%%%%%%%%%%%%%%%%%%%%%%%%%%%%%%%%%%%%%%%%%%%%%%%%%%%%%%%%%%%%%%%%%%%%%%%%%%%%%%%%%%%%%%%%%%%%%%%%%%%%%%%%%%%%%%%%%%%%%%%%%%%%%%%%%%%%%%%%%%%%%%%%%%%%

\documentclass[12pt]{article} 
\usepackage{alphalph}
\usepackage[utf8]{inputenc}
\usepackage[russian,english]{babel}
\usepackage{titling}
\usepackage{amsmath}
\usepackage{graphicx}
\usepackage{enumitem}
\usepackage{amssymb}
\usepackage[super]{nth}
\usepackage{everysel}
\usepackage{ragged2e}
\usepackage{geometry}
\usepackage{multicol}
\usepackage{fancyhdr}
\usepackage{cancel}
\usepackage{siunitx}
\usepackage{physics}
\usepackage{tikz}
\usepackage{mathdots}
\usepackage{yhmath}
\usepackage{cancel}
\usepackage{color}
\usepackage{array}
\usepackage{multirow}
\usepackage{gensymb}
\usepackage{tabularx}
\usepackage{extarrows}
\usepackage{booktabs}
\usepackage{lastpage}
\usetikzlibrary{fadings}
\usetikzlibrary{patterns}
\usetikzlibrary{shadows.blur}
\usetikzlibrary{shapes}

\geometry{top=1.0in,bottom=1.0in,left=1.0in,right=1.0in}
\newcommand{\subtitle}[1]{%
  \posttitle{%
    \par\end{center}
    \begin{center}\large#1\end{center}
    \vskip0.5em}%

}
\usepackage{hyperref}
\hypersetup{
colorlinks=true,
linkcolor=blue,
filecolor=magenta,      
urlcolor=blue,
citecolor=blue,
}


\title{Homework 7}
\date{November 15, 2023}
\author{Michael Brodskiy\\ \small Professor: A. Stepanyants}

\begin{document}

\maketitle

\begin{enumerate}

  \item

    \begin{enumerate}

      \item 

        First and foremost, we know:

        $$Q_h=Q_l+W$$
        $$W=Q_l-Q_h$$

        Per reversible conditions:

        $$\sigma_h=\sigma_l$$
        $$\frac{Q_h}{\tau_h}=\frac{Q_l}{\tau_l}$$
        $$Q_h=\frac{\tau_hQ_l}{\tau_l}$$

        Thus, we can combine to write:

        $$W=Q_h-\frac{\tau_l}{\tau_h}Q_h$$
        $$W=Q_h\left( 1-\frac{\tau_l}{\tau_h}\right)$$
        $$W=Q_h\left( \frac{\tau_h-\tau_l}{\tau_h}\right)$$

        And finally:

        $$\boxed{\frac{W}{Q_h}=\frac{\tau_h-\tau_l}{\tau_h}}$$

        If the heat pump is not reversible, we know the efficiency is:

        $$\frac{W}{Q_h}<\frac{\tau_h-\tau_l}{\tau_h}$$

      \item 

      \item 

    \end{enumerate}

  \item

    \begin{enumerate}

      \item 

      \item 

    \end{enumerate}

  \item

    \begin{enumerate}

      \item 
        
      \item 
        
      \item 
        
      \item 
        
    \end{enumerate}

    \setcounter{enumi}{4}

  \item

    First and foremost, we may write:

    $$W=(\tau_h-\tau_l)\sigma_l$$

    Given the assumption that the process is reversible, we may write:

    $$\sigma_l=\frac{Q_l}{\tau_l}$$

    This gives us:

    $$W=\left( \frac{\tau_h}{\tau_l}-1 \right)Q_l$$

    We can apply the given parameters to find:

    $$W=\left( \frac{500}{20}-1 \right)1500$$
    $$\boxed{W=36[\si{\giga\watt}]}$$

    Given improvements, we can write:

    $$W=\left( \frac{600}{20}-1 \right)1500$$
    $$\boxed{W=43.5[\si{\giga\watt}]}$$

    Clearly, there is nearly a 20\% increase in the ability to produce energy

  \item

    \begin{enumerate}

      \item 

      \item 

    \end{enumerate}

  \item

    We know from (1) that:

    $$\frac{W}{Q_l}=\left( \frac{\tau_h}{\tau_l}-1 \right)$$

    And also that:

    $$W+Q_l=Q_h$$

    We can express the work done by the light bulb as $W=Q_l$. The total work of the two processes needs to equal zero, which allows us to write:

    $$\left( \frac{\tau_h}{\tau_l}=1 \right)Q_l-Q_l=0$$

    This simplifies to:

    $$\tau_h=2\tau_l$$

\end{enumerate}

\end{document}


%%%%%%%%%%%%%%%%%%%%%%%%%%%%%%%%%%%%%%%%%%%%%%%%%%%%%%%%%%%%%%%%%%%%%%%%%%%%%%%%%%%%%%%%%%%%%%%%%%%%%%%%%%%%%%%%%%%%%%%%%%%%%%%%%%%%%%%%%%%%%%%%%%%%%%%%%%%%%%%%%%%
% Written By Michael Brodskiy
% Class: Thermodynamics & Statistical Mechanics
% Professor: A. Stepanyants
%%%%%%%%%%%%%%%%%%%%%%%%%%%%%%%%%%%%%%%%%%%%%%%%%%%%%%%%%%%%%%%%%%%%%%%%%%%%%%%%%%%%%%%%%%%%%%%%%%%%%%%%%%%%%%%%%%%%%%%%%%%%%%%%%%%%%%%%%%%%%%%%%%%%%%%%%%%%%%%%%%%

\include{Includes.tex}

\title{Homework 8}
\date{November 22, 2023}
\author{Michael Brodskiy\\ \small Professor: A. Stepanyants}

\begin{document}

\maketitle

\begin{enumerate}

  \item

    \begin{enumerate}

      \item 

        We begin by implementing the definition of the Gibb's Free Energy:

        $$G=U-\tau\sigma+PV$$

        According to this, we know:

        $$\left( \frac{\partial G}{\partial \tau} \right)_{N,P}=-\sigma\quad\text{and}\quad\left( \frac{\partial G}{\partial P} \right)_\tau=V$$

        From here, we can obtain the first Maxwell relation since the order of partial differentiation should not matter. This gives us:

        $$\left( \frac{\partial ^2G}{\partial P\partial \tau} \right)_{\tau}=-\left(\frac{\partial \sigma}{\partial P}\right)_\tau\quad\text{and}\quad \left( \frac{\partial^2 G}{\partial\tau\partial P} \right)_P=\left( \frac{\partial V}{\partial \tau} \right)_P$$

        Setting the two together, we get:

        $$\boxed{\left(  \frac{\partial V}{\partial \tau}\right)_P=-\left( \frac{\partial\sigma}{\partial P} \right)_\tau}$$

        We repeat similar steps for the other Maxwell relations:

        $$\left( \frac{\partial G}{\partial N} \right)_P=\mu\quad\text{and}\quad\left( \frac{\partial G}{\partial V} \right)_\tau=V$$

        Now we differentiate once again:

        $$\left( \frac{\partial^2G}{\partial P\partial N} \right)_N=\left( \frac{\partial\mu}{\partial P} \right)_N\quad\text{and}\quad\left( \frac{\partial^2G}{\partial N\partial P} \right)_P=\left( \frac{\partial V}{\partial N} \right)_P$$

        Setting these together:

        $$\boxed{\left(  \frac{\partial V}{\partial N}\right)_P=\left( \frac{\partial\mu}{\partial P} \right)_N}$$

        Finally, we can write:

        $$\left( \frac{\partial^2G}{\partial \tau\partial N} \right)_N=\left( \frac{\partial\mu}{\partial \tau} \right)_N\quad\text{and}\quad\left( \frac{\partial^2G}{\partial N\partial \tau} \right)_\tau=-\left( \frac{\partial \sigma}{\partial N} \right)_\tau$$

        And then we obtain the final relation:

        $$\boxed{\left( \frac{\partial\mu}{\partial\tau} \right)_N=-\left( \frac{\partial\sigma}{\partial N} \right)_\tau}$$

      \item 

        First, we know:

        $$\alpha=\frac{1}{V}\left( \frac{\partial V}{\partial\tau} \right)_P$$

        From above, we may write:

        $$\alpha=-\frac{1}{V}\left( \frac{\partial \sigma}{\partial P} \right)_\tau$$

        By the third law of thermodynamics, we know that, as $\tau\to0$, $\sigma\to0$. Thus, we know that:

        $$\boxed{\alpha=-\frac{1}{V}\underbrace{\left( \frac{\partial \sigma}{\partial P} \right)_\tau}_0}$$
        $$\alpha=0\,\text{ as $\tau\to0$}$$

    \end{enumerate}

  \item

    \begin{enumerate}

      \item 

        From the law of mass action, we may write:

        $$\frac{[\ce{e^-}][\ce{H^+}]}{[\ce{H}]}=\prod_jn_{Qj}^{v_j}e^{-\frac{v_jF_{j,int}}{\tau}}$$

        From the product, we may write:

        $$K(\tau)=(n_{\ce{e^-}})e^{-\frac{F_{\ce{e^-},int}}{\tau}}\cdot(n_{\ce{H^+}})e^{-\frac{F_{\ce{H^+},int}}{\tau}}\cdot (n_{\ce{H}})^{-1}e^{\frac{F_{\ce{H},int}}{\tau}}$$

        We know that:

        $$F_{\ce{e^-},int}+F_{\ce{H^+},int}-F_{\ce{H},int}=I$$

        Summing the exponentials, we get:

        $$K(\tau)=\frac{(n_{\ce{e^-}})(n_{\ce{H^+}})}{(n_{\ce{H}})}e^{-\frac{I}{\tau}}$$

        Since $n_{\ce{e^-}}\approx n_Q$, and $n_{\ce{H^+}}\approx n_{\ce{H}}$, we can finally obtain:

        $$\boxed{\frac{[\ce{e^-}][\ce{H^+ }]}{[\ce{H}]}=n_Qe^{-\frac{I}{\tau}}}$$

      \item 

        First and foremost, we are given:

        $$[\ce{e^-}]=\left( [\ce{H}]n_Qe^{-\frac{I}{\tau}} \right)^{\frac{1}{2}}$$

        Since $H_{exc}$ is the first excited state, and $H$ is the ground state, we may write:

        $$\varepsilon_{H_{exc}}-\varepsilon_{H}=\frac{3}{4}I$$

        This gives us:

        $$\frac{[\ce{H_{exc}}]}{[\ce{H}]}=e^{-\frac{3I}{4\tau}}$$

        This, however is not entirely correct. We must also account for the quadruple-degeneracy (one 2$s$ orbital and three $2p$ orbitals) of the excited state:

        $$\frac{[\ce{H_{exc}}]}{[\ce{H}]}=4e^{-\frac{3I}{4\tau}}$$

        This means:

        $$[\ce{H_{exc}}]=4[\ce{H}]e^{-\frac{3I}{4\tau}}$$

        Now we can calculate and compare $[\ce{e^-}]$ and $[\ce{H_{exc}}]$ (note that, for working with $\si{\eV}$, $k_B=8.617\cdot10^{-5}[\si{\eV}/\si{\kelvin}]$):

        $$[\ce{e^-}]=\left( [\ce{H}]n_Qe^{-\frac{I}{\tau}} \right)^{\frac{1}{2}}$$
        $$[\ce{e^-}]=\left( 10^{23}n_Qe^{-\frac{13.6}{5000k_B}} \right)^{\frac{1}{2}}$$
        $$[\ce{e^-}]=44221.8\sqrt{n_Q}\left[ \frac{1}{\si{\milli\liter}} \right]$$

        \begin{center}
          and
        \end{center}

        $$[\ce{H_{exc}}]=4[\ce{H}]e^{-\frac{3I}{4\tau}}$$
        $$[\ce{H_{exc}}]=4\cdot10^{23}\cdote^{-\frac{3(13.6)}{4(5000)(8.617\cdot10^{-5})}}$$
        $$[\ce{H_{exc}}]=2.092\cdot10^{13}\left[ \frac{1}{\si{\milli\liter}} \right]$$

        Comparing the two, we find that the concentration of electrons relative to excited hydrogen is:

        $$\frac{[\ce{e^-}]}{[\ce{H_{exc}}]}=\frac{44221.8\sqrt{n_Q}}{2.092\cdot10^{13}}$$
        $$\boxed{\frac{[\ce{e^-}]}{[\ce{H_{exc}}]}=2.114\cdot10^{-9}\sqrt{n_Q}}$$

        There are a factor of $2.114\cdot10^{-9}\sqrt{n_Q}$ more electrons than excited hydrogen atoms.

    \end{enumerate}

  \item

    Employing the law of mass action, we can write:

    $$\frac{[\ce{e^-}][\ce{d^+}]}{[\ce{d}]}=n_Qe^{-\frac{I^*}{\tau}}$$

    To find $I^*$, we may write:

    $$I^*=\frac{I\left( \frac{m^*}{m} \right)}{\varepsilon^2}$$
    $$I^*=\frac{(13.6)(.3)}{\varepsilon^2}$$
    $$I^*=.03[\si{\eV}]$$

    We can also redefine the concentration as:

    $$n_Q=\left( 2.415\cdot10^{15} \right)\left[ \frac{m^*}{m}\cdot T \right]^{1.5}$$
    $$n_Q=\left( 2.415\cdot10^{15} \right)\left[ 30 \right]^{1.5}$$
    $$n_Q=3.97\cdot10^{17}\left[ \frac{1}{\si{\milli\liter}} \right]$$

    Given this, we may find:

    $$K=1.22\cdot10^{16}\left[ \frac{1}{\si{\milli\liter}} \right]$$

    Though we are given that there are $10^{17}$ donors, we can find $[\ce{d}]$ by using:

    $$[\ce{d}]=10^{17}-[\ce{d^+}]=10^{17}-[\ce{e^-}]$$

    Solving, we find:

    $$[\ce{e^-}]=\left[ \left( \frac{K}{2} \right)^2+K(10^{17}) \right]^{\frac{1}{2}}-\left( \frac{K}{2} \right)$$
    $$[\ce{e^-}]=\left[ \left( \frac{1.22\cdot10^{16}}{2} \right)^2+(1.22\cdot10^{16})(10^{17}) \right]^{\frac{1}{2}}-\left( \frac{1.22\cdot10^{16}}{2} \right)$$
    $$[\ce{e^-}]=2.94\cdot10^{16}\left[ \frac{1}{\si{\milli\liter}} \right]$$

    Thus, by the ratio of electrons to donors, we see that there is a 29.4\% ionization.

\end{enumerate}

\end{document}


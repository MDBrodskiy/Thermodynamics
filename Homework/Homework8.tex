%%%%%%%%%%%%%%%%%%%%%%%%%%%%%%%%%%%%%%%%%%%%%%%%%%%%%%%%%%%%%%%%%%%%%%%%%%%%%%%%%%%%%%%%%%%%%%%%%%%%%%%%%%%%%%%%%%%%%%%%%%%%%%%%%%%%%%%%%%%%%%%%%%%%%%%%%%%%%%%%%%%
% Written By Michael Brodskiy
% Class: Thermodynamics & Statistical Mechanics
% Professor: A. Stepanyants
%%%%%%%%%%%%%%%%%%%%%%%%%%%%%%%%%%%%%%%%%%%%%%%%%%%%%%%%%%%%%%%%%%%%%%%%%%%%%%%%%%%%%%%%%%%%%%%%%%%%%%%%%%%%%%%%%%%%%%%%%%%%%%%%%%%%%%%%%%%%%%%%%%%%%%%%%%%%%%%%%%%

\include{Includes.tex}

\title{Homework 8}
\date{November 22, 2023}
\author{Michael Brodskiy\\ \small Professor: A. Stepanyants}

\begin{document}

\maketitle

\begin{enumerate}

  \item

    \begin{enumerate}

      \item 

        We begin by implementing the definition of the Gibb's Free Energy:

        $$G=U-\tau\sigma+PV$$

        According to this, we know:

        $$\left( \frac{\partial G}{\partial \tau} \right)_{N,P}=-\sigma\quad\text{and}\quad\left( \frac{\partial G}{\partial P} \right)_\tau=V$$

        From here, we can obtain the first Maxwell relation since the order of partial differentiation should not matter. This gives us:

        $$\left( \frac{\partial ^2G}{\partial P\partial \tau} \right)_{\tau}=-\left(\frac{\partial \sigma}{\partial P}\right)_\tau\quad\text{and}\quad \left( \frac{\partial^2 G}{\partial\tau\partial P} \right)_P=\left( \frac{\partial V}{\partial \tau} \right)_P$$

        Setting the two together, we get:

        $$\boxed{\left(  \frac{\partial V}{\partial \tau}\right)_P=-\left( \frac{\partial\sigma}{\partial P} \right)_\tau}$$

        We repeat similar steps for the other Maxwell relations:

        $$\left( \frac{\partial G}{\partial N} \right)_P=\mu\quad\text{and}\quad\left( \frac{\partial G}{\partial V} \right)_\tau=V$$

        Now we differentiate once again:

        $$\left( \frac{\partial^2G}{\partial P\partial N} \right)_N=\left( \frac{\partial\mu}{\partial P} \right)_N\quad\text{and}\quad\left( \frac{\partial^2G}{\partial N\partial P} \right)_P=\left( \frac{\partial V}{\partial N} \right)_P$$

        Setting these together:

        $$\boxed{\left(  \frac{\partial V}{\partial N}\right)_P=\left( \frac{\partial\mu}{\partial P} \right)_N}$$

        Finally, we can write:

        $$\left( \frac{\partial^2G}{\partial \tau\partial N} \right)_N=\left( \frac{\partial\mu}{\partial \tau} \right)_N\quad\text{and}\quad\left( \frac{\partial^2G}{\partial N\partial \tau} \right)_\tau=-\left( \frac{\partial \sigma}{\partial N} \right)_\tau$$

        And then we obtain the final relation:

        $$\boxed{\left( \frac{\partial\mu}{\partial\tau} \right)_N=-\left( \frac{\partial\sigma}{\partial N} \right)_\tau}$$

      \item 

        First, we know:

        $$\alpha=\frac{1}{V}\left( \frac{\partial V}{\partial\tau} \right)_P$$

        From above, we may write:

        $$\alpha=-\frac{1}{V}\left( \frac{\partial \sigma}{\partial P} \right)_\tau$$

        By the third law of thermodynamics, we know that, as $\tau\to0$, $\sigma\to0$. Thus, we know that:

        $$\boxed{\alpha=-\frac{1}{V}\underbrace{\left( \frac{\partial \sigma}{\partial P} \right)_\tau}_0}$$
        $$\alpha=0\,\text{ as $\tau\to0$}$$

    \end{enumerate}

  \item

    \begin{enumerate}

      \item 

        From the law of mass action, we may write:

        $$\frac{[\ce{e^-}][\ce{H^+}]}{[\ce{H}]}=\prod_jn_{Qj}^{v_j}e^{-\frac{v_jF_{j,int}}{\tau}}$$

        From the product, we may write:

        $$K(\tau)=(n_{\ce{e^-}})e^{-\frac{F_{\ce{e^-},int}}{\tau}}\cdot(n_{\ce{H^+}})e^{-\frac{F_{\ce{H^+},int}}{\tau}}\cdot (n_{\ce{H}})^{-1}e^{\frac{F_{\ce{H},int}}{\tau}}$$

        We know that:

        $$F_{\ce{e^-},int}+F_{\ce{H^+},int}-F_{\ce{H},int}=I$$

        Summing the exponentials, we get:

        $$K(\tau)=\frac{(n_{\ce{e^-}})(n_{\ce{H^+}})}{(n_{\ce{H}})}e^{-\frac{I}{\tau}}$$

        Since $n_{\ce{e^-}}\approx n_Q$, and $n_{\ce{H^+}}\approx n_{\ce{H}}$, we can finally obtain:

        $$\boxed{\frac{[\ce{e^-}][\ce{H^+ }]}{[\ce{H}]}=n_Qe^{-\frac{I}{\tau}}}$$

      \item 

    \end{enumerate}

  \item

\end{enumerate}

\end{document}


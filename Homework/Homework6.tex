%%%%%%%%%%%%%%%%%%%%%%%%%%%%%%%%%%%%%%%%%%%%%%%%%%%%%%%%%%%%%%%%%%%%%%%%%%%%%%%%%%%%%%%%%%%%%%%%%%%%%%%%%%%%%%%%%%%%%%%%%%%%%%%%%%%%%%%%%%%%%%%%%%%%%%%%%%%%%%%%%%%
% Written By Michael Brodskiy
% Class: Thermodynamics & Statistical Mechanics
% Professor: A. Stepanyants
%%%%%%%%%%%%%%%%%%%%%%%%%%%%%%%%%%%%%%%%%%%%%%%%%%%%%%%%%%%%%%%%%%%%%%%%%%%%%%%%%%%%%%%%%%%%%%%%%%%%%%%%%%%%%%%%%%%%%%%%%%%%%%%%%%%%%%%%%%%%%%%%%%%%%%%%%%%%%%%%%%%

\include{Includes.tex}

\title{Homework 6}
\date{November 4, 2023}
\author{Michael Brodskiy\\ \small Professor: A. Stepanyants}

\begin{document}

\maketitle

\begin{enumerate}

    \setcounter{enumi}{1}

  \item

    \begin{enumerate}

      \item 

        We can find the density of states $d\varepsilon$ as:

        $$D(\varepsilon)\,d\varepsilon=\frac{V\varepsilon^2}{\pi^2\hbar^3c^3}\,d\varepsilon$$

        We can then find:

        $$N=\int_0^{\varepsilon_F}\frac{V\varepsilon^2}{\pi^2\hbar^3c^3}\,d\varepsilon$$
        $$N=\frac{V\varepsilon_F^3}{3\pi^2\hbar^3c^3}$$
        $$N(3\pi^2\hbar^3c^3)=V\varepsilon_F^3$$
        $$\varepsilon_F^3=\frac{N}{V}(3\pi^2\hbar^3c^3)$$
        $$\varepsilon_F^3=3n(\pi^2\hbar^3c^3)$$
        $$\varepsilon_F^3=\frac{3n}{\pi}(\pi\hbar c)^3$$

        And finally:

        $$\boxed{\varepsilon_F=\left(\frac{3n}{\pi}\right)^{\frac{1}{3}}(\pi\hbar c)}$$

      \item 

        The energy can be defined in a similar manner:

        $$U=\int_0^{\varepsilon_F}\varepsilon D(\varepsilon)\,d\varepsilon$$

        Which gives us:

        $$U=\int_0^{\varepsilon_F}\frac{V\varepsilon^3}{\pi^2\hbar^3c^3}\,d\varepsilon$$
        $$U=\frac{V\varepsilon_F^4}{4\pi^2\hbar^3c^3}$$
        $$U=\left(\frac{V\varepsilon_F^3}{3\pi^2\hbar^3c^3}\right)\frac{3}{4}\varepsilon_F$$

        And finally:

        $$\boxed{U_o=\frac{3}{4}N\varepsilon_F}$$

    \end{enumerate}

  \item

    \begin{enumerate}

      \item 

        We know that the ground state energy may be expressed as:

        $$U_o=\frac{3N\hbar^2}{10M}\left(\frac{3\pi^2N}{V}\right)^{\frac{2}{3}}$$

        We also know the relation:

        $$P=-\frac{\partial U}{\partial V}$$

        This gives us:

        $$P=-\left( -\frac{2}{3}\frac{3\hbar^2(3\pi^2)^{\frac{2}{3}}}{10M}\left(\frac{N}{V}\right)^{\frac{5}{3}} \right)$$
        $$P=\frac{\hbar^2(3\pi^2)^{\frac{2}{3}}}{5M}\left(\frac{N}{V}\right)^{\frac{5}{3}}$$
        $$\boxed{P=\frac{(3\pi^2)^{\frac{2}{3}}}{5}\frac{\hbar^2}{M}\left(\frac{N}{V}\right)^{\frac{5}{3}}}$$

      \item 

        We can define:

        $$\delta\sigma(\tau)=\int\frac{1}{\tau}\,dU$$

        We know the heat capacity of an electron gas as:

        $$C_V=\frac{1}{2}\pi^2N\left( \frac{\tau}{\tau_F} \right)$$

        We know the heat capacity is equal to $\frac{\partial U}{\partial \tau}$, which allows us to write:

        $$\sigma(\tau)=\int_0^\tau \frac{1}{\tau}C_v\,d\tau$$
        $$\sigma(\tau)=\int_0^\tau \frac{1}{2}\pi^2N\frac{1}{\tau_F}\,d\tau$$
        $$\boxed{\sigma(\tau)=\frac{1}{2}\pi^2N\frac{\tau}{\tau_F}}$$

      Notice: $\sigma=C_V$, as is correct for a degenerate gas

    \end{enumerate}

    \setcounter{enumi}{4}

  \item

    \begin{enumerate}

      \item We can describe $n=\frac{N}{V}$ as:

        $$n=\frac{\rho}{M}=\frac{.081}{3\cdot1.67\cdot10^{-24}}$$
        $$n=1.617\cdot10^{22}\left[ \si{\per\centi\meter\cubed} \right]$$
        $$n=1.617\cdot10^{28}\left[ \si{\per\meter\cubed} \right]$$

        We can then use the formula:

        $$\varepsilon_F=\frac{\hbar^2}{2M}\left( 3\pi^2 n) \right)^{\frac{2}{3}}$$
        $$\varepsilon_F=\frac{(1.055\cdot10^{-34})^2}{2\cdot3\cdot1.67\cdot10^{-27}}\left( 3\pi^2 (1.617\cdot10^{28}) \right)^{\frac{2}{3}}$$
        $$\varepsilon_F=6.79\cdot10^{-23}[\si{\joule}]$$
        $$\boxed{\varepsilon_F=4.24\cdot10^{-4}[\si{\eV}]}$$

        Now, we can find $v_F$ as:

        $$\varepsilon_F=\frac{1}{2}Mv_F^2$$
        $$v_F=\sqrt{\frac{2\varepsilon_F}{M}}$$
        $$v_F=\sqrt{\frac{2(6.79\cdot10^{-23})}{3\cdot1.67\cdot10^{-27}}}$$
        $$\boxed{v_F=164.638\left[ \frac{\si{\meter}}{\si{\second}} \right]}$$

        And, finally, $T_F$:

        $$T_F=\frac{\tau_F}{k_B}=\frac{\varepsilon_F}{k_B}$$
        $$T_F=\frac{6.79\cdot10^{-23}}{1.381\cdot10^{-23}}$$
        $$T_F=\frac{6.79}{1.381}$$
        $$\boxed{T_F=4.92[\si{\kelvin}]}$$

      \item 

        We may begin by using:

        $$C_V=\frac{\pi^2N}{2}\frac{\tau}{\tau_F}$$
        $$C_V=\frac{\pi^2}{2\tau_F}(N\tau)$$
        $$C_V=\frac{2(\pi^2)}{(4.92)}(Nk_BT)$$
        $$\boxed{C_V=1.003Nk_BT}$$

        The coefficient $1.003<2.89$, the experimental value.

    \end{enumerate}

  \item

    \begin{enumerate}

      \item 

        The energy may be defined as:

        $$U=-G\int_0^R\frac{\rho^2(4\pi r^2)\left( \frac{4}{3}\pi r^3 \right)}{r}\,dr$$
        $$U=-\rho^2G\int_0^R\frac{(4\pi)^2r^5}{3r}\,dr$$
        $$U=-\frac{16\pi^2\rho^2G}{3}\int_0^R r^4\,dr$$
        $$U=-\frac{16\pi^2\rho^2G}{3}\frac{R^5}{5}$$
        $$U=-\frac{16\pi^2\rho^2GR^5}{15}$$

        We know the density becomes $\rightarrow\dfrac{3M}{4\pi R^3}$, which gives:

        $$\boxed{U=-\frac{3M^2G}{5R}}$$

        Thus, we can see this is on the order of $\dfrac{GM^2}{R}$

      \item 

        The Fermi energy may be defined as:

        $$\varepsilon_F=\frac{\hbar^2}{2M}(3\pi^2n)^{\frac{2}{3}}$$

        And the volume is:

        $$V=\frac{4}{3}\pi R^3$$

        We can write the total energy as:

        $$NK_e=N\varepsilon_F$$
        $$K_{tot}=N\frac{\hbar^2}{2m}\left(3\pi^2\frac{N}{V}\right)^{\frac{2}{3}}$$
        $$K_{tot}=\frac{\hbar^2}{2m}\left(\frac{9\pi^2}{4\pi R^3}\right)^{\frac{2}{3}}N^{\frac{5}{3}}$$
        $$K_{tot}=\frac{\hbar^2}{2mR^2}\left(\frac{9}{4}\pi\right)^{\frac{2}{3}}N^{\frac{5}{3}}$$
        $$K_{tot}=\frac{\hbar^2}{mR^2}\left(\frac{81}{128}\pi^2\right)^{\frac{2}{3}}N^{\frac{5}{3}}$$

        This can then be written:

        $$K_{tot}=c\frac{\hbar^2}{mR^2}N^{\frac{5}{3}}$$

        with $c=\left(\dfrac{81}{128}\pi^2\right)^{\frac{2}{3}}$. Now, since $M_H<<M$, we can write $N=\dfrac{M}{M_H}$, which gives us:

        $$\boxed{K_{tot}=c\frac{\hbar^2M^{\frac{5}{3}}}{mM_H^{\frac{5}{3}}R^2}}$$

        Thus, we can see the momentum is on the order of  $\dfrac{\hbar^2M^{\frac{5}{3}}}{mM_H^{\frac{5}{3}}R^2}$

      \item 

        From the two results above, we may write:

        $$\frac{\hbar^2M^{\frac{5}{3}}}{mM_H^{\frac{5}{3}}R^2}\approx\frac{GM^2}{R}$$

        Rearranging, we get:

        $$\frac{\hbar^2}{GmM_H^{\frac{5}{3}}}\approx M^{\frac{1}{3}}R$$

        Now we can calculate:

        $$M^{\frac{1}{3}}R\approx\frac{\hbar^2}{GmM_H^{\frac{5}{3}}}$$

        We will have two different values for $\hbar$, one in electron-volts, and one in joules to verify the correct dimensionality with respect to $G$ and $m$:

        $$M^{\frac{1}{3}}R\approx\frac{(6.582\cdot10^{-22})(1.055\cdot10^{-34})c^2}{(6.67\cdot10^{-11})[.511](1.67\cdot10^{-24})^{\frac{5}{3}}}$$
        $$M^{\frac{1}{3}}R\approx 7.8\cdot10^{11}$$

        Note: this result does not have the correct units. Correcting this, we find:

        $$7.8\cdot10^{11}(10^8)=7.8\cdot10^{19}$$

        Then multiplying by the constants/coefficients removed in (a) and (b), we get:

        $$7.8\cdot10^{19}\cdot\left( \frac{81\pi^2}{128} \right)^{\frac{2}{3}}\frac{5}{3}=4.4\cdot10^{20}$$

        Thus, we can see that the value is on the order of

        $$\boxed{M^{\frac{1}{3}}R\to 10^{20}[\si{\gram^{\frac{1}{3}}\centi\meter}]}$$

      \item 

        We know:

        $$\rho=\frac{3M}{4\pi R^3}$$
        $$\rho=\frac{3M}{4\pi \left( \frac{\cdot10^{20}}{M} \right)}$$
        $$\rho=\frac{3M^2}{4\pi \left( 10^{60} \right)}$$
        $$\rho=\frac{3(2\cdot10^{66})}{4\pi \left( 10^{60} \right)}$$
        $$\boxed{\rho=4.77465\cdot10^5\left[ \frac{\si{\gram}}{\si{\centi\meter\cubed}} \right]}$$

      \item 

        The only difference between a proton and electron gas would be the energy per $c^2$, which, for neutrons, is on the order of $10^9$ (giga) instead of $10^6$ (mega). Thus, $M^{\frac{1}{3}}R$ would be divided by a factor of $10^3$:

        $$10^{20}(10^{-3})=10^{17}[\si{\gram^{\frac{1}{3}}\centi\meter}]$$

        We can then write:

        $$R=\frac{10^{17}}{(2\cdot10^{33})^{\frac{1}{3}}}=793701[\si{\centi\meter}]$$
        $$\boxed{R=7.937[\si{\kilo\meter}]}$$

    \end{enumerate}

  \item

    Since we want to find the critical temperature at which $N_e<N$, we can define:

    $$N=N_e=\frac{2.404V\tau^3}{\pi^2\hbar^3c^3}$$

    Which can be rearranged to give:

    $$\tau=\left( \frac{N\pi^2\hbar^3c^3}{2.404V} \right)^{\frac{1}{3}}$$

    We can convert:

    $$10^{20}[\si{\per\centi\meter\cubed}]\to10^{26}[\si{\per\meter\cubed}]$$

    We can then calculate:

    $$k_BT=\left( \frac{\pi^2(1.055\cdot10^{-34})^3(3\cdot10^8)^3}{2.404}10^{26} \right)^{\frac{1}{3}}$$
    $$k_BT=2.35\cdot10^{-17}$$
    $$\boxed{T=1.7\cdot10^6[\si{\kelvin}]}$$

    Below this temperature, $N_e<N$

\end{enumerate}

\end{document}


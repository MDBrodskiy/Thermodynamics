%%%%%%%%%%%%%%%%%%%%%%%%%%%%%%%%%%%%%%%%%%%%%%%%%%%%%%%%%%%%%%%%%%%%%%%%%%%%%%%%%%%%%%%%%%%%%%%%%%%%%%%%%%%%%%%%%%%%%%%%%%%%%%%%%%%%%%%%%%%%%%%%%%%%%%%%%%%%%%%%%%%
% Written By Michael Brodskiy
% Class: Thermodynamics & Statistical Mechanics
% Professor: A. Stepanyants
%%%%%%%%%%%%%%%%%%%%%%%%%%%%%%%%%%%%%%%%%%%%%%%%%%%%%%%%%%%%%%%%%%%%%%%%%%%%%%%%%%%%%%%%%%%%%%%%%%%%%%%%%%%%%%%%%%%%%%%%%%%%%%%%%%%%%%%%%%%%%%%%%%%%%%%%%%%%%%%%%%%

\documentclass[12pt]{article} 
\usepackage{alphalph}
\usepackage[utf8]{inputenc}
\usepackage[russian,english]{babel}
\usepackage{titling}
\usepackage{amsmath}
\usepackage{graphicx}
\usepackage{enumitem}
\usepackage{amssymb}
\usepackage[super]{nth}
\usepackage{everysel}
\usepackage{ragged2e}
\usepackage{geometry}
\usepackage{multicol}
\usepackage{fancyhdr}
\usepackage{cancel}
\usepackage{siunitx}
\usepackage{physics}
\usepackage{tikz}
\usepackage{mathdots}
\usepackage{yhmath}
\usepackage{cancel}
\usepackage{color}
\usepackage{array}
\usepackage{multirow}
\usepackage{gensymb}
\usepackage{tabularx}
\usepackage{extarrows}
\usepackage{booktabs}
\usepackage{lastpage}
\usetikzlibrary{fadings}
\usetikzlibrary{patterns}
\usetikzlibrary{shadows.blur}
\usetikzlibrary{shapes}

\geometry{top=1.0in,bottom=1.0in,left=1.0in,right=1.0in}
\newcommand{\subtitle}[1]{%
  \posttitle{%
    \par\end{center}
    \begin{center}\large#1\end{center}
    \vskip0.5em}%

}
\usepackage{hyperref}
\hypersetup{
colorlinks=true,
linkcolor=blue,
filecolor=magenta,      
urlcolor=blue,
citecolor=blue,
}


\title{Homework 6}
\date{November 4, 2023}
\author{Michael Brodskiy\\ \small Professor: A. Stepanyants}

\begin{document}

\maketitle

\begin{enumerate}

    \setcounter{enumi}{1}

  \item

    \begin{enumerate}

      \item 

        We can find the density of states $d\varepsilon$ as:

        $$D(\varepsilon)\,d\varepsilon=\frac{V\varepsilon^2}{\pi^2\hbar^3c^3}\,d\varepsilon$$

        We can then find:

        $$N=\int_0^{\varepsilon_F}\frac{V\varepsilon^2}{\pi^2\hbar^3c^3}\,d\varepsilon$$
        $$N=\frac{V\varepsilon_F^3}{3\pi^2\hbar^3c^3}$$
        $$N(3\pi^2\hbar^3c^3)=V\varepsilon_F^3$$
        $$\varepsilon_F^3=\frac{N}{V}(3\pi^2\hbar^3c^3)$$
        $$\varepsilon_F^3=3n(\pi^2\hbar^3c^3)$$
        $$\varepsilon_F^3=\frac{3n}{\pi}(\pi\hbar c)^3$$

        And finally:

        $$\boxed{\varepsilon_F=\left(\frac{3n}{\pi}\right)^{\frac{1}{3}}(\pi\hbar c)}$$

      \item 

        The energy can be defined in a similar manner:

        $$U=\int_0^{\varepsilon_F}\varepsilon D(\varepsilon)\,d\varepsilon$$

        Which gives us:

        $$U=\int_0^{\varepsilon_F}\frac{V\varepsilon^3}{\pi^2\hbar^3c^3}\,d\varepsilon$$
        $$U=\frac{V\varepsilon_F^4}{4\pi^2\hbar^3c^3}$$
        $$U=\left(\frac{V\varepsilon_F^3}{3\pi^2\hbar^3c^3}\right)\frac{3}{4}\varepsilon_F$$

        And finally:

        $$\boxed{U_o=\frac{3}{4}N\varepsilon_F}$$

    \end{enumerate}

  \item

    \begin{enumerate}

      \item 

        We know that the ground state energy may be expressed as:

        $$U_o=\frac{3N\hbar^2}{10M}\left(\frac{3\pi^2N}{V}\right)^{\frac{2}{3}}$$

        We also know the relation:

        $$P=-\frac{\partial U}{\partial V}$$

        This gives us:

        $$P=-\left( -\frac{2}{3}\frac{3\hbar^2(3\pi^2)^{\frac{2}{3}}}{10M}\left(\frac{N}{V}\right)^{\frac{5}{3}} \right)$$
        $$P=\frac{\hbar^2(3\pi^2)^{\frac{2}{3}}}{5M}\left(\frac{N}{V}\right)^{\frac{5}{3}}$$
        $$\boxed{P=\frac{(3\pi^2)^{\frac{2}{3}}}{5}\frac{\hbar^2}{M}\left(\frac{N}{V}\right)^{\frac{5}{3}}}$$

      \item 

        We can define:

        $$\delta\sigma(\tau)=\int\frac{1}{\tau}\,dU$$

        We know the heat capacity of an electron gas as:

        $$C_V=\frac{1}{2}\pi^2N\left( \frac{\tau}{\tau_F} \right)$$

        We know the heat capacity is equal to $\frac{\partial U}{\partial \tau}$, which allows us to write:

        $$\sigma(\tau)=\int_0^\tau \frac{1}{\tau}C_v\,d\tau$$
        $$\sigma(\tau)=\int_0^\tau \frac{1}{2}\pi^2N\frac{1}{\tau_F}\,d\tau$$
        $$\boxed{\sigma(\tau)=\frac{1}{2}\pi^2N\frac{\tau}{\tau_F}}$$

      Notice: $\sigma=C_V$

    \end{enumerate}

    \setcounter{enumi}{4}

  \item

    \begin{enumerate}

      \item 

      \item 

    \end{enumerate}

  \item

    \begin{enumerate}

      \item 

      \item 

      \item 

      \item 

      \item 

    \end{enumerate}

  \item

\end{enumerate}

\end{document}


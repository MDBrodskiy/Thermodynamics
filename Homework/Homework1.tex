%%%%%%%%%%%%%%%%%%%%%%%%%%%%%%%%%%%%%%%%%%%%%%%%%%%%%%%%%%%%%%%%%%%%%%%%%%%%%%%%%%%%%%%%%%%%%%%%%%%%%%%%%%%%%%%%%%%%%%%%%%%%%%%%%%%%%%%%%%%%%%%%%%%%%%%%%%%%%%%%%%%
% Written By Michael Brodskiy
% Class: Thermodynamics & Statistical Mechanics
% Professor: A. Stepanyants
%%%%%%%%%%%%%%%%%%%%%%%%%%%%%%%%%%%%%%%%%%%%%%%%%%%%%%%%%%%%%%%%%%%%%%%%%%%%%%%%%%%%%%%%%%%%%%%%%%%%%%%%%%%%%%%%%%%%%%%%%%%%%%%%%%%%%%%%%%%%%%%%%%%%%%%%%%%%%%%%%%%

\documentclass[12pt]{article} 
\usepackage{alphalph}
\usepackage[utf8]{inputenc}
\usepackage[russian,english]{babel}
\usepackage{titling}
\usepackage{amsmath}
\usepackage{graphicx}
\usepackage{enumitem}
\usepackage{amssymb}
\usepackage[super]{nth}
\usepackage{everysel}
\usepackage{ragged2e}
\usepackage{geometry}
\usepackage{multicol}
\usepackage{fancyhdr}
\usepackage{cancel}
\usepackage{siunitx}
\usepackage{physics}
\usepackage{tikz}
\usepackage{mathdots}
\usepackage{yhmath}
\usepackage{cancel}
\usepackage{color}
\usepackage{array}
\usepackage{multirow}
\usepackage{gensymb}
\usepackage{tabularx}
\usepackage{extarrows}
\usepackage{booktabs}
\usepackage{lastpage}
\usetikzlibrary{fadings}
\usetikzlibrary{patterns}
\usetikzlibrary{shadows.blur}
\usetikzlibrary{shapes}

\geometry{top=1.0in,bottom=1.0in,left=1.0in,right=1.0in}
\newcommand{\subtitle}[1]{%
  \posttitle{%
    \par\end{center}
    \begin{center}\large#1\end{center}
    \vskip0.5em}%

}
\usepackage{hyperref}
\hypersetup{
colorlinks=true,
linkcolor=blue,
filecolor=magenta,      
urlcolor=blue,
citecolor=blue,
}


\title{Homework 1}
\date{\today}
\author{Michael Brodskiy\\ \small Professor: A. Stepanyants}

\begin{document}

\maketitle

\begin{enumerate}

  \item

    \begin{enumerate}

      \item 

        We are given the following:

        $$g(U)=CU^{\frac{3}{2}N}$$

        We know, by definition, that $\sigma(U)=\ln(g(U))$. Thus, we may write:

        $$\sigma(U)=\ln(CU^{\frac{3}{2}N})$$

        By properties of logarithms, we may expand this to:

        $$\sigma(U)=\ln(C)+\frac{3N}{2}\ln(U)$$

        Furthermore, by definition of the fundamental temperature, we know that $\displaystyle\frac{\partial\sigma(U)}{\partial U}=\frac{1}{\tau}$. As such, we obtain:

        $$\frac{\partial \sigma(U)}{\partial U}=\frac{3N}{2U}$$

        Substituting and rearranging, we get:

        $$\frac{1}{\tau}=\frac{3N}{2U}$$
        $$\boxed{U=\frac{3N}{2}\tau}$$

      \item 

        If we return to the step before converting $\displaystyle\frac{\partial\sigma(U)}{\partial U}$ to $\frac{1}{\tau}$, we can differentiate once more:

        $$\frac{\partial \sigma(U)}{\partial U}=\frac{3N}{2U}$$
        $$\left(\frac{\partial^2\sigma(U)}{\partial U^2}\right)_N=\frac{3N}{2}\frac{\partial}{\partial U}\left( \frac{1}{U} \right)$$
        $$\boxed{\left(\frac{\partial^2\sigma(U)}{\partial U^2}\right)_N=-\frac{3N}{U^2}}$$

        Thus, we see this results in a function with a negative coefficient. Furthermore, because we know the quantity of particles can not be negative, and the square of a number can not be negative, the second order partial derivative must always be negative.

    \end{enumerate}

  \item

    We know, by definition:

    $$U=-(2s)mB$$

    Rearranging, we see that:

    $$s=-\frac{U}{2mB}$$

    We also know that:

    $$g(N,s)=\undebrace{2^N\sqrt{\frac{2}{\pi N}}}_{g(N,0)}e^{-\frac{2s^2}{N}}$$

    Substituting the value of $s$ above, we get:

    $$g(N,s)=\underbrace{2^N\sqrt{\frac{2}{\pi N}}}_{g(N,0)}e^{-\frac{U^2}{2m^2B^2N}}$$

    Finding the entropy, the resulting function looks as follows:

    $$\sigma(N,s)=\left( -\frac{U^2}{2m^2B^2N} \right)+\underbrace{\ln\left(2^N\sqrt{\frac{2}{\pi N}}\right)}_{\sigma_o}$$

    Now finding the fundamental temperature, we differentiate with respect to $U$:

    $$\frac{\partial \sigma(N,s)}{\partial U}=\left(\left( -\frac{U^2}{2m^2B^2N} \right)+\underbrace{\ln\left(2^N\sqrt{\frac{2}{\pi N}}}_{\sigma_o}\right)\right)_{N,s}=-\frac{U}{m^2B^2N}$$

    Now assuming that $U$ is the average thermal energy, we get:

    $$U=-2mB\langle s\rangle$$

    We then return this to our function:

    $$\frac{1}{\tau}=\frac{2\langle s\rangle}{mBN}$$
    $$\tau=\frac{mBN}{2\langle s\rangle}$$
    
  \item

    \begin{enumerate}

      \item 

        From 1.55, we know:

        $$g(N,n)=\frac{(N+n-1)!}{n!(N-1)!}$$

        Thus, by definition of entropy, we know:

        $$\sigma(N,n)=\ln\left( \frac{(N+n-1)!}{n!(N-1)!} \right)\rightarrow\ln( (N+n-1)!)-\ln(n!)-\ln ((N-1)!)$$

        Replacing $N-1$ with $N$, we get:

        $$\ln( (N+n)!)-\ln(n!)-\ln(N!)$$

        Through the Stirling approximation, we get:

        $$(N+n)\ln(N+n)-(N+n)-n\ln(n)+n-N\ln(N)+N$$

        This can be simplified to:

        $$\boxed{\sigma(N,n)=(N+n)\ln(N+n)-n\ln(n)-N\ln(N)}$$

      \item 

        We are given $U=n\hbar\omega$, which means $n=\frac{U}{\hbar\omega}$. Thus, we substitute to get:

        $$\sigma\left(N,\frac{U}{\hbar\omega}\right)=\left(N+\frac{U}{\hbar\omega}\right)\ln\left( N+\frac{U}{\hbar\omega} \right)-\frac{U}{\hbar\omega}\ln\left( \frac{U}{\hbar\omega} \right)-N\ln(N)$$

        By definition of the fundamental temperature, we get:

        $$\frac{1}{\tau}=\frac{\partial}{\partial U}\left(  \left(N+\frac{U}{\hbar\omega}\right)\ln\left( N+\frac{U}{\hbar\omega} \right)-\frac{U}{\hbar\omega}\ln\left( \frac{U}{\hbar\omega} \right)-N\ln(N)\right)$$
        $$\frac{1}{\tau}=\left( \frac{1}{\hbar\omega}\right)\ln\left( N+\frac{U}{\hbar\omega} \right)-\frac{1}{\hbar\omega}\ln\left( \frac{U}{\hbar\omega} \right)$$
        $$\frac{1}{\tau}=\frac{1}{\hbar\omega}\ln\left( \frac{N\hbar\omega}{U}+1 \right)$$

        Rearranging for $U$, we get:

        $$e^{\frac{\hbar\omega}{\tau}}-1=\frac{N\hbar\omega}{U}$$

        And finally:

        $$\boxed{U=\frac{N\hbar\omega}{e^{\frac{\hbar\omega}{\tau}}-1}}$$

    \end{enumerate}
    
  \item

    \begin{enumerate}

      \item 

        Given that there is only one correct key out of the 44 possibilities per press, we know that the possibility of a single key being correct is:

        $$\frac{1}{44}\approx.0227$$

        Upon repeating this for a sequence of $10^5$ characters, this probability becomes:

        $$\left( \frac{1}{44} \right)^{100,000}\approx 10^{-164,345}$$

      \item 

        From the given values, we know the typing speed is 10 keys per second, and the age of the universe is $10^{18}$ seconds. Thus, we get:

        $$10^{18}(10)=10^{19}[\text{keys}]$$

        Given that there are $10^{10}$ monkeys, we can multiply the numbers out to produce the total keys pressed:

        $$\left(10^{19}\right)\left( 10^{10} \right)=10^{29}[\text{keys total}]$$

        Then, we also know that, per one hamlet, there are $10^5$ characters. This yields:

        $$10^{29}\left( \frac{1}{10^5} \right)=10^{24}[\text{monkey-Hamlets}]$$

        Finally, given are probability from above, we obtain:

        $$\left(10^{24}\right)\left( 10^{-164,345} \right)=10^{-164,316}[\text{probability of Hamlet in given time}]$$

    \end{enumerate}
    
  \item

    \begin{enumerate}

      \item

      \item 

      \item 

    \end{enumerate}
    
  \item
    
\end{enumerate}

\end{document}


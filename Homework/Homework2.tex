%%%%%%%%%%%%%%%%%%%%%%%%%%%%%%%%%%%%%%%%%%%%%%%%%%%%%%%%%%%%%%%%%%%%%%%%%%%%%%%%%%%%%%%%%%%%%%%%%%%%%%%%%%%%%%%%%%%%%%%%%%%%%%%%%%%%%%%%%%%%%%%%%%%%%%%%%%%%%%%%%%%
% Written By Michael Brodskiy
% Class: Thermodynamics & Statistical Mechanics
% Professor: A. Stepanyants
%%%%%%%%%%%%%%%%%%%%%%%%%%%%%%%%%%%%%%%%%%%%%%%%%%%%%%%%%%%%%%%%%%%%%%%%%%%%%%%%%%%%%%%%%%%%%%%%%%%%%%%%%%%%%%%%%%%%%%%%%%%%%%%%%%%%%%%%%%%%%%%%%%%%%%%%%%%%%%%%%%%

\include{Includes.tex}

\title{Homework 2}
\date{September 30, 2023}
\author{Michael Brodskiy\\ \small Professor: A. Stepanyants}

\begin{document}

\maketitle

\begin{enumerate}

  \item

    \begin{enumerate}

      \item 

        First, we know that:

        $$z(\varepsilon)=e^{-\frac{\varepsilon}{\tau}}$$

        Plugging in the respective values of $\varepsilon=0$ and $\varepsilon=\varepsilon_o$:

        $$z=z(0)+z(\varepsilon_o)=1+e^{-\frac{\varepsilon_o}{\tau}}$$

        Free energy can also be expressed in terms of $z$:

        $$\boxed{F=-\tau\ln(z)\rightarrow-\tau\ln\left( 1+e^{\frac{-\varepsilon_o}{\tau}}\right)}$$

      \item 

        We know the energy may be expressed as:

        $$U=-\tau^2\frac{\partial}{\partial \tau}\left( \frac{F}{\tau} \right)$$
        $$=\tau^2\frac{\partial}{\partial \tau}\left( \ln\left( 1+e^{-\frac{\varepsilon_o}{\tau}} \right) \right)$$
        $$=\tau^2\left( \frac{\varepsilon_oe^{-\frac{\varepsilon_o}{\tau}}}{\tau^2\left(1+e^{-\frac{\varepsilon_o}{\tau}}}\right) \right)$$
        $$=\frac{\varepsilon_oe^{-\frac{\varepsilon_o}{\tau}}}{\left(1+e^{-\frac{\varepsilon_o}{\tau}}}\right)$$

        Furthermore, entropy may be expressed as:

        $$\sigma=-\frac{\partial F}{\partial \tau}$$
        $$=\frac{\partial}{\partial \tau}\left( \tau\ln\left( 1+e^{-\frac{\varepsilon_o}{\tau}} \right) \right)$$
        $$\boxed{=\ln\left( 1+e^{-\frac{\varepsilon_o}{\tau}}\right) + \frac{\varepsilon_oe^{-\frac{\varepsilon_o}{\tau}}}{\tau\left(1+e^{-\frac{\varepsilon_o}{\tau}}}\right)}$$

    \end{enumerate}

  \item

    \begin{enumerate}

      \item

        We need to calculate the multiplicity of the $N$-spin system. First and foremost, we know:

        $$U=-2msB$$

        We also know the $N$-spin system can be rewritten using:

        $$2s=N_{\uparrow}-N_{\downarrow}=N_{\uparrow}-(N-N_{\uparrow})=2N_{\uparrow}-N$$

        This yields:

        $$U=(N-2N_{\uparrow})mB$$

        We then need to form a sum to find the partition function. We can do this by knowing the boundaries of $s$:

        $$-\frac{N}{2}\leq s\leq \frac{N}{2}$$

        $$\sum_{s=-\frac{N}{2}}^{\frac{N}{2}}$$

        We need to multiply the exponential term by $N$ choose $N_{\uparrow}:$

        $$\sum_{s=-\frac{N}{2}}^{\frac{N}{2}}\left( ^N_{N_{\uparrow}} \right)e^{\frac{2smB}{\tau}}$$

        Since we know $s$ can be expressed as $N_{\uparrow}-\frac{N}{2}$, $N_{\uparrow}$ can be expressed as $s+\frac{N}{2}$; furthermore, $N_{\downarrow}$ can be expressed as $\frac{N}{2}-s$, which gives us:

        $$z=\sum_{s=-\frac{N}{2}}^{\frac{N}{2}}\frac{N!}{(\frac{N}{2}+s)!(\frac{N}{2}-s)!}e^{\frac{2smB}{\tau}}$$

        To simplify, we can shift the boundaries by taking $s\to s-\frac{N}{2}$

        $$z=\sum_{s=0}^{N}\frac{N!}{s!(N-s)!}e^{\frac{2mB}{\tau}\left( s-\frac{N}{2} \right)}$$
        $$z=\sum_{s=0}^{N}\frac{N!}{s!(N-s)!}e^{\frac{2smB}{\tau}-\frac{NmB}{\tau} \right)}$$

        We then use $\sum_j^N\left( ^N_j \right)x^j=(1+x)^N$, dropping $s=0$ because the term is extremely small:

        $$e^{-\frac{NmB}{\tau}}(1+e^{\frac{2mB}{\tau}})^N=(e^{-\frac{mB}{\tau}}+e^{\frac{mB}{\tau}})^N=2^N\cosh^N\left( \frac{mB}{\tau} \right)$$

        We then differentiate $z$ with respect to $\tau$ to get:

        $$\frac{\partial z}{\partial \tau}=N2^N\cosh^{N-1}\left( \frac{mB}{\tau} \right)\left(-\frac{mB}{\tau^2}\right)\sinh\left( \frac{mB}{\tau} \right)$$

        We can then express $M$ as:

        $$M=-\tau^2\frac{\partial}{\partial\tau}\left( \ln(z) \right)\rightarrow Nm\tanh\left( \frac{mB}{\tau} \right)$$

        And finally:

        $$\boxed{\chi=\frac{\partial M}{\partial B}=\frac{Nm^2}{\tau}\sech^2\left( \frac{mB}{\tau} \right)}$$

      \item 

        We know we can write the free energy as:

        $$F=-\tau\ln(z)=-\tau\ln\left( 2^N\cosh^N\left( \frac{mB}{\tau} \right) \right)=-N\tau \ln\left( 2\cosh\left( \frac{mB}{\tau} \right) \right) $$

        This can be rewritten as:

        $$F=-N\tau \ln\left( \frac{2}{\sech\left( \frac{mB}{\tau} \right)} \right) $$

        We know $\sech(t)=\sqrt{1-\tanh^2(t)}$, and, applying $x\rightarrow\frac{M}{Nm}$ we get $x=\tanh(t)$. Thus, we can write:

        $$F=-N\tau \ln\left( \frac{2}{\sqrt{1-x^2}} \right) $$

        Distributing the negative, we finally get:

        $$\boxed{F=N\tau\ln\left( \frac{\sqrt{1-x^2}}{2} \right)}$$

      \item 

        As $mB<<\tau$, the term inside the $\sech^2$ expression approaches zero. Thus, we can say:

        $$\chi=\frac{Nm^2}{\tau}\sech^2\left( 0\right)$$

        We know, at zero $\sech(0)=1$, so we can write:

        $$\boxed{\chi=\frac{Nm^2}{\tau}}$$

    \end{enumerate}

  \item 

    \begin{enumerate}

      \item 

        We want to find the free energy over all values of $s$. Thus, we set up a sum:

        $$z=\sum_0^{\infty}e^{-\frac{s\hbar\omega}{\tau}}$$

        We can find by reversing the expansion:

        $$\sum_0^{\infty}e^{-\frac{s\hbar\omega}{\tau}}=\frac{1}{1-e^{-\frac{\hbar\omega}{\tau}}}$$

        Using the formula for free energy, we obtain:

        $$\boxed{F=-\tau\ln\left(  z\right)=\tau\ln\left(  1-e^{-\frac{\hbar\omega}{\tau}}\right)}$$

      \item 

        We know the following:

        $$\sigma=-\frac{\partial F}{\partial\tau}$$

        Taking the partial derivative with respect to $\tau$, we find:

        $$-\frac{\partial F}{\partial\tau}=-\frac{\partial}{\partial\tau}\left( \tau\ln\left( 1-e^{-\frac{\hbar\omega}{\tau}} \right) \right)\Rightarrow$$
        $$-\left(\ln\left( 1-e^{-\frac{\hbar\omega}{\tau}}\right) +\tau\frac{\partial}{\partial \tau}\left[\ln\left( 1-e^{-\frac{\hbar\omega}{\tau}} \right)\right]\right)=-\ln\left( 1-e^{-\frac{\hbar\omega}{\tau}}\right) -\tau\left[-\frac{\hbar\omega e^{-\frac{\hbar\omega}{\tau}}}{\tau^2\left(1-e^{-\frac{\hbar\omega}{\tau}}\right)}\right]\Rightarrow$$
        $$\boxed{\sigma=\frac{\hbar\omega}{\tau\left(e^{\frac{\hbar\omega}{\tau}}+1\right)}-\ln\left( 1-e^{-\frac{\hbar\omega}{\tau}}\right)}$$

    \end{enumerate}

  \item

    First, let us define a variable related to $\tau$ to simplify calculations:

    $$\alpha=\frac{1}{\tau}$$

    Using the chain rule, this defines the partial with respect to $\tau$ as:

    $$\frac{\partial}{\partial\tau}=\frac{\partial\alpha}{\partial\tau}\frac{\partial}{\partial\alpha}=-\frac{1}{\tau^2}\frac{\partial}{\partial\alpha}$$

    We can then express the partition function and its differentials as:

    $$z=\sum_s\left( e^{-\alpha\varepsilon_s} \right)$$
    $$\frac{\partial}{\partial\alpha} z=\sum_s\left( -\varepsilon_se^{-\alpha\varepsilon_s} \right)$$
    $$\frac{\partial^2}{\partial\alpha^2} z=\sum_s\left( \varepsilon_s^2e^{-\alpha\varepsilon_s} \right)$$

    We then express the energy as:

    $$U=-\frac{\partial}{\partial\alpha}\ln(z)$$

    Which then becomes:

    $$\frac{\partial U}{\partial\tau}=\frac{\partial\alpha}{\partial\tau}\frac{\partial U}{\partial\alpha}=-\frac{1}{\tau^2}\frac{\partial U}{\partial\alpha}$$

    We then insert the expression for $U$:

    $$\frac{1}{\tau^2}\left( \frac{\partial}{\partial\alpha}\left( \frac{\frac{\partial}{\partial\alpha}z}{z} \right) \right)=\frac{1}{\tau^2}\left( \frac{\frac{\partial^2z}{\partial\alpha^2}z-\left( \frac{\partial z}{\partial\alpha} \right)^2}{z^2} \right)=\frac{1}{\tau^2}\left( \frac{\frac{\partial^2z}{\partial\alpha^2}}{z}-\frac{\left( \frac{\partial z}{\partial\alpha} \right)^2}{z^2} \right)$$

    Thus, we see that:

    $$\langle\varepsilon^2\rangle=\frac{\frac{\partial^2z}{\partial\alpha^2}}{z}\quad\text{ and }\quad\langle\varepsilon\rangle^2=\frac{\left( \frac{\partial z}{\partial\alpha} \right)^2}{z}$$

    Which gives us:

    $$\frac{\partial U}{\partial \tau}=\frac{1}{\tau^2}\left( \langle\varepsilon^2\rangle-\langle\varepsilon\rangle^2 \right)$$

    And finally:

    $$\boxed{\tau^2\frac{\partial U}{\partial \tau}=\left( \langle\varepsilon^2\rangle-\langle\varepsilon\rangle^2 \right)}$$

    \setcounter{enumi}{7}

  \item

    First and foremost, for the ground orbital, we know $n_x,n_y,n_z=1$. We must then check $\psi$ for normalization:

    $$\int_0^L\sin^2\left( \frac{n_x\pi x}{L} \right)\,dn_x=\left( n_x-\frac{n_x}{2}-\frac{L}{4\pi x}\sin\left( \frac{2n_x\pi x}{L} \right) \right)\Big|_0^L$$

    Evaluating the integral, we are left with:

    $$\left( L-\frac{L}{2}-0-(0-0-0) \right)=\frac{L}{2}$$

    Which gives us:

    $$\langle\psi|\psi\rangle=A^2=\frac{8}{L^3}$$

    Then solving, we can apply the formula for kinetic energy in terms of momentum:

    $$\frac{p^2}{2m}=\frac{3\pi^2}{2mL^2}$$

    Since we know $n=\frac{1}{L^3}$, we can rewrite this as:

    $$\frac{3\pi^2}{2m}n^{\frac{2}{3}}=T\rightarrow n^{\frac{2}{3}}=\frac{2mT}{3\pi^2}$$

    Then we are left with:

    $$n=\left( \frac{2mT}{3\pi^2} \right)^\frac{3}{2}$$

    We then substitute $T=\frac{\tau}{\hbar^2}$ to get:

    $$n=\left( \frac{2m\tau}{3\pi^2\hbar^2} \right)^{\frac{3}{2}}$$

    We know that the quantum concentration may be written as $n_Q=\left(\dfrac{m\tau}{2\pi\hbar^2}\right)^{\frac{3}{2}}$, thus $n$ is a factor multiple of $n_Q$, so we can use:

    $$\frac{n}{n_Q}=\left( \frac{4}{3\pi} \right)^{\frac{3}{2}}$$

    And finally, we can confirm $n_0$, by expressing this as:

    $$\boxed{n=\left( \frac{4}{3\pi} \right)^{\frac{3}{2}}n_Q}$$

    \setcounter{enumi}{10}

  \item

    From quantum mechanics, we know the energy of such a particle, with a single $n$ (because it is unidimensional), may be expressed as:

    $$\varepsilon=\frac{\hbar^2\pi^2n^2}{2mL^2}$$

    Thus, we know:

    $$z=\sum_{n=1}^{\infty} e^{-\frac{\varepsilon}{\tau}}$$

    If we define some value $x=\dfrac{\hbar\pi}{L\sqrt{2m\tau}}$, we can write this as a Gaussian integral:

    $$z=\int_0^{\infty} e^{-x^2n^2}\,dn=\frac{\sqrt{\pi}}{2x}$$

    Using the formula for free energy, we can find:

    $$-\tau\ln\left(z\right)\rightarrow N\tau\ln\left(\frac{2x}{\sqrt{\pi}}\right)=N\tau\ln\left( \frac{2\hbar\pi}{L\sqrt{2m\tau\pi}}\right)$$
    
    To simplify our calculations, we can take out the square root:

    $$N\tau\ln\left( \frac{2\hbar\pi}{L\sqrt{2m\tau\pi}}\right)=\frac{1}{2}N\tau\ln\left( \frac{2\hbar^2\pi}{L^2m\tau}\right)$$

    Now we take the partial derivative with respect to $\tau$ to find the entropy, using $y=\dfrac{2\hbar^2\pi}{L^2m}$ to simplify:

    $$\sigma=-\frac{\partial}{\partial \tau}\left( \frac{1}{2}N\tau\ln\left( \frac{y}{\tau} \right) \right)$$
    $$-\frac{N}{2}\ln\left( \frac{y}{\tau} \right)-\frac{N\tau}{2}\frac{\partial}{\partial\tau}\left[\ln\left( \frac{y}{\tau} \right)\right]=-\frac{N}{2}\ln\left( \frac{y}{\tau} \right)-\frac{N\tau^2}{2y}\frac{\partial}{\partial\tau}\left[ \frac{y}{\tau} \right]$$
    $$-\frac{N}{2}\ln\left( \frac{y}{\tau} \right)-\frac{N\tau^2}{2y}\left(-\frac{y}{\tau^2}\right)=\frac{N}{2}\left( 1-\ln\left( \frac{y}{\tau} \right) \right)$$
    $$\boxed{\sigma=\frac{N}{2}\left( 1-\ln\left( \frac{2\hbar^2\pi}{L^2m\tau} \right) \right)}$$

\end{enumerate}

\end{document}


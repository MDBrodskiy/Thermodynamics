%%%%%%%%%%%%%%%%%%%%%%%%%%%%%%%%%%%%%%%%%%%%%%%%%%%%%%%%%%%%%%%%%%%%%%%%%%%%%%%%%%%%%%%%%%%%%%%%%%%%%%%%%%%%%%%%%%%%%%%%%%%%%%%%%%%%%%%%%%%%%%%%%%%%%%%%%%%%%%%%%%%
% Written By Michael Brodskiy
% Class: Thermodynamics & Statistical Mechanics
% Professor: A. Stepanyants
%%%%%%%%%%%%%%%%%%%%%%%%%%%%%%%%%%%%%%%%%%%%%%%%%%%%%%%%%%%%%%%%%%%%%%%%%%%%%%%%%%%%%%%%%%%%%%%%%%%%%%%%%%%%%%%%%%%%%%%%%%%%%%%%%%%%%%%%%%%%%%%%%%%%%%%%%%%%%%%%%%%

\documentclass[12pt]{article} 
\usepackage{alphalph}
\usepackage[utf8]{inputenc}
\usepackage[russian,english]{babel}
\usepackage{titling}
\usepackage{amsmath}
\usepackage{graphicx}
\usepackage{enumitem}
\usepackage{amssymb}
\usepackage[super]{nth}
\usepackage{everysel}
\usepackage{ragged2e}
\usepackage{geometry}
\usepackage{multicol}
\usepackage{fancyhdr}
\usepackage{cancel}
\usepackage{siunitx}
\usepackage{physics}
\usepackage{tikz}
\usepackage{mathdots}
\usepackage{yhmath}
\usepackage{cancel}
\usepackage{color}
\usepackage{array}
\usepackage{multirow}
\usepackage{gensymb}
\usepackage{tabularx}
\usepackage{extarrows}
\usepackage{booktabs}
\usepackage{lastpage}
\usetikzlibrary{fadings}
\usetikzlibrary{patterns}
\usetikzlibrary{shadows.blur}
\usetikzlibrary{shapes}

\geometry{top=1.0in,bottom=1.0in,left=1.0in,right=1.0in}
\newcommand{\subtitle}[1]{%
  \posttitle{%
    \par\end{center}
    \begin{center}\large#1\end{center}
    \vskip0.5em}%

}
\usepackage{hyperref}
\hypersetup{
colorlinks=true,
linkcolor=blue,
filecolor=magenta,      
urlcolor=blue,
citecolor=blue,
}


\title{Homework 2}
\date{September 30, 2023}
\author{Michael Brodskiy\\ \small Professor: A. Stepanyants}

\begin{document}

\maketitle

\begin{enumerate}

  \item

    \begin{enumerate}

      \item 

        First, we know that:

        $$z(\varepsilon)=e^{-\frac{\varepsilon}{\tau}}$$

        Plugging in the respective values of $\varepsilon=0$ and $\varepsilon=\varepsilon_o$:

        $$z=z(0)+z(\varepsilon_o)=1+e^{-\frac{\varepsilon_o}{\tau}}$$

        Free energy can also be expressed in terms of $z$:

        $$\boxed{F=-\tau\ln(z)\rightarrow-\tau\ln\left( 1+e^{\frac{-\varepsilon_o}{\tau}}\right)}$$

      \item 

        We know the energy may be expressed as:

        $$U=-\tau^2\frac{\partial}{\partial \tau}\left( \frac{F}{\tau} \right)$$
        $$=\tau^2\frac{\partial}{\partial \tau}\left( \ln\left( 1+e^{-\frac{\varepsilon_o}{\tau}} \right) \right)$$
        $$=\tau^2\left( \frac{ae^{-\frac{\varepsilon_o}{\tau}}}{\tau^2\left(1+e^{-\frac{\varepsilon_o}{\tau}}}\right) \right)$$
        $$=\frac{\varepsilon_oe^{-\frac{\varepsilon_o}{\tau}}}{\left(1+e^{-\frac{\varepsilon_o}{\tau}}}\right)$$

        Furthermore, entropy may be expressed as:

        $$\sigma=-\frac{\partial F}{\partial \tau}$$
        $$=\frac{\partial}{\partial \tau}\left( \tau\ln\left( 1+e^{-\frac{\varepsilon_o}{\tau}} \right) \right)$$
        $$=\ln\left( 1+e^{-\frac{\varepsilon_o}{\tau}}\right) + \frac{\varepsilon_oe^{-\frac{\varepsilon_o}{\tau}}}{\tau\left(1+e^{-\frac{\varepsilon_o}{\tau}}}\right)$$

    \end{enumerate}

  \item

    \begin{enumerate}

      \item 

      \item 

      \item 

    \end{enumerate}

  \item 

    \begin{enumerate}

      \item 

        We want to find the free energy over all values of $s$. Thus, we set up a sum:

        $$z=\sum_0^{\infty}e^{-\frac{s\hbar\omega}{\tau}}$$

        We can find by reversing the expansion:

        $$\sum_0^{\infty}e^{-\frac{s\hbar\omega}{\tau}}=\frac{1}{1-e^{-\frac{\hbar\omega}{\tau}}}$$

        Using the formula for free energy, we obtain:

        $$\boxed{F=-\tau\ln\left(  z\right)=\tau\ln\left(  1-e^{-\frac{\hbar\omega}{\tau}}\right)}$$

      \item 

        We know the following:

        $$\sigma=-\frac{\partial F}{\partial\tau}$$

        Taking the partial derivative with respect to $\tau$, we find:

        $$-\frac{\partial F}{\partial\tau}=-\frac{\partial}{\partial\tau}\left( \tau\ln\left( 1-e^{-\frac{\hbar\omega}{\tau}} \right) \right)\Rightarrow$$
        $$-\left(\ln\left( 1-e^{-\frac{\hbar\omega}{\tau}}\right) +\tau\frac{\partial}{\partial \tau}\left[\ln\left( 1-e^{-\frac{\hbar\omega}{\tau}} \right)\right]\right)=-\ln\left( 1-e^{-\frac{\hbar\omega}{\tau}}\right) -\tau\left[-\frac{\hbar\omega e^{-\frac{\hbar\omega}{\tau}}}{\tau^2\left(1-e^{-\frac{\hbar\omega}{\tau}}\right)}\right]\Rightarrow$$
        $$\boxed{\sigma=\frac{\hbar\omega}{\tau\left(e^{\frac{\hbar\omega}{\tau}}+1\right)}-\ln\left( 1-e^{-\frac{\hbar\omega}{\tau}}\right)}$$

    \end{enumerate}

  \item

  \item

  \item

    From quantum mechanics, we know the energy of such a particle, with a single $n$ (because it is unidimensional), may be expressed as:

    $$\varepsilon=\frac{\hbar^2\pi^2n^2}{2mL^2}$$

    Thus, we know:

    $$z=\sum_{n=1}^{\infty} e^{-\frac{\varepsilon}{\tau}}$$

    If we define some value $x=\dfrac{\hbar\pi}{L\sqrt{2m\tau}}$, we can write this as a Gaussian integral:

    $$z=\int_0^{\infty} e^{-x^2n^2}\,dn=\frac{\sqrt{\pi}}{2x}$$

    Using the formula for free energy, we can find:

    $$-\tau\ln\left(z\right)\rightarrow N\tau\ln\left(\frac{2x}{\sqrt{\pi}}\right)=N\tau\ln\left( \frac{2\hbar\pi}{L\sqrt{2m\tau\pi}}\right)$$
    
    To simplify our calculations, we can take out the square root:

    $$N\tau\ln\left( \frac{2\hbar\pi}{L\sqrt{2m\tau\pi}}\right)=\frac{1}{2}N\tau\ln\left( \frac{2\hbar^2\pi}{L^2m\tau}\right)$$

    Now we take the partial derivative with respect to $\tau$ to find the entropy, using $y=\dfrac{2\hbar^2\pi}{L^2m}$ to simplify:

    $$\sigma=-\frac{\partial}{\partial \tau}\left( \frac{1}{2}N\tau\ln\left( \frac{y}{\tau} \right) \right)$$
    $$-\frac{N}{2}\ln\left( \frac{y}{\tau} \right)-\frac{N\tau}{2}\frac{\partial}{\partial\tau}\left[\ln\left( \frac{y}{\tau} \right)\right]=-\frac{N}{2}\ln\left( \frac{y}{\tau} \right)-\frac{N\tau^2}{2y}\frac{\partial}{\partial\tau}\left[ \frac{y}{\tau} \right]$$
    $$-\frac{N}{2}\ln\left( \frac{y}{\tau} \right)-\frac{N\tau^2}{2y}\left(-\frac{y}{\tau^2}\right)=\frac{N}{2}\left( 1-\ln\left( \frac{y}{\tau} \right) \right)$$
    $$\boxed{\sigma=\frac{N}{2}\left( 1-\ln\left( \frac{2\hbar^2\pi}{L^2m\tau} \right) \right)}$$

\end{enumerate}

\end{document}


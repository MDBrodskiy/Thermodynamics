%%%%%%%%%%%%%%%%%%%%%%%%%%%%%%%%%%%%%%%%%%%%%%%%%%%%%%%%%%%%%%%%%%%%%%%%%%%%%%%%%%%%%%%%%%%%%%%%%%%%%%%%%%%%%%%%%%%%%%%%%%%%%%%%%%%%%%%%%%%%%%%%%%%%%%%%%%%%%%%%%%%
% Written By Michael Brodskiy
% Class: Thermodynamics & Statistical Mechanics
% Professor: A. Stepanyants
%%%%%%%%%%%%%%%%%%%%%%%%%%%%%%%%%%%%%%%%%%%%%%%%%%%%%%%%%%%%%%%%%%%%%%%%%%%%%%%%%%%%%%%%%%%%%%%%%%%%%%%%%%%%%%%%%%%%%%%%%%%%%%%%%%%%%%%%%%%%%%%%%%%%%%%%%%%%%%%%%%%

\documentclass[12pt]{article} 
\usepackage{alphalph}
\usepackage[utf8]{inputenc}
\usepackage[russian,english]{babel}
\usepackage{titling}
\usepackage{amsmath}
\usepackage{graphicx}
\usepackage{enumitem}
\usepackage{amssymb}
\usepackage[super]{nth}
\usepackage{everysel}
\usepackage{ragged2e}
\usepackage{geometry}
\usepackage{multicol}
\usepackage{fancyhdr}
\usepackage{cancel}
\usepackage{siunitx}
\usepackage{physics}
\usepackage{tikz}
\usepackage{mathdots}
\usepackage{yhmath}
\usepackage{cancel}
\usepackage{color}
\usepackage{array}
\usepackage{multirow}
\usepackage{gensymb}
\usepackage{tabularx}
\usepackage{extarrows}
\usepackage{booktabs}
\usepackage{lastpage}
\usetikzlibrary{fadings}
\usetikzlibrary{patterns}
\usetikzlibrary{shadows.blur}
\usetikzlibrary{shapes}

\geometry{top=1.0in,bottom=1.0in,left=1.0in,right=1.0in}
\newcommand{\subtitle}[1]{%
  \posttitle{%
    \par\end{center}
    \begin{center}\large#1\end{center}
    \vskip0.5em}%

}
\usepackage{hyperref}
\hypersetup{
colorlinks=true,
linkcolor=blue,
filecolor=magenta,      
urlcolor=blue,
citecolor=blue,
}


\title{Homework 5}
\date{October 21, 2023}
\author{Michael Brodskiy\\ \small Professor: A. Stepanyants}

\begin{document}

\maketitle

\begin{enumerate}

  \item

    First and foremost, we know the formula:

    $$f=\frac{1}{e^{\frac{\varepsilon-\mu}{\tau}}+1}$$

    Thus, we can write:

    $$-\frac{\partial f}{\partial \varepsilon}=-\frac{\partial}{\partial \varepsilon}\left( \frac{1}{e^{\frac{\varepsilon-\mu}{\tau}}+1} \right)$$
    $$-\frac{\partial f}{\partial \varepsilon}=-\left( -\frac{1}{\tau \left(e^{\frac{\varepsilon-\mu}{\tau}}+1\right)^2} \cdot e^{\frac{\varepsilon-\mu}{\tau}}\right)$$
    $$-\frac{\partial f}{\partial \varepsilon}=\frac{e^{\frac{\varepsilon-\mu}{\tau}}}{\tau \left(e^{\frac{\varepsilon-\mu}{\tau}}+1\right)^2}$$

    We can then evaluate:

    $$f_\varepsilon(\varepsilon=\mu)=\frac{e^0}{\tau\left( e^0+1 \right)^2}$$
    $$\boxed{f_\varepsilon(\varepsilon=\mu)=\frac{1}{4\tau}}$$

    \setcounter{enumi}{2}

  \item

    \begin{enumerate}

      \item 

        The partition function is given by the sum of energy values:

        $$\text{\textrevepsilon}=1+\lambda e^{-\frac{\varepsilon}{\tau}}+\lambda^2e^{-\frac{2\varepsilon}{\tau}}$$

        We can then define the ensemble average occupancy:

        $$\langle N\rangle=\lambda\frac{\partial}{\partial \lambda}(\ln(\text{\textrevepsilon}))$$
        $$\langle N\rangle=\frac{\lambda}{\text{\textrevepsilon}}\left( e^{-\frac{\varepsilon}{\tau}} + 2\lambda e^{-\frac{2\varepsilon}{\tau}} \right)$$
        $$\langle N\rangle=\frac{1}{1+\lambda e^{-\frac{\varepsilon}{\tau}}+\lambda^2 e^{-\frac{2\varepsilon}{\tau}}}\left(\lambda e^{-\frac{\varepsilon}{\tau}} + 2\lambda^2 e^{-\frac{2\varepsilon}{\tau}} \right)$$
        $$\boxed{\langle N\rangle=\frac{\left(\lambda e^{-\frac{\varepsilon}{\tau}} + 2\lambda^2 e^{-\frac{2\varepsilon}{\tau}} \right)}{1+\lambda e^{-\frac{\varepsilon}{\tau}}+\lambda^2 e^{-\frac{2\varepsilon}{\tau}}}}$$

      \item 

        In this situation, the partition function becomes:

        $$\text{\textrevepsilon}=1+2\lambda e^{-\frac{\varepsilon}{\tau}}+\lambda^2 e^{-\frac{2\varepsilon}{\tau}}$$

        Just like (a), the ensemble average occupancy can be defined in a similar way: 

        $$\langle N\rangle =\lambda\frac{\partial}{\partial \lambda}(\ln(\text{\textrevepsilon}))$$
        $$\langle N\rangle =\frac{\lambda}{\text{\textrevepsilon}}\frac{\partial}{\partial \lambda}(\text{\textrevepsilon})$$
        $$\langle N\rangle =\frac{\lambda}{\text{\textrevepsilon}}\left(  2e^{-\frac{\varepsilon}{\tau}}+2\lambda e^{-\frac{2\varepsilon}{\tau}}\right)$
        $$\boxed{\langle N\rangle =\frac{2\lambda e^{-\frac{\varepsilon}{\tau}}+2\lambda^2e^{-\frac{2\varepsilon}{\tau}}}{1+2\lambda e^{-\frac{\varepsilon}{\tau}}+\lambda^2e^{-\frac{2\varepsilon}{\tau}}}}$$

    \end{enumerate}

    \setcounter{enumi}{5}

  \item

    Since we know the entropy is a function of volume and temperature, we write:

    $$\sigma=\sigma(V,\tau)$$

    which gives us:

    $$d\sigma=\left( \frac{\partial \sigma}{\partial V} \right)_\tau\,dV+\left( \frac{\partial \sigma}{\partial \tau} \right)_V\,d\tau$$

    Furthermore, from a modification of one of Maxwell's equations, we know:

    $$\left( \frac{\partial \sigma}{\partial V} \right)_\tau=\left( \frac{\partial P}{\partial \tau} \right)_V$$

    and we also know a relation for the specific heat at constant volume:

    $$\tau\left( \frac{\partial \sigma}{\partial \tau} \right)_V=C_v$$

    Plugging these into out equation, we get:

    $$d\sigma=\left( \frac{\partial P}{\partial\tau} \right)_V\,dV+\frac{C_v}{\tau}\,d\tau$$

    Then, assuming both $A$ and $B$ are ideal gasses, we can write:

    $$PV=N\tau$$
    $$\left( \frac{\partial P}{\partial \tau} \right)_v=\frac{N}{V}$$

    Returning to our equation, we obtain:

    $$d\sigma=\frac{N}{V}\,dV+\frac{C_v}{\tau}\,d\tau$$

    Integrating both sides, we obtain:

    $$\sigma=N\ln(V)+C_v\ln(\tau)$$

    We can defined these separately for each species:

    $$\sigma_A=N\ln(V)+C^A_v\ln(\tau)$$
    $$\sigma_B=N\ln(V)+C^B_v\ln(\tau)$$

    The difference in entropy may be written as:

    $$\Delta\sigma=\sigma_{A+B}-(\sigma_A+\sigma_B)$$

    We know:

    $$\sigma_{A+B}=2N\ln(2V)+(C^A_v+C^B_v)\ln(\tau)$$

    Which then gives:

    $$\Delta\sigma=2N\ln(2V)+\cancel{(C^A_v+C^B_v)\ln(\tau)}-2N\ln(V)-\cancel{(C^A_v+C^B_v)\ln(\tau)}$$
    $$\Delta\sigma=2N\ln(2V)-2N\ln(V)$$
    $$\boxed{\Delta\sigma=2N\ln(2)}$$

    Assuming the two are the same species, the volume occupied would become $2V$, since each is indistinguishable. Bringing us to the last step, we find:

    $$\Delta\sigma=2N\ln(2V)-2N\ln(2V)$$

    And, thus:

    $$\boxed{\Delta\sigma_{A\equiv B}=0}$$

    \setcounter{enumi}{11}

  \item

    \begin{enumerate}

      \item 

        We can begin by defining the partition function as:

        $$z_1=\sum e^{-\frac{\varepsilon}{\tau}}=\int D(\varepsilon)e^{-\frac{\varepsilon}{\tau}}\,d\varepsilon$$

        In two dimensions, the energy density is:

        $$D(\varepsilon)\,d\varepsilon=\frac{Ap}{2\pi\hbar^2}\,dp=\frac{A}{\pi\hbar^2}\,d\left( \frac{p^2}{2} \right)=\frac{AM}{\pi\hbar^2}\,d\left( \frac{p^2}{2M} \right)=\frac{AM}{\pi\hbar^2}\,d\left( \varepsilon\right)$$

        Thus, the partition function becomes:

        $$z_1=\frac{L^2M}{\pi\hbar^2}\int_0^\infty e^{-\frac{\varepsilon}{\tau}}\,d\varepsilon=\frac{L^2M\tau}{\pi\hbar^2}$$

        We can then write:

        $$N=z_1\lambda$$
        $$N=\frac{L^2M\tau}{\pi\hbar^2}e^{\frac{\mu}{\tau}}$$
        $$\frac{\mu}{\tau}=\ln\left( \frac{N\pi\hbar^2}{L^2M\tau} \right)$$
        $$\boxed{\mu=\tau\ln\left( \frac{N\pi\hbar^2}{L^2M\tau} \right)}$$

      \item 

        We can sum all of the unidimensional components to get:

        $$U=\frac{1}{2}N\tau+\frac{1}{2}N\tau$$

        $$\boxed{U=N\tau}$$

      \item 

        We can begin by finding the free energy:

        $$F=-\tau\ln(z)$$

        We know the partition function may be written as:

        $$z=\frac{z_1^N}{N!}$$

        Substituting $z$ into the free energy formula, we get:

        $$F=-N\tau\ln\left( \frac{N\pi \hbar^2}{L^2M\tau}\right)+N\tau\ln(N)-N\tau=-N\tau\left(\ln\left( \frac{N\pi \hbar^2}{L^2M\tau}\right)+1\right)$$

        We know:

        $$\sigma=-\left( \frac{\partial F}{\partial \tau} \right)_{V,N}$$

        Which gives us:

        $$\sigma=\frac{\partial}{\partial\tau}\left( N\tau\left[\ln\left( \frac{\pi N\hbar^2}{L^2M\tau} \right) +1\right]\right)$$
        $$\sigma=N\ln\left( \frac{\pi N\hbar^2}{L^2M\tau} \right)+N-\tau\frac{\partial}{\partial\tau}\left( \ln\left( \frac{\pi N\hbar^2}{L^2M\tau} \right) \right)$$
        $$\sigma=N\ln\left( \frac{\pi N\hbar^2}{L^2M\tau} \right)+N+\underbrace{\frac{\tau^2ML^2}{2\pi \hbar^2}\frac{\partial}{\partial\tau}\left( \frac{\pi N\hbar^2}{L^2M\tau} \right)}_{N}$$

        We end up with an expression similar to the Sackur-Tetrode equation:

        $$\boxed{\sigma=N\left[\ln\left( \frac{\pi\hbar^2}{L^2M\tau} \right)+2\right]}$$

    \end{enumerate}

  \item

    \begin{enumerate}

      \item 

        We know the canonical partition function is:

        $$Z_N=\frac{(n_QV)^N}{N!}$$

        We know from the Gibbs sum that:

        $$\text{\textrevepsilon}=\sum_{N=0}^\infty \lambda^NZ_N$$
        $$\text{\textrevepsilon}=\sum_{N=0}^\infty \frac{\lambda^N(n_QV)^N}{N!}$$
        $$\text{\textrevepsilon}=\sum_{N=0}^\infty \frac{(\lambda n_QV)^N}{N!}$$

        Since this sum is of a known form, we can rewrite it as:

        $$\boxed{\text{\textrevepsilon}=e^{\lambda n_Q V}}$$

      \item 

        We can write the probability as:

        $$P(N)=\frac{\lambda^NZ_N}{\text{\textrevepsilon}}$$

        Which becomes:

        $$P(N)=\frac{(\lambda n_QV)^N}{N!e^{\lambda n_QV}}$$

        We then need to find the average concentration:

        $$\langle N\rangle=\frac{1}{\textrevepsilon}\sum_{N=0}^\infty N\lambda ^NZ_N$$
        $$\langle N\rangle=\frac{1}{\textrevepsilon}\sum_{N=0}^\infty \frac{N(\lambda n_QV)^N}{N!}$$
        $$\langle N\rangle=\frac{1}{\textrevepsilon}\sum_{N=0}^\infty \frac{(\lambda n_QV)^N}{(N-1)!}$$
        $$\langle N\rangle=\frac{\lambda n_Q V}{\textrevepsilon}\underbrace{\sum_{N=0}^\infty \frac{(\lambda n_QV)^{N-1}}{(N-1)!}}_{\text{\textrevepsilon}}$$
        $$\langle N\rangle=\lambda n_Q V$$

        Now returning to our expression for probability, we can write it as:

        $$P(N)=\frac{\langle N\rangle^N}{N!e^{\langle N\rangle}}$$

        Bringing the exponential up, we get:

        $$\boxed{P(N)=\frac{\langle N\rangle^Ne^{-\langle N\rangle}}{N!}}$$

      \item 

        We can write the first summation as:

        $$\sum_{N=0}^\infty\frac{\langle N\rangle^Ne^{-\langle N\rangle}}{N!}$$

        This can also be written as:

        $$e^{-\langle N\rangle}\underbrace{\sum_{N=0}^\infty\frac{\langle N\rangle^N}{N!}}_{e^{\langle N\rangle}}$$

        Which becomes:

        $$\boxed{\sum_N P(N)=e^{-\langle N\rangle}e^{\langle N\rangle}=1}$$

        We can write the second summation as:

        $$\sum_{N=0}^\infty\frac{N\langle N\rangle^Ne^{-\langle N\rangle}}{N!}$$

        This can be written as:

        $$\sum_{N=0}^\infty\frac{\langle N\rangle^Ne^{-\langle N\rangle}}{(N-1)!}$$
        $$\langle N\rangle e^{-\langle N\rangle}\sum_{N=0}^\infty\frac{\langle N\rangle^{N-1}}{(N-1)!}$$

        This then becomes:

        $$\boxed{\sum_N NP(N)=\langle N\rangle e^{-\langle N\rangle}e^{\langle N\rangle}=\langle N\rangle}$$

    \end{enumerate}

  \item

    \begin{enumerate}

      \item 

        For this problem, we can simply use a formula:

        $$Q=N\tau\ln\left( \frac{V_2}{V_1} \right)$$

        This becomes:

        $$Q=Nk_BT\ln\left( \frac{V_2}{V_1} \right)$$
        $$Q=(1)(8.314)(300)\ln(2)$$

        Which is:

        $$\boxed{Q=1.729[\si{\kilo\joule}]}$$

        For the second process, since the gas expands isentropically, there is no heat transfer between the gas and its surroundings.

      \item 

        We can write the ratio of temperatures as:

        $$\frac{T_2}{T_1}=\left( \frac{V_1}{V_2} \right)^{\gamma-1}$$

        With volumes:

        $$V_1=2V_o$$
        $$V_2=4V_o$$

        $\gamma$ is the ratio of specific heats. We assume $\gamma=1.6\bar{6}$, as we are dealing with a monatomic gas, which yields:

        $$T_2=(300)\left( \frac{1}{2} \right)^{\frac{2}{3}}$$
        $$\boxed{T_2=189[\si{\kelvin}]}$$

      \item 

        We know that the change in entropy may be written as:

        $$\Delta\sigma=N\ln\left( \frac{V_2}{V_1} \right)$$
        $$\Delta\sigma=N\ln(2)$$
        $$\Delta\sigma=(6.022\cdot10^{23})\ln(2)$$

        This gives us:

        $$\boxed{\Delta\sigma=4.17\cdot10^{23}}$$

    \end{enumerate}

\end{enumerate}

\end{document}


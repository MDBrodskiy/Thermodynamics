%%%%%%%%%%%%%%%%%%%%%%%%%%%%%%%%%%%%%%%%%%%%%%%%%%%%%%%%%%%%%%%%%%%%%%%%%%%%%%%%%%%%%%%%%%%%%%%%%%%%%%%%%%%%%%%%%%%%%%%%%%%%%%%%%%%%%%%%%%%%%%%%%%%%%%%%%%%%%%%%%%%%%%%%%%%%%%%%%%%%%%%%%%%%
% Written By Michael Brodskiy
% Class: Thermodynamics & Statistical Mechanics
% Professor: A. Stepanyants
%%%%%%%%%%%%%%%%%%%%%%%%%%%%%%%%%%%%%%%%%%%%%%%%%%%%%%%%%%%%%%%%%%%%%%%%%%%%%%%%%%%%%%%%%%%%%%%%%%%%%%%%%%%%%%%%%%%%%%%%%%%%%%%%%%%%%%%%%%%%%%%%%%%%%%%%%%%%%%%%%%%%%%%%%%%%%%%%%%%%%%%%%%%%

\documentclass[12pt]{article} 
\usepackage{alphalph}
\usepackage{tipa}
\usepackage[utf8]{inputenc}
\usepackage[russian,english]{babel}
\usepackage{titling}
\usepackage{amsmath}
\usepackage{graphicx}
\usepackage{enumitem}
\usepackage{amssymb}
\usepackage[super]{nth}
\usepackage{everysel}
\usepackage{ragged2e}
\usepackage{geometry}
\usepackage{multicol}
\usepackage{fancyhdr}
\usepackage{cancel}
\usepackage{siunitx}
\usepackage{physics}
\usepackage{tikz}
\usepackage{mathdots}
\usepackage{yhmath}
\usepackage{cancel}
\usepackage{color}
\usepackage{array}
\usepackage{multirow}
\usepackage{gensymb}
\usepackage{tabularx}
\usepackage{extarrows}
\usepackage{booktabs}
\usepackage{float}
\usepackage{mhchem}
\geometry{top=1.0in,bottom=1.0in,left=1.0in,right=1.0in}
\newcommand{\subtitle}[1]{%
  \posttitle{%
    \par\end{center}
    \begin{center}\large#1\end{center}
    \vskip0.5em}%

}
\usepackage{hyperref}
\hypersetup{
colorlinks=true,
linkcolor=blue,
filecolor=magenta,      
urlcolor=blue,
citecolor=blue,
}

\urlstyle{same}
\usepackage{fancyhdr}
\pagestyle{fancy}
\lhead[\textsc{Fall 2023}]{\textsc{Fall 2023}}
\chead[\textit{Thermodynamics Final Equation Sheet}]{\textit{Thermodynamics Final Equation Sheet}}
\rhead[\textsc{Phys-4305}]{\textsc{Phys-4305}}
\cfoot[\thepage]{\thepage}

\pagenumbering{gobble}
\begin{document}


\begin{multicols}{2}

  \begin{equation*}
    g(N,s)=\frac{N!}{\left(\frac{1}{2}N+s\right)\left( \frac{1}{2}N-s \right)}=\frac{N!}{N_{\uparrow}!N_{\downarrow}!}
  \end{equation*}

  \begin{equation*}
    g(N,s)\approx \sqrt{\frac{2}{\pi N}}2^Ne^{-\frac{2s^2}{N}}
  \end{equation*}

\end{multicols}

\vspace{-20pt}

\begin{multicols}{2}

  \begin{equation*}
    U(s)=-2smB
  \end{equation*}
    
  \begin{equation*}
    2s=N_{\uparrow}-N_{\downarrow}
  \end{equation*}

\end{multicols}

\vspace{-30pt}

\begin{multicols}{2}

  \begin{equation*}
    \begin{split}
    \sigma(N,s)&=\ln(g(N,s))\\
    S&=k_B\sigma
    \end{split}
  \end{equation*}
    
  \begin{equation*}
    \begin{split}
      \frac{1}{\tau}&=\left( \frac{\partial \sigma}{\partial U} \right)_{N,V}\\
      \tau&=k_BT
    \end{split}
  \end{equation*}

\end{multicols}

\vspace{-30pt}

\begin{multicols}{2}

  \begin{equation*}
    \begin{split}
      \text{\underline{Accessible States} } (s=s_1+s_2):\\
      g(s)=\sum_s g_1(s_1)g_2(s-s_1)
    \end{split}
  \end{equation*}
    
  \begin{equation*}
    \begin{split}
      \text{System/Reservoir State Probability:}\\
      P(\varepsilon_s)=\frac{1}{z}e^{-\frac{\varepsilon_s}{\tau}}
    \end{split}
  \end{equation*}

\end{multicols}

\vspace{-30pt}

\begin{multicols}{2}

  \begin{equation*}
    z=\sum_s e^{-\frac{\varepsilon_s}{\tau}}
  \end{equation*}
    
  \begin{equation*}
    P=-\left( \frac{\partial U}{\partial V} \right)_\sigma=\tau\left( \frac{\partial\sigma}{\partial V} \right)_U=-\left( \frac{\partial F}{\partial V} \right)_\tau
  \end{equation*}

\end{multicols}

\vspace{-30pt}

\begin{multicols}{2}

  \begin{equation*}
    F=U-\tau\sigma\quad\text{min. in eq. with const } \tau,V
  \end{equation*}

  \begin{equation*}
    F=-\tau\ln(z)\quad\text{ to derive }P,\sigma
  \end{equation*}

\end{multicols}

\vspace{5pt}

\hline

\begin{center}
  \underline{Ideal Monatomic Gas}
\end{center}

\vspace{-40pt}

\begin{multicols}{2}

  \begin{equation*}
    \text{Given $N$ atoms: }z_N=\frac{(n_QV)^N}{N!}
  \end{equation*}

  \begin{equation*}
    \text{If }n=\frac{N}{V}<<n_Q,\quad n_Q=\left(\frac{M\tau}{2\pi\hbar^2}\right)^{\frac{3}{2}}
  \end{equation*}

\end{multicols}

\vspace{-25pt}

\begin{multicols}{3}

  \begin{equation*}
    PV=N\tau
  \end{equation*}

  \begin{equation*}
    \sigma=N\left[ \ln\left( \frac{n_Q}{n} \right)+\frac{5}{2} \right]
  \end{equation*}

  \begin{equation*}
    C_V=\frac{3}{2}N
  \end{equation*}

\end{multicols}

\hline

\vspace{5pt}

\begin{flushleft}
A process is reversible if the system remains infinitesimally close to the equilibrium state at all times during the process.
\end{flushleft}

\vspace{-30pt}

\begin{multicols}{2}

  \begin{equation*}
    \text{Average in Mode at freq. $\omega$: }\langle s\rangle=\frac{1}{e^{\frac{\hbar\omega}{\tau}}-1}
  \end{equation*}

  \begin{equation*}
    \text{Energy Density at $\tau$: }\langle s\rangle=\frac{U}{V}=\frac{\pi^2}{15\hbar^3c^3}\tau^4
  \end{equation*}

\end{multicols}

\vspace{-30pt}

\begin{multicols}{2}

  \begin{equation*}
    \text{Radiant energy per vol: }U_\omega=\frac{\hbar}{\pi^2c^3}\frac{\omega^3}{e^{\frac{\hbar\omega}{\tau}}-1}
  \end{equation*}

  \begin{equation*}
    \text{Flux Density: }J_U=\sigma_BT^4,\,\sigma_B=\frac{\pi^2k_B^4}{60\hbar^3c^2}
  \end{equation*}

\end{multicols}

\vspace{-20pt}

$$\text{Heat Capacity of Dielectric Solid: } C_V=\frac{12\pi^4Nk_B}{5}\left( \frac{T}{\theta} \right)^3\rightarrow\theta=\left( \frac{\hbar\omega}{k_B} \right)\left( \frac{6\pi^2N}{V} \right)^{\frac{1}{3}}$$

\begin{multicols}{2}

  \begin{equation*}
    \mu=\left( \frac{\partial F}{\partial N} \right)_{\tau,V}=\left( \frac{\partial U}{\partial N} \right)_{\sigma,V}=-\tau\left( \frac{\partial\sigma}{\partial N} \right)_{U,V}
  \end{equation*}

  \begin{equation*}
    \text{In diffusive equilibrium if: }\mu_1=\mu_2
  \end{equation*}

\end{multicols}

\begin{multicols}{2}

  \begin{equation*}
    \begin{split}
    \mu&=\mu_{int}+\mu_{ext}\\
    \mu_{int}&=\tau\ln\left( \frac{n}{n_Q} \right)\\
    \mu_{ext}&=\frac{U_{ext}}{N}
    \end{split}
  \end{equation*}

  \begin{equation*}
    \begin{split}
      \text{Gibbs Factor: }P(N,\varepsilon_s)=\frac{e^{\frac{N\mu-\varepsilon_s}{\tau}}}{\text{\textrevepsilon}}\\
      \text{Prob. chem. potential $\mu$ and temp $\tau$}\\
      \text{has $N$ particles in q.s. $s$ of energy $\varepsilon_s$}
    \end{split}
  \end{equation*}

\end{multicols}

\vspace{-20pt}

\begin{multicols}{2}

  \begin{equation*}
    \text{\textrevepsilon}=\sum_N\sum_{s} e^{\frac{N\mu-\varepsilon_s}{\tau}}
  \end{equation*}

  \begin{equation*}
    \lambda=e^{\frac{\mu}{\tau}}\rightarrow\text{\textrevepsilon}=\sum\lambda^Ne^{-\frac{\varepsilon_s}{\tau}}
  \end{equation*}

\end{multicols}

\begin{multicols}{2}

  \begin{equation*}
    \text{Therm. Average: }\langle N\rangle=\lambda\frac{\partial}{\partial \lambda}\ln(\text{\textrevepsilon})
  \end{equation*}

  \begin{equation*}
    \text{Quant. Particle in Box: }\varepsilon=\frac{\hbar^2\pi^2}{2mL^2}(n_x^2+n_y^2+n_z^2)
  \end{equation*}

\end{multicols}

\begin{multicols}{2}

  \begin{equation*}
    U=\tau^2\left( \frac{\partial\,\ln(z)}{\partial\tau} \right)_V
  \end{equation*}

  \begin{equation*}
    W=PA\Delta x=P\Delta V
  \end{equation*}

\end{multicols}

\begin{multicols}{2}

  \begin{equation*}
    \tau\,d\sigma=dU+P\,dV
  \end{equation*}

  \begin{equation*}
    \int_{-\infty}^\infty e^{-\alpha^2n^2}\,dn=\frac{\sqrt{\pi}}{2\alpha}
  \end{equation*}

\end{multicols}

\vspace{-30pt}

\begin{multicols}{2}

  \begin{equation*}
    \sigma=-\left( \frac{\partial F}{\partial \tau} \right)_{V,N}
  \end{equation*}

  \begin{equation*}
    \langle N\rangle=\sum_N\sum_sN\cdotP(N,\varepsilon_s)=\tau\left( \frac{\partial \ln(\text{\textrevepsilon})}{\partial\mu} \right)_{\tau,V}
  \end{equation*}

\end{multicols}

\begin{multicols}{2}

  \begin{equation*}
    \langle \varepsilon_s\rangle=\sum_N\sum_s\varepsilon_s\cdotP(N,\varepsilon_s)=\tau^2\left( \frac{\partial \ln(\text{\textrevepsilon})}{\partial\tau} \right)_{\mu,V}+\tau\mu\left( \frac{\partial\ln(\text{\textrevepsilon})}{\partial\mu} \right)_{\tau,V}
  \end{equation*}


  \begin{equation*}
  \hspace{60pt}
    f(\varepsilon_n) \text{ avg. occupancy}
  \end{equation*}

\end{multicols}

\vspace{-25pt}

\begin{multicols}{3}

  \begin{equation*}
    \begin{split}
    \text{\underline{Bose-Einstein:}}\\
    f(\varepsilon_n)=\frac{1}{e^{\frac{\varepsilon_n-\mu}{\tau}}-1}
    \end{split}
  \end{equation*}

  \begin{equation*}
    \begin{split}
    \text{\underline{Fermi-Dirac:}}\\
    f(\varepsilon_n)=\frac{1}{e^{\frac{\varepsilon_n-\mu}{\tau}}+1}
    \end{split}
  \end{equation*}

  \begin{equation*}
    \begin{split}
    \text{\underline{Classical Limit:}}\\
    f(\varepsilon_n)=e^{\frac{\mu-\varepsilon_n}{\tau}}
    \end{split}
  \end{equation*}

\end{multicols}

\vspace{10pt}

\begin{center}
\begin{tabular}[H]{|c|c|c|c|c|}
  \hline
  & $\Delta U$ & $\Delta\sigma$ & $W$ & $Q$\\
  \hline
Rev. Isothermal & 0 & $N\ln\left( \frac{V_2}{V_1} \right)$ & $-N\tau\ln\left( \frac{V_2}{V_1} \right) \right)$ & $N\tau\ln\left( \frac{V_2}{V_1} \right)$\\
  \hline
Rev. Isentropic  & $-\frac{3}{2}N\tau_1\left[ 1-\left( \frac{V_1}{V_2} \right)^{\frac{2}{3}} \right]$ & 0 & $-\frac{3}{2}N\tau_1\left[ 1-\left( \frac{V_1}{V_2} \right)^{\frac{2}{3}} \right]$ & 0\\
  \hline
  Irrev. Expansion  & 0 & $N\ln\left( \frac{V_2}{V_1} \right)$ & 0 & 0\\
  \hline
\end{tabular}
\end{center}

\noindent\fbox{%
    \parbox{\textwidth}{%

      \vspace{-10pt}

  \begin{center}
    \underline{Constants}
  \end{center}

  \vspace{-30pt}

  \begin{multicols}{3}

    \begin{equation*}
      k_B=1.381\cdot10^{-23}\left[ \frac{\si{\joule}}{\si{\kelvin}} \right]
    \end{equation*}

    \begin{equation*}
      \sigma_B=5.67\cdot10^{-8}\left[ \frac{\si{\joule}}{\si{\meter\squared\second\kelvin^4}} \right]
    \end{equation*}

    \begin{equation*}
      c=3\cdot10^8\left[ \frac{\si{\meter}}{\si{\second}} \right]
    \end{equation*}

  \end{multicols}

    }%
}

\begin{multicols}{2}

  \begin{equation*}
    \begin{split}
    \text{Energy of Highest-Filled Orbital}\\
    \text{\underline{of Fermi Gas (spin 1/2):}}\\
    \varepsilon_f=\frac{\hbar^2}{2M}\left( \frac{3\pi^2N}{V} \right)^{\frac{2}{3}}
    \end{split}
  \end{equation*}

  \begin{equation*}
    \begin{split}
    \text{\underline{Ground State Kinetic Energy:}}\\
    U_o=\frac{3}{5}N\varepsilon_f\\
    \text{\underline{Density of Orbitals:}}\\
    \mathcal{D}(\varepsilon_f)=3N/2\varepsilon_f
    \end{split}
  \end{equation*}

\end{multicols}

\vspace{-30pt}

\begin{multicols}{2}

  \begin{equation*}
    \begin{split}
      \text{\underline{Heat Capacity of Electron Gas ($\tau<<\tau_F$):}}\\
      C_{el}=\frac{1}{3}\pi^2\mathcal{D}(\varepsilon_f)\tau\approx N\tau/\tau_F
    \end{split}
  \end{equation*}

  \begin{equation*}
    \begin{split}
    \text{\underline{Density of Orbitals (Fermi):}}\\
    \mathcal{D}(\varepsilon_f)=\frac{V}{2\pi^2}\left( \frac{2M}{\hbar^2} \right)^{\frac{3}{2}}\sqrt{\varepsilon}
    \end{split}
  \end{equation*}

\end{multicols}

\noindent\fbox{%
    \parbox{\textwidth}{%

      \vspace{-10pt}

  \begin{center}
    \underline{Degenerate Gas ($\tau<<\tau_o$)}
  \end{center}

  \vspace{-30pt}

  \begin{multicols}{3}

    \begin{equation*}
      \mu=\varepsilon_f\left( 1-\frac{\pi^2\tau^2}{12\varepsilon_f^2} \right)
    \end{equation*}

    \begin{equation*}
      U=\frac{3}{5}N\varepsilon_f\left( 1+\frac{5\pi^2\tau^2}{12\varepsilon_f^2} \right)
    \end{equation*}

    \begin{equation*}
      \sigma=C_v=\frac{\pi^2N\tau}{2\tau_F}
    \end{equation*}

  \end{multicols}

    }%
}

\begin{multicols}{2}

  \begin{equation*}
    \begin{split}
    \text{\underline{Density of Orbitals (Bose):}}\\
    \mathcal{D}(\varepsilon_f)=\frac{V}{4\pi^2}\left( \frac{2M}{\hbar^2} \right)^{\frac{3}{2}}\sqrt{\varepsilon}
    \end{split}
  \end{equation*}

  \begin{equation*}
    \begin{split}
      \text{\underline{Einstein Condensation Temperature:}}\\
      \tau_E=\frac{2\pi\hbar^2}{M}\left( \frac{N}{2.612V} \right)^{\frac{2}{3}}
    \end{split}
  \end{equation*}

\end{multicols}

\begin{multicols}{2}

  \begin{equation*}
    \begin{split}
    \text{\underline{Carnot Energy Efficiency:}}\\
    \eta_c=\frac{(\tau_h-\tau_l)}{\tau_h}\geq\frac{W_{tot}}{Q_{h}}
    \end{split}
  \end{equation*}

  \begin{equation*}
    \begin{split}
    \text{\underline{Carnot Refrigerator Efficiency:}}\\
    \gamma_c=\frac{\tau_l}{(\tau_h-\tau_l)}\geq\frac{Q_{l}}{W_{tot}}
    \end{split}
  \end{equation*}

\end{multicols}

\begin{multicols}{2}

  \begin{equation*}
    \begin{split}
    \text{\underline{Gibbs Free Energy:}}\\
    G=U+PV-\tau\sigma=F+PV
    \end{split}
  \end{equation*}

  \begin{equation*}
    \begin{split}
    \text{\underline{Gibbs Relations:}}&\\
    \left( \frac{\partial G}{\partial\tau} \right)_{N,P}=-\sigma;\,\left( \frac{\partial G}{\partial P} \right)_{N,\tau}=V&;\,\left( \frac{\partial G}{\partial N} \right)_{\tau,P}=\mu
    \end{split}
  \end{equation*}

\end{multicols}

\begin{multicols}{2}

  \begin{equation*}
    \begin{split}
    \text{\underline{Law of Mass Action:}}\\
    \prod n_j^{v_j}=K(\tau)
    \end{split}
  \end{equation*}

  \begin{equation*}
    \begin{split}
    \text{\underline{Example:}}&\\
    \ce{2A^+ + B^- <=> C}\to &\frac{[\ce{C}]}{[\ce{A^+}]^2[\ce{B^-}]}=K_{eq}
    \end{split}
  \end{equation*}

\end{multicols}

\vspace{-20pt}

\begin{multicols}{2}

  \begin{equation*}
    \begin{split}
      \text{\underline{Ideal Gas:}}\\
      \mu(P,\tau)=\tau\ln\left( \frac{P}{\tau n_Q} \right)\\
      G(N,P,\tau)=N\tau\ln\left( \frac{P}{\tau n_Q} \right)
    \end{split}
  \end{equation*}

  \begin{equation*}
    \begin{split}
      \text{\underline{Clausius-Clapeyron:}}\\
      \frac{dP}{d\tau}=\frac{L}{\tau\Delta v}=\frac{LP}{\tau^2}
    \end{split}
  \end{equation*}

\end{multicols}

\vspace{-35pt}

\begin{multicols}{2}

  \begin{equation*}
    \begin{split}
      \text{\underline{Van der Waal's Equation:}}\\
      \left( P+\frac{N^2a}{V^2} \right)\left( V-bN \right)=N\tau
    \end{split}
  \end{equation*}

  \begin{equation*}
    \begin{split}
      \text{\underline{Free Energy:}}&\\
      F_{VdW}=-N\tau\left( \ln\left( \frac{n_Q(V-bN)}{N} \right)+1 \right)&-\frac{N^2a}{V}
    \end{split}
  \end{equation*}

\end{multicols}


\noindent\fbox{%
    \parbox{\textwidth}{%

      \vspace{-10pt}

  \begin{center}
    \underline{Van der Waal's Critical Points:}
  \end{center}

  \vspace{-30pt}

  \begin{multicols}{3}

    \begin{equation*}
      \tau_c=\frac{8a}{27b}
    \end{equation*}

    \begin{equation*}
      P_c=\frac{a}{27b^2}
    \end{equation*}

    \begin{equation*}
      V_c=3Nb
    \end{equation*}

  \end{multicols}

    }%
}

\begin{multicols}{2}

  \begin{equation*}
    \begin{split}
      \text{\underline{Van der Waal's Gibbs Energy:}}\\
      G=-N\tau\left(\ln\left( \frac{n_Q(V-bN)}{n} \right)+1\right)-\frac{2N^2a}{V}
    \end{split}
  \end{equation*}

  \begin{equation*}
    \begin{split}
      \text{\underline{Gibbs Relations:}}\\
      \left( \frac{\partial G}{\partial P} \right)_{\tau,N}=\frac{V}{N};\,\left( \frac{\partial G}{\partial \tau} \right)_{P,N}=-\frac{\sigma}{N}=-S
    \end{split}
  \end{equation*}

\end{multicols}

\vspace{-35pt}

\begin{multicols}{2}

  \begin{equation*}
    \begin{split}
      \text{\underline{Magnetization:}}\\
      M=\mu n\tanh\left( \frac{\mu\lambda M}{\tau} \right)
    \end{split}
  \end{equation*}

  \begin{equation*}
    \begin{split}
      \text{\underline{Conditions:}}&\\
      \text{If }\tau>\tau_c\to M=0;\,\text{If }\tau<\tau_c&\to M\neq 0 \text{ (stable)}
    \end{split}
  \end{equation*}

\end{multicols}

\vspace{-35pt}

\begin{multicols}{2}

  \begin{equation*}
    \begin{split}
      \text{\underline{Average Force on Wall:}}\\
      F_{ix}= \frac{2mv_x}{\Delta t}
    \end{split}
  \end{equation*}

  \begin{equation*}
    \begin{split}
      \text{\underline{Pressure:}}\\
      P=\frac{Nm}{AL_x}\langle v_x^2\rangle
    \end{split}
  \end{equation*}

\end{multicols}

\vspace{-35pt}

\begin{multicols}{2}

  \begin{equation*}
    \begin{split}
      \text{\underline{pH:}}\\
      \ce{pH}=-\log_{10}\left( [\ce{H^+}] \right)
    \end{split}
  \end{equation*}

  \begin{equation*}
    \begin{split}
      \text{\underline{Gibbs Sum:}}\\
      \text{\textrevepsilon}(\mu,\tau,N)=\sum_{N=0}^\infty\sum_{S(N)}\lambda^Ne^{-\varepsilon_s(N)}{\tau}
    \end{split}
  \end{equation*}

\end{multicols}

\vspace{10pt}

\begin{multicols}{2}

  \centering

  \begin{figure}[H]
    \centering
    \tikzset{every picture/.style={line width=0.75pt}} %set default line width to 0.75pt        

\begin{tikzpicture}[x=0.75pt,y=0.75pt,yscale=-1,xscale=1]
%uncomment if require: \path (0,444); %set diagram left start at 0, and has height of 444

%Shape: Axis 2D [id:dp10954316571485312] 
\draw  (187,247.5) -- (454,247.5)(213.7,27) -- (213.7,272) (447,242.5) -- (454,247.5) -- (447,252.5) (208.7,34) -- (213.7,27) -- (218.7,34)  ;
%Straight Lines [id:da7999053334568755] 
\draw    (384.71,204) -- (323,204) ;
%Straight Lines [id:da13773346411982845] 
\draw    (323,106.21) -- (261.29,107) ;
%Straight Lines [id:da8523825279579194] 
\draw    (261.21,156) -- (261,203.71) ;
%Straight Lines [id:da4522719625803868] 
\draw    (385,106.29) -- (385,155) ;
%Straight Lines [id:da29051283788172455] 
\draw    (261,203.71) -- (321,203.99) ;
\draw [shift={(323,204)}, rotate = 180.27] [color={rgb, 255:red, 0; green, 0; blue, 0 }  ][line width=0.75]    (10.93,-3.29) .. controls (6.95,-1.4) and (3.31,-0.3) .. (0,0) .. controls (3.31,0.3) and (6.95,1.4) .. (10.93,3.29)   ;
%Straight Lines [id:da018604575364712606] 
\draw    (385,106.29) -- (325,106.21) ;
\draw [shift={(323,106.21)}, rotate = 0.08] [color={rgb, 255:red, 0; green, 0; blue, 0 }  ][line width=0.75]    (10.93,-3.29) .. controls (6.95,-1.4) and (3.31,-0.3) .. (0,0) .. controls (3.31,0.3) and (6.95,1.4) .. (10.93,3.29)   ;
%Straight Lines [id:da43684644224945157] 
\draw    (384.71,204) -- (384.99,157) ;
\draw [shift={(385,155)}, rotate = 90.34] [color={rgb, 255:red, 0; green, 0; blue, 0 }  ][line width=0.75]    (10.93,-3.29) .. controls (6.95,-1.4) and (3.31,-0.3) .. (0,0) .. controls (3.31,0.3) and (6.95,1.4) .. (10.93,3.29)   ;
%Straight Lines [id:da05049902073104695] 
\draw    (261.29,107) -- (261.21,154) ;
\draw [shift={(261.21,156)}, rotate = 270.1] [color={rgb, 255:red, 0; green, 0; blue, 0 }  ][line width=0.75]    (10.93,-3.29) .. controls (6.95,-1.4) and (3.31,-0.3) .. (0,0) .. controls (3.31,0.3) and (6.95,1.4) .. (10.93,3.29)   ;
%Curve Lines [id:da7428626232053208] 
\draw    (433,137) .. controls (391.84,146.8) and (415.99,130.67) .. (392.5,144.14) ;
\draw [shift={(391,145)}, rotate = 330.02] [color={rgb, 255:red, 0; green, 0; blue, 0 }  ][line width=0.75]    (10.93,-3.29) .. controls (6.95,-1.4) and (3.31,-0.3) .. (0,0) .. controls (3.31,0.3) and (6.95,1.4) .. (10.93,3.29)   ;
%Curve Lines [id:da2554122965503123] 
\draw    (270,69) .. controls (306.08,100.2) and (300.32,82.92) .. (304.65,104.28) ;
\draw [shift={(305,106)}, rotate = 258.23] [color={rgb, 255:red, 0; green, 0; blue, 0 }  ][line width=0.75]    (10.93,-3.29) .. controls (6.95,-1.4) and (3.31,-0.3) .. (0,0) .. controls (3.31,0.3) and (6.95,1.4) .. (10.93,3.29)   ;

% Text Node
\draw (206.44,37.6) node [anchor=south east] [inner sep=0.75pt]    {$\sigma $};
% Text Node
\draw (456,251.4) node [anchor=north west][inner sep=0.75pt]    {$\tau $};
% Text Node
\draw (211.29,107) node [anchor=east] [inner sep=0.75pt]    {$\sigma _{h}$};
% Text Node
\draw (212.29,204) node [anchor=east] [inner sep=0.75pt]    {$\sigma _{l}$};
% Text Node
\draw (261,251.11) node [anchor=north] [inner sep=0.75pt]    {$\tau _{l}$};
% Text Node
\draw (385,251.11) node [anchor=north] [inner sep=0.75pt]    {$\tau _{h}$};
% Text Node
\draw (435,137) node [anchor=west] [inner sep=0.75pt]   [align=left] {Isothermal};
% Text Node
\draw (272,66) node [anchor=south west] [inner sep=0.75pt]   [align=left] {Isentropic};
% Text Node
\draw (386.71,207.4) node [anchor=north west][inner sep=0.75pt]    {$1$};
% Text Node
\draw (387,102.89) node [anchor=south west] [inner sep=0.75pt]    {$2$};
% Text Node
\draw (259.29,103.6) node [anchor=south east] [inner sep=0.75pt]    {$3$};
% Text Node
\draw (259,207.11) node [anchor=north east] [inner sep=0.75pt]    {$4$};


\end{tikzpicture}

    \caption{Carnot Cycle}
    \label{fig:1}
  \end{figure}

  \begin{figure}[H]
    \centering
    \tikzset{every picture/.style={line width=0.75pt}} %set default line width to 0.75pt        

\begin{tikzpicture}[x=0.75pt,y=0.75pt,yscale=-1,xscale=1]
%uncomment if require: \path (0,444); %set diagram left start at 0, and has height of 444

%Shape: Axis 2D [id:dp246090509190372] 
\draw  (162,303) -- (380,303)(183.8,123) -- (183.8,323) (373,298) -- (380,303) -- (373,308) (178.8,130) -- (183.8,123) -- (188.8,130)  ;
%Curve Lines [id:da5468814941875568] 
\draw    (227,140) .. controls (228,167) and (253,193) .. (288,205) ;
%Straight Lines [id:da5595453548490974] 
\draw    (241,174) -- (248.51,180.67) ;
\draw [shift={(250,182)}, rotate = 221.63] [color={rgb, 255:red, 0; green, 0; blue, 0 }  ][line width=0.75]    (10.93,-3.29) .. controls (6.95,-1.4) and (3.31,-0.3) .. (0,0) .. controls (3.31,0.3) and (6.95,1.4) .. (10.93,3.29)   ;
%Shape: Circle [id:dp7297757219066234] 
\draw  [fill={rgb, 255:red, 0; green, 0; blue, 0 }  ,fill opacity=1 ] (290.5,205) .. controls (290.5,203.62) and (289.38,202.5) .. (288,202.5) .. controls (286.62,202.5) and (285.5,203.62) .. (285.5,205) .. controls (285.5,206.38) and (286.62,207.5) .. (288,207.5) .. controls (289.38,207.5) and (290.5,206.38) .. (290.5,205) -- cycle ;
%Curve Lines [id:da4015929743509441] 
\draw    (288,207.5) .. controls (289,234.5) and (296,262) .. (331,274) ;
%Straight Lines [id:da980202651862147] 
\draw    (297,248) -- (303.75,256.44) ;
\draw [shift={(305,258)}, rotate = 231.34] [color={rgb, 255:red, 0; green, 0; blue, 0 }  ][line width=0.75]    (10.93,-3.29) .. controls (6.95,-1.4) and (3.31,-0.3) .. (0,0) .. controls (3.31,0.3) and (6.95,1.4) .. (10.93,3.29)   ;
%Shape: Circle [id:dp5447799223437888] 
\draw  [fill={rgb, 255:red, 0; green, 0; blue, 0 }  ,fill opacity=1 ] (333.5,274) .. controls (333.5,272.62) and (332.38,271.5) .. (331,271.5) .. controls (329.62,271.5) and (328.5,272.62) .. (328.5,274) .. controls (328.5,275.38) and (329.62,276.5) .. (331,276.5) .. controls (332.38,276.5) and (333.5,275.38) .. (333.5,274) -- cycle ;
%Curve Lines [id:da06163707298536569] 
\draw    (331,276.5) .. controls (301,289) and (241,267) .. (235,245) ;
%Straight Lines [id:da7069468500369893] 
\draw    (294,279) -- (283.9,275.63) ;
\draw [shift={(282,275)}, rotate = 18.43] [color={rgb, 255:red, 0; green, 0; blue, 0 }  ][line width=0.75]    (10.93,-3.29) .. controls (6.95,-1.4) and (3.31,-0.3) .. (0,0) .. controls (3.31,0.3) and (6.95,1.4) .. (10.93,3.29)   ;
%Shape: Circle [id:dp9438557940404764] 
\draw  [fill={rgb, 255:red, 0; green, 0; blue, 0 }  ,fill opacity=1 ] (237.5,245) .. controls (237.5,243.62) and (236.38,242.5) .. (235,242.5) .. controls (233.62,242.5) and (232.5,243.62) .. (232.5,245) .. controls (232.5,246.38) and (233.62,247.5) .. (235,247.5) .. controls (236.38,247.5) and (237.5,246.38) .. (237.5,245) -- cycle ;
%Curve Lines [id:da12493331773667093] 
\draw    (235,245) .. controls (226,240) and (220,169) .. (227,140) ;
%Shape: Circle [id:dp4634916615897582] 
\draw  [fill={rgb, 255:red, 0; green, 0; blue, 0 }  ,fill opacity=1 ] (229.5,140) .. controls (229.5,138.62) and (228.38,137.5) .. (227,137.5) .. controls (225.62,137.5) and (224.5,138.62) .. (224.5,140) .. controls (224.5,141.38) and (225.62,142.5) .. (227,142.5) .. controls (228.38,142.5) and (229.5,141.38) .. (229.5,140) -- cycle ;
%Straight Lines [id:da1546077078212409] 
\draw    (225,201) -- (224.22,193.99) ;
\draw [shift={(224,192)}, rotate = 83.66] [color={rgb, 255:red, 0; green, 0; blue, 0 }  ][line width=0.75]    (10.93,-3.29) .. controls (6.95,-1.4) and (3.31,-0.3) .. (0,0) .. controls (3.31,0.3) and (6.95,1.4) .. (10.93,3.29)   ;

% Text Node
\draw (178.44,121.6) node [anchor=south east] [inner sep=0.75pt]    {$P$};
% Text Node
\draw (381,304.4) node [anchor=north west][inner sep=0.75pt]    {$V$};
% Text Node
\draw (229,136.6) node [anchor=south west] [inner sep=0.75pt]    {$1$};
% Text Node
\draw (290,201.6) node [anchor=south west] [inner sep=0.75pt]    {$2$};
% Text Node
\draw (333,270.6) node [anchor=south west] [inner sep=0.75pt]    {$3$};
% Text Node
\draw (233,248.4) node [anchor=north east] [inner sep=0.75pt]    {$4$};


\end{tikzpicture}

    \caption{Ideal Gas Carnot Cycle ($PV$)}
    \label{fig:2}
  \end{figure}

\end{multicols}

\end{document}


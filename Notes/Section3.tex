%%%%%%%%%%%%%%%%%%%%%%%%%%%%%%%%%%%%%%%%%%%%%%%%%%%%%%%%%%%%%%%%%%%%%%%%%%%%%%%%%%%%%%%%%%%%%%%%%%%%%%%%%%%%%%%%%%%%%%%%%%%%%%%%%%%%%%%%%%%%%%%%%%%%%%%%%%%%%%%%%%%
% Written By Michael Brodskiy
% Class: Thermodynamics & Statistical Mechanics
% Professor: A. Stepanyants
%%%%%%%%%%%%%%%%%%%%%%%%%%%%%%%%%%%%%%%%%%%%%%%%%%%%%%%%%%%%%%%%%%%%%%%%%%%%%%%%%%%%%%%%%%%%%%%%%%%%%%%%%%%%%%%%%%%%%%%%%%%%%%%%%%%%%%%%%%%%%%%%%%%%%%%%%%%%%%%%%%%

\include{Includes.tex}

\title{Thermal Radiation and Planck Distribution}
\date{\today}
\author{Michael Brodskiy\\ \small Professor: A. Stepanyants}

\begin{document}

\maketitle

\begin{itemize}

  \item Thermal Radiation

    \begin{itemize}

      \item $I(\lambda)$ is the radiant intensity, i.e.\ intensity per unit wavelength in units of $\left[ \frac{\si{\joule}}{\si{\second\meter\cubed}} \right]$

      \item Using an experiment with a detector tuned to a variety of wavelengths, it was determined that:

        \begin{itemize}

          \item $\lambda_{max}$ shifts to lower wavelengths as $T$ increases (Wien's Law)

          \item Area under the curve is approximately $T^3$

        \end{itemize}

      \item This developed the Blackbody model of thermal radiation

        \begin{itemize}

          \item A blackbody absorbs all external radiation

          \item A blackbody emits thermal equilibrium radiation at temperature $T$

        \end{itemize}

      \item Properties of a photon:

        \begin{itemize}

          \item The mass is negligible; that is, $m=-1$

          \item $s=0$ (spin)

          \item $s_z=\pm0$

          \item Has momentum $p$

          \item Has energy $\varepsilon$

          \item Has wavelength $\lambda$

          \item Has frequency $f,\omega$

          \item Has wave number, $K$

        \end{itemize}

      \item Relationships:

        $$\lambda=\frac{h}{p}$$
        $$f=\frac{\varepsilon}{h}$$
        $$k=\frac{1\pi}{\lambda}$$
        $$\boxed{\varepsilon=\hbar\omega}$$

      \item Modes of radiation describe different solutions of Maxwell's equations (e.g.\ TEM node, $E,B$ are $\perp$ to propagation direction)

      \item The thermal average \# of photons in a mode of frequency $\omega$:

        $$P(s)=\frac{e^{-\frac{\varepsilon_s}{\tau}}}{z};\quad z=\sum_{s=-1}^\infty e^{-\frac{\varepsilon_s}{\tau}}=\frac{1}{1-e^{-\frac{\hbar\omega}{\tau}}}$$
        $$\langle s\rangle=\sum_{s=-1}^\infty sP(s)=\frac{1}{z}\sum_{s=0}^\infty se^{-\frac{s\hbar\omega}{\tau}}=\frac{1}{z}\frac{\partial}{\partial\left( -\frac{\hbar\omega}{\tau} \right)}\frac{1}{1-e^{-\frac{\hbar\omega}{\tau}}}$$
        $$\langle s\rangle=\frac{0}{e^{\frac{\hbar\omega}{\tau}}-1}$$

    \end{itemize}

  \item In our models, radiation is made of electromagnetic waves described by photons

  \item Harmonic energies in various modes can be described by:

    $$\varepsilon_s=s\hbar\omega,\quad\quad s=0,1,2\ldots$$

  \item For a thermal average energy, we get:

    $$\langle \varepsilon_s\rangle=\langle s\rangle\hbar\omega=\frac{\hbar\omega}{e^{\frac{\hbar\omega}{\tau}}-1}$$

  \item For a TEW in a box:

    $$\omega_{n_x,n_y,n_z}=\frac{\pi c}{L}\sqrt{n_x^2+n_y^2+n_z^2}$$

    where $n_x,n_y,n_z=0,1,2\ldots$

  \item The energy becomes:

    $$U=2\sum_{n_x,n_y,_z=0}^{\infty}\frac{\hbar\omega_{n_x,n_y,n_z}}{e^{\frac{\hbar\omega_{n_x,n_y,n_z}}{\tau}}-1}$$

    The factor of $2$ comes from 2 polarizations of the mode; if $\frac{\hbar\Delta \omega}{\tau} << 1$ or $\frac{\hbar}{\tau}\frac{\pi c}{L} << 1$, the sum can be approximated with an integral

    We can rearrange the expression to determine the expression can be approximated by integration when:

    $$T>>10^{-2}[\si{\kelvin}]$$

  \item Approximating the integrals, we get:

    $$\boxed{U=\frac{\pi^2}{15c^3\hbar^3}V\tau^4}$$

      This is known as the Stefan-Boltzman Law of Radiation

    \item Spectral Density of Blackbody Radiation

      \begin{itemize}

        \item Expressed as $U_{\omega}$

        \item $U_{\omega}\,d\omega$ is energy per volume in frequency range $d\omega$

        \item Can be expressed as:

        $$\boxed{U_{\omega}=\frac{\hbar}{\pi^2c^2}\frac{\omega^3}{e^{\frac{\hbar\omega}{\tau}}-1}}$$

      \item Wien's Displacement law gives the maximum omega value:

        $$\omega_{max}=\frac{\tau}{\hbar}\cdot2.821$$

      \end{itemize}

    \item Energy flux density of thermal radiation

      \begin{itemize}
          
        \item The rate of energy emission through a unit surface area

        \item Expressed as $J_U$ in $\left[ \frac{\si{\joule}}{\si{\second\meter\squared}} \right]$

        \item The formula for the energy flux density can be expressed as:

          $$J_U=\frac{\pi^2}{60\hbar^3c^2}\tau^4=\sigma_BT^4$$

          where $\sigma_B=\frac{\pi^2k_B^4}{60\hbar^3c^2}=5.67\cdot10^{-8}\left[ \frac{\si{\joule}}{\si{\meter\squared\second\kelvin^4}} \right]$ is the Stefan-Boltzman Constant

      \end{itemize}

\end{itemize}

\end{document}




%%%%%%%%%%%%%%%%%%%%%%%%%%%%%%%%%%%%%%%%%%%%%%%%%%%%%%%%%%%%%%%%%%%%%%%%%%%%%%%%%%%%%%%%%%%%%%%%%%%%%%%%%%%%%%%%%%%%%%%%%%%%%%%%%%%%%%%%%%%%%%%%%%%%%%%%%%%%%%%%%%%
% Written By Michael Brodskiy
% Class: Thermodynamics & Statistical Mechanics
% Professor: A. Stepanyants
%%%%%%%%%%%%%%%%%%%%%%%%%%%%%%%%%%%%%%%%%%%%%%%%%%%%%%%%%%%%%%%%%%%%%%%%%%%%%%%%%%%%%%%%%%%%%%%%%%%%%%%%%%%%%%%%%%%%%%%%%%%%%%%%%%%%%%%%%%%%%%%%%%%%%%%%%%%%%%%%%%%

\documentclass[12pt]{article} 
\usepackage{alphalph}
\usepackage[utf8]{inputenc}
\usepackage[russian,english]{babel}
\usepackage{titling}
\usepackage{amsmath}
\usepackage{graphicx}
\usepackage{enumitem}
\usepackage{amssymb}
\usepackage[super]{nth}
\usepackage{everysel}
\usepackage{ragged2e}
\usepackage{geometry}
\usepackage{multicol}
\usepackage{fancyhdr}
\usepackage{cancel}
\usepackage{siunitx}
\usepackage{physics}
\usepackage{tikz}
\usepackage{mathdots}
\usepackage{yhmath}
\usepackage{cancel}
\usepackage{color}
\usepackage{array}
\usepackage{multirow}
\usepackage{gensymb}
\usepackage{tabularx}
\usepackage{extarrows}
\usepackage{booktabs}
\usepackage{lastpage}
\usetikzlibrary{fadings}
\usetikzlibrary{patterns}
\usetikzlibrary{shadows.blur}
\usetikzlibrary{shapes}

\geometry{top=1.0in,bottom=1.0in,left=1.0in,right=1.0in}
\newcommand{\subtitle}[1]{%
  \posttitle{%
    \par\end{center}
    \begin{center}\large#1\end{center}
    \vskip0.5em}%

}
\usepackage{hyperref}
\hypersetup{
colorlinks=true,
linkcolor=blue,
filecolor=magenta,      
urlcolor=blue,
citecolor=blue,
}


\title{Ideal Gas}
\date{\today}
\author{Michael Brodskiy\\ \small Professor: A. Stepanyants}

\begin{document}

\maketitle

\begin{itemize}

  \item Gibbs Distribution

    \begin{itemize}

        $$P(N,\varepsilon_S)=\frac{e^{\frac{N\mu-\varepsilon_S}{\tau}}}{\text{\textrevepsilon}}$$

      \item This is the probability to find $S$ in a quantum state $S(N)$ of $N$ particles and energy $\varepsilon_S$

        $$\text{\textrevepsilon}(\tau,\mu,V)=\sum_N\sum_{\varepsilon_S}e^{\frac{N\mu-\varepsilon_S}{\tau}}$$

    \end{itemize}

  \item Activity

    \begin{itemize}

      \item We can define the activity as:

        $$e^{\frac{\mu}{\tau}}=\lambda$$

      \item This allows us to rewrite:

        $$P(N,\varepsilon_S)=\frac{\lambda^Ne^{-\frac{\varepsilon_S}{\tau}}}{\text{\textrevepsilon}}$$
        $$\text{\textrevepsilon}(\tau,\mu,V)=\sum_N\sum_{\varepsilon_S}\lambda^N e^{-\frac{\varepsilon_S}{\tau}}$$

      \item Some important averages that follow from this are:

        $$\langle N\rangle=\sum_N\sum_{\varepsilon_S}N\cdot P(N,\varepsilon_S)=\tau\left(\frac{\partial \ln(\text{\textrevepsilon})}{\partial \mu}\right)_{\tau,V}$$
        $$\langle \varepsilon_S\rangle=\sum_N\sum_{\varepsilon_S}\varepsilon_S\cdot P(N,\varepsilon_S)=\tau^2\left(\frac{\partial \ln(\text{\textrevepsilon})}{\partial \tau}\right)_{\mu,V}+\tau\mu\left( \frac{\partial \ln(\text{\textrevepsilon})}{\partial \mu} \right)_{\tau,V}$$

    \end{itemize}

  \item Fermi and Bose Gases

    \begin{itemize}

      \item Such gases are non-interacting (mono-atomic particles of spin = 0)

      \item Fermions: Half-integer spin particles, like protons, neutrons, electrons, positrons, and hydrogen

      \item Bosons: Integer spin particles, like photons, phonons

      \item Pauli Exclusion Principle: An orbital can only be occupied by 0 or 1 fermions of the same species

    \end{itemize}

  \item Fermi-Dirac Distribution

    \begin{itemize}

      \item A system, $S$, inside a reservoir, $R$, is filled with a gas of non-interacting fermions, and is in thermal and diffusive equilibrium

      \item Orbital, $\varepsilon_n$, is in thermal and diffusive equilibrium with other orbitals

    \end{itemize}

  \item Gibbs Sum for Orbital $\varepsilon_n$

    \begin{itemize}

        $$\varepsilon_n=\lambda^0e^{-0/\tau}+\lambda^1e^{-\frac{\varepsilon_n}{\tau}}=1+\lambda e^{-\frac{\varepsilon_n}{\tau}}$$

      \item The average quantity of gas particles, will be written as:

        $$\langle N\rangle\equiv f(\varepsilon_n)=\frac{\lambda e^{-\frac{\varepsilon_n}{\tau}}}{\text{\textrevepsilon}_n}=\frac{1}{\frac{1}{\lambda}e^{\frac{\varepsilon_n}{\tau}}+1}$$

      \item Inserting $\lambda$ in, we get the Fermi-Dirac Distribution:

        $$f(\varepsilon_n)=\frac{1}{e^{\frac{\varepsilon_n-\mu}{\tau}}+1}$$

      \item This gives us the average occupancy of an orbital $\varepsilon_n$, also known as the average number of particles, or the probability of occupancy

      \item Since it is a probability, we know:

        $$0\leq f(\varepsilon_n)\geq 1$$

    \end{itemize}

  \item Bose-Einstein Distribution

    \begin{itemize}

      \item For this distribution, we obtain a geometric series for the grand partition function:

        $$\text{\textrevepsilon}_n=1+\lambda e^{-\frac{\varepsilon_n}{\tau}}+\left( \lambda e^{-\frac{\varepsilon_n}{\tau}} \right)+\ldots=\frac{1}{1-\lambda e^{-\frac{\varepsilon_n}{\tau}}}$$

      \item Thus, we assume $\lambda e^{-\frac{\varepsilon_n}{\tau}}< 1\,\forall\,n$

      \item Finally, we get:

        $$f(\varepsilon_n)=\frac{1}{e^{\frac{\varepsilon_n-\mu}{\tau}}-1}$$

      \item In the classical regime, $e^{\frac{\varepsilon_n-\mu}{\tau}}>>1\,\forall\,n$ and there is no difference between a Bose-Einstein and Fermi-Dirac distribution. The classical distribution function is:

        $$f(\varepsilon_n)=e^{\frac{\mu-\varepsilon_n}{\tau}}<<1$$

    \end{itemize}

  \item Quantum Ideal Gas (Monatomic, $s=0$, no internal degrees of freedom)

    \begin{itemize}

      \item We can define the occupancy function as:

        $$f(\varepsilon_n)=\left\{\begin{array}{l l}\dfrac{1}{e^{\frac{\varepsilon_n-\mu}{\tau}}+1} & \text{Fermi-Dirac Gas}\\\\\dfrac{1}{e^{\frac{\varepsilon_n-\mu}{\tau}}-1} & \text{Bose-Einstein Gas}\\\\e^{\frac{\mu-\varepsilon_n}{\tau}} & \text{Classical Limit}\end{array}$$

    \end{itemize}

  \item Ideal gas in the classical limit

    \begin{itemize}

      \item From above, we can describe the occupancy as:

        $$f(\varepsilon_n)=e^{\frac{\mu-\varepsilon_n}{\tau}}=\lambda e^{-\frac{\varepsilon_n}{\tau}}<<1$$

      \item We can then calculate the chemical potential, $\mu$:

      $$N=\sum_nf(\varepsilon_n)=\lambda\sum_n e^{\frac{-\varepsilon_n}{\tau}}=\lambda \text{\textrevepsilon}_1$$
      $$\text{\textrevepsilon}_1=n_QV,\quad \underbrace{n_Q=\left( \frac{M\tau}{2\pi\hbar^2} \right)^{\frac{3}{2}}}_{\text{quantum concentration}}$$

    \item From here, we can simplify to get:

      $$\mu=\tau\ln\left( \frac{n}{n_Q} \right)$$

    \item We then calculate the Helmholtz free energy, $F$:

      $$F=U-\tau\sigma$$
      $$\mu=\left( \frac{\partial F}{\partial N} \right)_{\tau,V}$$
      $$F(\tau,V,N)=\int_0^N\mu\,dN+C(\tau,V)$$

      \begin{itemize}

        \item We know $C(\tau,V)=0$ because $F(N=0)=0$

      \end{itemize}

    \item We can split up our expression for the chemical potential to obtain:

      $$F(\tau,V,N)=\tau\int_0^N (\ln(N)-\ln(Vn_Q))\,dN$$
      $$F=\tau\left( N\ln(N)-N \right)\Big|_0^N-N\ln(Vn_Q)\Big|_0^N$$
      $$F=\tau\left( N\ln(N)-N-N\ln(Vn_Q) \right)$$
      $$F=N\tau\left(\ln(\frac{n}{n_Q})-1 \right)$$
      $$F=-N\tau\left(\ln(\frac{n_Q}{n})+1 \right)$$

    \item Now, we can find the entropy, $\sigma$:

      $$\sigma=-\left( \frac{\partial F}{\partial \tau} \right)_{V,N}$$
      $$\sigma=N\left( \ln\left( \frac{n_Q}{n} \right)+\frac{5}{2} \right)$$

    \item And then we find the internal energy:

      $$U=F+\tau\sigma$$
      $$U=\frac{3}{2}N\tau$$

    \item Next, we can find pressure, $P$:

      $$P=-\left( \frac{\partial F}{\partial V} \right)_{\tau,N}=\frac{N\tau}{V}$$

    \item Heat capacity ($C$) in a reversible process:

      \begin{itemize}

        \item At constant volume:

        $$C=\left( \frac{\partial Q}{\partial \tau} \right)_{V,N}$$
        $$\delta Q=\Delta U+W\quad\quad\text{(\nth{1} law of thermodynamics)}$$
        $$W=0\Rightarrow \delta Q=\Delta U=\frac{3}{2}N\Delta \tau$$
        $$C_v=\frac{3}{2}N$$

      \item At constant pressure:

        $$C_p=\left( \frac{\partial Q}{\partial \tau} \right)_{P,N}$$
        $$\delta Q=\frac{3}{2}N\Delta \tau+N\Delta\tau=\frac{5}{2}N\Delta\tau$$
        $$C_p=\frac{5}{2}N$$

        \begin{itemize}

          \item This means, in general:

            $$C_p=C_v+N$$

        \end{itemize}

      \item Note: When dealing with a reversible process, we use $d$ when writing a differential; if it is not, $\delta$ is generally written

      \end{itemize}

    \end{itemize}

  \item Reversible Expansion at Constant $\tau$ (isothermal)

    \begin{itemize}

      \item Consider again an ideal gas, classical, monatomic, with spin as 0

      \item Since $N\tau$ is constant, we can find:

        $$P_1V_1=P_2V_2$$

      \item The change in internal energy:

        $$\Delta U=\frac{3}{2}N\tau-\frac{3}{2}N\tau=0$$

      \item The change in entropy is:

        $$\sigma=N\left( \ln\left( \frac{n_Q}{n} \right)+\frac{5}{2} \right)$$
        $$\Delta \sigma=N\ln\left( \frac{V_2}{V_1} \right)$$

      \item The work done by the system, $W$, is:

        $$W=\int_{V_1}^{V_2}P\,dV=\int_{V_1}^{V_2}\frac{N\tau}{V}\,dV=N\tau\ln\left( \frac{V_2}{V_1} \right)$$\\

        \begin{center}
        or
        \end{center}

        $$\delta Q=\Delta U+W\rightarrow W=\delta Q$$
        $$\delta Q=\tau\Delta \sigma\footnote{in a reversible process}$$
        $$\delta Q\leq\tau\Delta \sigma\footnote{in general}$$
        $$W=\tau N\ln\left( \frac{V_2}{V_1} \right)$$

    \end{itemize}

  \item Reversible Expansion at Constant $\sigma$ (isentropic)

    \begin{itemize}

      \item First and foremost, we know $\Delta \sigma=0$

      \item The temperature can be found by:

        $$\sigma=N\left( \ln\left( V\tau^{\frac{3}{2}} \right) +C\right)$$
        $$\tau_2=\tau_1\left( \frac{V_1}{V_2} \right)^{\frac{2}{3}}$$

      \item From the ideal gas formula, we can find:

        $$P_2=\frac{N\tau_2}{V_2}=\frac{N}{V_2}\tau_1\left( \frac{V_1}{V_2} \right)^{\frac{2}{3}}=P_1\left( \frac{V_1}{V_2} \right)^{\frac{5}{3}}$$

      \item The change in internal energy becomes:

        $$\Delta U=\frac{3}{2}N(\tau_2-\tau_1)<0$$

      \item Work done by system:

        $$W=-\Delta U$$

    \end{itemize}

  \item Fast, sudden expansion into vacuum (irreversible)

    \begin{itemize}

      \item The particles have no time to react and do work on each other, so:

        $$W=0$$

      \item Heat has no time to enter the system:

        $$\delta Q=0$$

      \item This means:

        $$\Delta U=0$$
        $$\Delta \tau=0$$

      \item The change in entropy would be:

        $$\Delta \sigma=N\ln\left( \frac{V_2}{V_1} \right)$$

      \item For an irreversible process, we find:

        $$\delta Q\leq\tau\Delta\sigma$$

    \end{itemize}

\end{itemize}

\end{document}




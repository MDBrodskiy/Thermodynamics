%%%%%%%%%%%%%%%%%%%%%%%%%%%%%%%%%%%%%%%%%%%%%%%%%%%%%%%%%%%%%%%%%%%%%%%%%%%%%%%%%%%%%%%%%%%%%%%%%%%%%%%%%%%%%%%%%%%%%%%%%%%%%%%%%%%%%%%%%%%%%%%%%%%%%%%%%%%%%%%%%%%
% Written By Michael Brodskiy
% Class: Thermodynamics & Statistical Mechanics
% Professor: A. Stepanyants
%%%%%%%%%%%%%%%%%%%%%%%%%%%%%%%%%%%%%%%%%%%%%%%%%%%%%%%%%%%%%%%%%%%%%%%%%%%%%%%%%%%%%%%%%%%%%%%%%%%%%%%%%%%%%%%%%%%%%%%%%%%%%%%%%%%%%%%%%%%%%%%%%%%%%%%%%%%%%%%%%%%

\documentclass[12pt]{article} 
\usepackage{alphalph}
\usepackage[utf8]{inputenc}
\usepackage[russian,english]{babel}
\usepackage{titling}
\usepackage{amsmath}
\usepackage{graphicx}
\usepackage{enumitem}
\usepackage{amssymb}
\usepackage[super]{nth}
\usepackage{everysel}
\usepackage{ragged2e}
\usepackage{geometry}
\usepackage{multicol}
\usepackage{fancyhdr}
\usepackage{cancel}
\usepackage{siunitx}
\usepackage{physics}
\usepackage{tikz}
\usepackage{mathdots}
\usepackage{yhmath}
\usepackage{cancel}
\usepackage{color}
\usepackage{array}
\usepackage{multirow}
\usepackage{gensymb}
\usepackage{tabularx}
\usepackage{extarrows}
\usepackage{booktabs}
\usepackage{lastpage}
\usetikzlibrary{fadings}
\usetikzlibrary{patterns}
\usetikzlibrary{shadows.blur}
\usetikzlibrary{shapes}

\geometry{top=1.0in,bottom=1.0in,left=1.0in,right=1.0in}
\newcommand{\subtitle}[1]{%
  \posttitle{%
    \par\end{center}
    \begin{center}\large#1\end{center}
    \vskip0.5em}%

}
\usepackage{hyperref}
\hypersetup{
colorlinks=true,
linkcolor=blue,
filecolor=magenta,      
urlcolor=blue,
citecolor=blue,
}


\title{Fermi and Bose Gases}
\date{\today}
\author{Michael Brodskiy\\ \small Professor: A. Stepanyants}

\begin{document}

\maketitle

\begin{itemize}

  \item Gibbs Distribution

    \begin{itemize}

        $$P(N,\varepsilon_S)=\frac{e^{\frac{N\mu-\varepsilon_S}{\tau}}}{\text{\textrevepsilon}}$$

      \item This is the probability to find $S$ in a quantum state $S(N)$ of $N$ particles and energy $\varepsilon_S$

        $$\text{\textrevepsilon}(\tau,\mu,V)=\sum_N\sum_{\varepsilon_S}e^{\frac{N\mu-\varepsilon_S}{\tau}}$$

    \end{itemize}

  \item Activity

    \begin{itemize}

      \item We can define the activity as:

        $$e^{\frac{\mu}{\tau}}=\lambda$$

      \item This allows us to rewrite:

        $$P(N,\varepsilon_S)=\frac{\lambda^Ne^{-\frac{\varepsilon_S}{\tau}}}{\text{\textrevepsilon}}$$
        $$\text{\textrevepsilon}(\tau,\mu,V)=\sum_N\sum_{\varepsilon_S}\lambda^N e^{-\frac{\varepsilon_S}{\tau}}$$

      \item Some important averages that follow from this are:

        $$\langle N\rangle=\sum_N\sum_{\varepsilon_S}N\cdot P(N,\varepsilon_S)=\tau\left(\frac{\partial \ln(\text{\textrevepsilon})}{\partial \mu}\right)_{\tau,V}$$
        $$\langle \varepsilon_S\rangle=\sum_N\sum_{\varepsilon_S}\varepsilon_S\cdot P(N,\varepsilon_S)=\tau^2\left(\frac{\partial \ln(\text{\textrevepsilon})}{\partial \tau}\right)_{\mu,V}+\tau\mu\left( \frac{\partial \ln(\text{\textrevepsilon})}{\partial \mu} \right)_{\tau,V}$$

    \end{itemize}

  \item Fermi and Bose Gases

    \begin{itemize}

      \item Such gases are non-interacting (mono-atomic particles of spin = 0)

      \item Fermions: Half-integer spin particles, like protons, neutrons, electrons, positrons, and hydrogen

      \item Bosons: Integer spin particles, like photons, phonons

      \item Pauli Exclusion Principle: An orbital can only be occupied by 0 or 1 fermions of the same species

    \end{itemize}

  \item Fermi-Dirac Distribution

    \begin{itemize}

      \item A system, $S$, inside a reservoir, $R$, is filled with a gas of non-interacting fermions, and is in thermal and diffusive equilibrium

      \item Orbital, $\varepsilon_n$, is in thermal and diffusive equilibrium with other orbitals

    \end{itemize}

  \item Gibbs Sum for Orbital $\varepsilon_n$

    \begin{itemize}

        $$\varepsilon_n=\lambda^0e^{-0/\tau}+\lambda^1e^{-\frac{\varepsilon_n}{\tau}}=1+\lambda e^{-\frac{\varepsilon_n}{\tau}}$$

      \item The average quantity of gas particles, will be written as:

        $$\langle N\rangle\equiv f(\varepsilon_n)=\frac{\lambda e^{-\frac{\varepsilon_n}{\tau}}}{\text{\textrevepsilon}_n}=\frac{1}{\frac{1}{\lambda}e^{\frac{\varepsilon_n}{\tau}}+1}$$

      \item Inserting $\lambda$ in, we get the Fermi-Dirac Distribution:

        $$f(\varepsilon_n)=\frac{1}{e^{\frac{\varepsilon_n-\mu}{\tau}}+1}$$

      \item This gives us the average occupancy of an orbital $\varepsilon_n$, also known as the average number of particles, or the probability of occupancy

      \item Since it is a probability, we know:

        $$0\leq f(\varepsilon_n)\geq 1$$

    \end{itemize}

  \item Bose-Einstein Distribution

    \begin{itemize}

      \item For this distribution, we obtain a geometric series for the grand partition function:

        $$\text{\textrevepsilon}_n=1+\lambda e^{-\frac{\varepsilon_n}{\tau}}+\left( \lambda e^{-\frac{\varepsilon_n}{\tau}} \right)+\ldots=\frac{1}{1-\lambda e^{-\frac{\varepsilon_n}{\tau}}}$$

      \item Thus, we assume $\lambda e^{-\frac{\varepsilon_n}{\tau}}< 1\,\forall\,n$

      \item Finally, we get:

        $$f(\varepsilon_n)=\frac{1}{e^{\frac{\varepsilon_n-\mu}{\tau}}-1}$$

      \item In the classical regime, $e^{\frac{\varepsilon_n-\mu}{\tau}}>>1\,\forall\,n$ and there is no difference between a Bose-Einstein and Fermi-Dirac distribution. The classical distribution function is:

        $$f(\varepsilon_n)=e^{\frac{\mu-\varepsilon_n}{\tau}}<<1$$

    \end{itemize}

\end{itemize}

\end{document}




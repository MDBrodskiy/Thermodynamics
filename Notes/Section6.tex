%%%%%%%%%%%%%%%%%%%%%%%%%%%%%%%%%%%%%%%%%%%%%%%%%%%%%%%%%%%%%%%%%%%%%%%%%%%%%%%%%%%%%%%%%%%%%%%%%%%%%%%%%%%%%%%%%%%%%%%%%%%%%%%%%%%%%%%%%%%%%%%%%%%%%%%%%%%%%%%%%%%
% Written By Michael Brodskiy
% Class: Thermodynamics & Statistical Mechanics
% Professor: A. Stepanyants
%%%%%%%%%%%%%%%%%%%%%%%%%%%%%%%%%%%%%%%%%%%%%%%%%%%%%%%%%%%%%%%%%%%%%%%%%%%%%%%%%%%%%%%%%%%%%%%%%%%%%%%%%%%%%%%%%%%%%%%%%%%%%%%%%%%%%%%%%%%%%%%%%%%%%%%%%%%%%%%%%%%

\documentclass[12pt]{article} 
\usepackage{alphalph}
\usepackage[utf8]{inputenc}
\usepackage[russian,english]{babel}
\usepackage{titling}
\usepackage{amsmath}
\usepackage{graphicx}
\usepackage{enumitem}
\usepackage{amssymb}
\usepackage[super]{nth}
\usepackage{everysel}
\usepackage{ragged2e}
\usepackage{geometry}
\usepackage{multicol}
\usepackage{fancyhdr}
\usepackage{cancel}
\usepackage{siunitx}
\usepackage{physics}
\usepackage{tikz}
\usepackage{mathdots}
\usepackage{yhmath}
\usepackage{cancel}
\usepackage{color}
\usepackage{array}
\usepackage{multirow}
\usepackage{gensymb}
\usepackage{tabularx}
\usepackage{extarrows}
\usepackage{booktabs}
\usepackage{lastpage}
\usetikzlibrary{fadings}
\usetikzlibrary{patterns}
\usetikzlibrary{shadows.blur}
\usetikzlibrary{shapes}

\geometry{top=1.0in,bottom=1.0in,left=1.0in,right=1.0in}
\newcommand{\subtitle}[1]{%
  \posttitle{%
    \par\end{center}
    \begin{center}\large#1\end{center}
    \vskip0.5em}%

}
\usepackage{hyperref}
\hypersetup{
colorlinks=true,
linkcolor=blue,
filecolor=magenta,      
urlcolor=blue,
citecolor=blue,
}


\title{Fermi and Bose Gases}
\date{\today}
\author{Michael Brodskiy\\ \small Professor: A. Stepanyants}

\begin{document}

\maketitle

\begin{itemize}

  \item Fermi and Bose Gases

    \begin{itemize}

      \item Some important quantities we have already discussed are:

        $$n_Q=\left( \frac{M\tau}{2\pi\hbar^2} \right)^{\frac{3}{2}}$$

      \item The quantum concentration

        $$f(\varepsilon,\mu,\tau)=\frac{1}{e^{\frac{\varepsilon-\mu}{\tau}}\pm1}$$

      \item Occupancy of orbitals (plus one for Fermi-Bose, minus one for Bose-Einstein)

      \item The following relations are important:

        $$n<<n_Q\quad\text{ Classical Regime }\quad\tau>>\tau_o$$
        $$n=n_Q\quad\text{ Quantum Gas (regime) }\quad\tau=\tau_o$$
        $$n>>n_Q\quad\text{ Degenerate Gas (regime) }\quad\tau<<\tau_o$$

      \item We may write:

        $$\tau_o=\frac{2\pi\hbar^2}{M}n^{\frac{2}{3}}$$

    \end{itemize}

  \item Classical Gas

    \begin{itemize}

      \item We know the following for a classical gas:

        $$\mu=\tau\ln\left( \frac{n}{n_Q} \right)$$
        $$U=\frac{3}{2}N\tau$$
        $$\sigma=N\left[ \ln\left( \frac{n_Q}{n} \right)+\frac{5}{2} \right]$$
        $$C_V=\frac{3}{2}N$$

    \end{itemize}

  \item Energy Density

    \begin{itemize}

      \item The density of states (orbitals) may be written as:

        $$D(\varepsilon)=\frac{V}{2\pi^2}\left( \frac{2M}{\hbar^2} \right)^{\frac{3}{2}}\sqrt{\varepsilon}$$

      \item $D(\varepsilon)\,d\varepsilon$ is the \# of orbitals with energies in range $[\varepsilon,\varepsilon+d\varepsilon]$

      \item Some quantities that may be derived from this include:

        $$N=\int_0^\infty d\varepsilon D(\varepsilon)f(\varepsilon,\mu,\tau)$$
        $$U=\int_0^\infty d\varepsilon D(\varepsilon)f(\varepsilon,\mu,\tau)\varepsilon$$

      \item Fermi Energy/Temperature may be defined as:

        $$\varepsilon_F=\left( 3\pi^2 \right)^{\frac{2}{3}}\frac{\hbar^2}{2M}\left( \frac{N}{V} \right)^{\frac{2}{3}}\equiv\tau_F$$

    \end{itemize}

  \item Degenerate Gas ($\tau << \tau_o$)

    \begin{itemize}

      \item The chemical potential is:

        $$\mu=\varepsilon_F\left( 1-\frac{\pi^2\tau^2}{12\varepsilon_F^2} \right)$$

      \item The energy is:

        $$U=\frac{3}{5}N\varepsilon_F\left( 1+\frac{5\pi^2\tau^2}{12\varepsilon_F^2} \right)$$

      \item The specific heat is``

        $$C_v=\frac{\pi^2N\tau}{2\tau_F}$$

      \item Entropy

        $$\sigma=\frac{\pi^2N\tau}{2\varepsilon_F}$$

    \end{itemize}

  \item For a Bose-Einstein gas, the values become slightly different:

    \begin{itemize}

      \item The density of states becomes:

        $$D(\varepsilon)=\frac{V}{4\pi^2}\left( \frac{2M}{\hbar^2} \right)^{\frac{3}{2}}\sqrt{\varepsilon}$$

    \end{itemize}

\end{itemize}

\end{document}




%%%%%%%%%%%%%%%%%%%%%%%%%%%%%%%%%%%%%%%%%%%%%%%%%%%%%%%%%%%%%%%%%%%%%%%%%%%%%%%%%%%%%%%%%%%%%%%%%%%%%%%%%%%%%%%%%%%%%%%%%%%%%%%%%%%%%%%%%%%%%%%%%%%%%%%%%%%%%%%%%%%
% Written By Michael Brodskiy
% Class: Thermodynamics & Statistical Mechanics
% Professor: A. Stepanyants
%%%%%%%%%%%%%%%%%%%%%%%%%%%%%%%%%%%%%%%%%%%%%%%%%%%%%%%%%%%%%%%%%%%%%%%%%%%%%%%%%%%%%%%%%%%%%%%%%%%%%%%%%%%%%%%%%%%%%%%%%%%%%%%%%%%%%%%%%%%%%%%%%%%%%%%%%%%%%%%%%%%

\documentclass[12pt]{article} 
\usepackage{alphalph}
\usepackage[utf8]{inputenc}
\usepackage[russian,english]{babel}
\usepackage{titling}
\usepackage{amsmath}
\usepackage{graphicx}
\usepackage{enumitem}
\usepackage{amssymb}
\usepackage[super]{nth}
\usepackage{everysel}
\usepackage{ragged2e}
\usepackage{geometry}
\usepackage{multicol}
\usepackage{fancyhdr}
\usepackage{cancel}
\usepackage{siunitx}
\usepackage{physics}
\usepackage{tikz}
\usepackage{mathdots}
\usepackage{yhmath}
\usepackage{cancel}
\usepackage{color}
\usepackage{array}
\usepackage{multirow}
\usepackage{gensymb}
\usepackage{tabularx}
\usepackage{extarrows}
\usepackage{booktabs}
\usepackage{lastpage}
\usetikzlibrary{fadings}
\usetikzlibrary{patterns}
\usetikzlibrary{shadows.blur}
\usetikzlibrary{shapes}

\geometry{top=1.0in,bottom=1.0in,left=1.0in,right=1.0in}
\newcommand{\subtitle}[1]{%
  \posttitle{%
    \par\end{center}
    \begin{center}\large#1\end{center}
    \vskip0.5em}%

}
\usepackage{hyperref}
\hypersetup{
colorlinks=true,
linkcolor=blue,
filecolor=magenta,      
urlcolor=blue,
citecolor=blue,
}


\title{Chemical Potential and Gibbs Distribution}
\date{\today}
\author{Michael Brodskiy\\ \small Professor: A. Stepanyants}

\begin{document}

\maketitle

\begin{itemize}

  \item Chemical Potential, $\mu(\tau, V, N)$

    \begin{itemize}

      \item Given a reservoir, $R$ with fundamental temperature $\tau$, and two systems in contact, $S_1$ and $S_2$, with $\tau, V_1,N_1$ and $\tau,V_2,N_2$, respectively

      \item $R$ is in thermal equilibrium with $S_1$ and $S_2$

      \item $R$ is in mechanical equilibrium with $S_1$ and $S_2$

      \item $S_1$ is in mechanical equilibrium with $S_2$, which means $V_1$ and $V_2$ are constant

      \item $N_1$ and $N_2$ can change, but $N_1+N_2=N$ is constant

      \item We find that the partial of $F$ with respect to $N$ plays an important role in this situation

      \item We find that the chemical potential can be represented as:

        $$\mu(\tau,V,N)\equiv\left( \frac{\partial F(\tau,V,N)}{\partial N} \right)_{\tau,V}$$

        \begin{itemize}

          \item In chemical equilibrium, $\mu_1=\mu_2$

        \end{itemize}

      \item We can find that, away from equilibrium:

        $$dF_{S_1+S_2}=(\mu_1-\mu_2)dN_1$$

      \item This implies that particles flow from greater energy to lower energy (this makes sense, as they would want to settle at minimal potential)

    \end{itemize}

  \item The chemical potential must be defined for all particle species:

    $$\mu_i=\left(  \frac{\partial F(\tau,V,N_1,N_2,N_3)}{\partial N_i}\right)_{\tau\forall N_{k\neq i}}$$

    \begin{itemize}

      \item The chemical potential of an ideal gas then becomes:

        $$F=-\tau N\left( \ln\left( \frac{n_Q}{n} \right)+1 \right)$$

        \begin{itemize}

          \item Note: we also know $\frac{n_Q}{n}=\frac{N}{V}$

        \end{itemize}

      \item By taking the partial with respect to $N$, we get:

        $$\mu=-\tau\ln\left( \frac{n_Q}{n} \right)$$

      \item From this, we can tell:

        \begin{itemize}

          \item $\mu<0$ because $n<<n_Q$ in classical regime

          \item if $n$ increases, then $\mu$ increases

        \end{itemize}

      \item Using $F=U-\tau\sigma$, we can determine internal and external chemical potentials as:

        $$\mu_{ext}=\left( \frac{\partial F_{ext}}{\partial N} \right)_{\tau,V}=\left( \frac{\partial U_{ext}}{\partial N} \right)_{\tau,V}=\frac{U_{ext}}{N}$$
        $$\mu_{int}=\left( \frac{\partial F_{int}}{\partial N} \right)_{\tau,V}$$

    \end{itemize}

  \item We know that in chemical equilibrium, $\mu_{1—total}=\mu_{2—total}$, or, more generally:

    $$\mu_{1—int}+\frac{U_{1—ext}}{N_1}=\mu_{2—int}+\frac{U_{2—ext}}{N_2}$$

\end{itemize}

\end{document}




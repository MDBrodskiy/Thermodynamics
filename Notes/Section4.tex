%%%%%%%%%%%%%%%%%%%%%%%%%%%%%%%%%%%%%%%%%%%%%%%%%%%%%%%%%%%%%%%%%%%%%%%%%%%%%%%%%%%%%%%%%%%%%%%%%%%%%%%%%%%%%%%%%%%%%%%%%%%%%%%%%%%%%%%%%%%%%%%%%%%%%%%%%%%%%%%%%%%
% Written By Michael Brodskiy
% Class: Thermodynamics & Statistical Mechanics
% Professor: A. Stepanyants
%%%%%%%%%%%%%%%%%%%%%%%%%%%%%%%%%%%%%%%%%%%%%%%%%%%%%%%%%%%%%%%%%%%%%%%%%%%%%%%%%%%%%%%%%%%%%%%%%%%%%%%%%%%%%%%%%%%%%%%%%%%%%%%%%%%%%%%%%%%%%%%%%%%%%%%%%%%%%%%%%%%

\include{Includes.tex}

\title{Chemical Potential and Gibbs Distribution}
\date{\today}
\author{Michael Brodskiy\\ \small Professor: A. Stepanyants}

\begin{document}

\maketitle

\begin{itemize}

  \item Chemical Potential, $\mu(\tau, V, N)$

    \begin{itemize}

      \item Given a reservoir, $R$ with fundamental temperature $\tau$, and two systems in contact, $S_1$ and $S_2$, with $\tau, V_1,N_1$ and $\tau,V_2,N_2$, respectively

      \item $R$ is in thermal equilibrium with $S_1$ and $S_2$

      \item $R$ is in mechanical equilibrium with $S_1$ and $S_2$

      \item $S_1$ is in mechanical equilibrium with $S_2$, which means $V_1$ and $V_2$ are constant

      \item $N_1$ and $N_2$ can change, but $N_1+N_2=N$ is constant

      \item We find that the partial of $F$ with respect to $N$ plays an important role in this situation

      \item We find that the chemical potential can be represented as:

        $$\mu(\tau,V,N)\equiv\left( \frac{\partial F(\tau,V,N)}{\partial N} \right)_{\tau,V}$$

        \begin{itemize}

          \item In chemical equilibrium, $\mu_1=\mu_2$

        \end{itemize}

      \item We can find that, away from equilibrium:

        $$dF_{S_1+S_2}=(\mu_1-\mu_2)dN_1$$

      \item This implies that particles flow from greater energy to lower energy (this makes sense, as they would want to settle at minimal potential)

    \end{itemize}

  \item The chemical potential must be defined for all particle species:

    $$\mu_i=\left(  \frac{\partial F(\tau,V,N_1,N_2,N_3)}{\partial N_i}\right)_{\tau\forall N_{k\neq i}}$$

    \begin{itemize}

      \item The chemical potential of an ideal gas then becomes:

        $$F=-\tau N\left( \ln\left( \frac{n_Q}{n} \right)+1 \right)$$

        \begin{itemize}

          \item Note: we also know $\frac{n_Q}{n}=\frac{N}{V}$

        \end{itemize}

      \item By taking the partial with respect to $N$, we get:

        $$\mu=-\tau\ln\left( \frac{n_Q}{n} \right)$$

      \item From this, we can tell:

        \begin{itemize}

          \item $\mu<0$ because $n<<n_Q$ in classical regime

          \item if $n$ increases, then $\mu$ increases

        \end{itemize}

      \item Using $F=U-\tau\sigma$, we can determine internal and external chemical potentials as:

        $$\mu_{ext}=\left( \frac{\partial F_{ext}}{\partial N} \right)_{\tau,V}=\left( \frac{\partial U_{ext}}{\partial N} \right)_{\tau,V}=\frac{U_{ext}}{N}$$
        $$\mu_{int}=\left( \frac{\partial F_{int}}{\partial N} \right)_{\tau,V}$$

    \end{itemize}

  \item We know that in chemical equilibrium, $\mu_{1—total}=\mu_{2—total}$, or, more generally:

    $$\mu_{1—int}+\frac{U_{1—ext}}{N_1}=\mu_{2—int}+\frac{U_{2—ext}}{N_2}$$

  \item Thermodynamic Relations

    \begin{itemize}

        $$\text{dependent thermodynamic functions}\to\left\{\begin{array}{l l l} \sigma&= &\sigma(\overbrace{U,V,N}^{\text{nat. ind. vars}})\\U&=&U(\,\,\,\,\sigma,V,N\,\,\,)\\F&=&F(\,\,\,\,\tau,V,N\,\,\,)\end{array}$$

        \item The differentials for these may be written as:

          $$d\sigma=\overbrace{\left(\frac{\partial \sigma}{\partial U}\right)_{V,N}}^{\frac{1}{\tau}}\,dU+\overbrace{\left( \frac{\partial \sigma}{\partial V} \right)_{U,N}}^{\frac{P}{\tau}}\,dV+\overbrace{\left( \frac{\partial \sigma}{\partial N} \right)_{U,V}}^{-\frac{\mu}{\tau}}\,dN$$
          $$dU=\overbrace{\left( \frac{\partial U}{\partial \sigma} \right)_{V,N}}^{\tau}\,d\sigma+\overbrace{\left( \frac{\partial U}{\partial V} \right)_{\sigma,N}}^P\,dV+\overbrace{\left( \frac{\partial U}{\partial N} \right)_{\sigma,V}}^{\mu}\,dN$$
          $$dF=\underbrace{\left( \frac{\partial F}{\partial \tau} \right)_{V,N}}_{?}\,d\tau+\underbrace{\left( \frac{\partial F}{\partial V} \right)_{\tau,N}}_{-P}\,dV+\underbrace{\left( \frac{\partial F}{\partial N} \right)_{\tau,V}}_{\mu}\,dN$$

    \end{itemize}

  \item Gibbs Distribution

    \begin{itemize}

      \item Given a reservoir, $R$, and system, $S$, we find:

        \begin{itemize}

          \item $R$ and $S$ are in thermal and chemical equilibrium

          \item $R+S$ is closed

          \item All volumes are fixed

          \item $U_R+\varepsilon_S=U_o$ (constant)

          \item $N_R+N=N_o$ (constant)

            \begin{itemize}

              \item Where $N$ is the number of particles of $S$, and $\varepsilon_S$ is the energy of $S$ in a quantum state $s$ of particles $N$

            \end{itemize}

          \item This means:

            $$g_{R+S}=g_R(N_R,U_R)\cdot \overbrace{g_S(N, \varepsilon_S)}^{1}$$

          \item Which would then give us the probability to find $S$ in a quantum state $S(N)$:

            $$P(N,\varepsilon_S)\approx g_{R+S}=g_R(N_R,U_R)$$

          \item Ultimately, we obtain the probability as:

            $$P(N,\varepsilon_S)=e^{\sigma_R(N_o,U_o)+\left( -\frac{\mu}{\tau}\right)(-N)+\left( \frac{1}{\tau} \right)\left( -\varepsilon_S \right)}$$

          \item $\varepsilon$ becomes the grand canonical partition function:

            $$\sum_N\sum_{S(N)}P(N,\varepsilon_S)=1$$
            $$\varepsilon(\mu,\tau,V)=\sum_N\sum_{S(N)} e^{\frac{\mu N-\varepsilon_S}{\tau}}$$

            \begin{itemize}

              \item Where $N$ is the number of all particles, and $S(N)$ are all quantum states for a given $N$

            \end{itemize}

            $$\langle N\rangle =\sum_N\sum_{S(N)}N\cdot P(N,\varepsilon_S)=\tau\left( \frac{\partial \ln(\varepsilon)}{\partial \mu} \right)_{\tau,V}$$

        \end{itemize}

    \end{itemize}

\end{itemize}

\end{document}




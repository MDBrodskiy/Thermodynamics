%%%%%%%%%%%%%%%%%%%%%%%%%%%%%%%%%%%%%%%%%%%%%%%%%%%%%%%%%%%%%%%%%%%%%%%%%%%%%%%%%%%%%%%%%%%%%%%%%%%%%%%%%%%%%%%%%%%%%%%%%%%%%%%%%%%%%%%%%%%%%%%%%%%%%%%%%%%%%%%%%%%
% Written By Michael Brodskiy
% Class: Thermodynamics & Statistical Mechanics
% Professor: A. Stepanyants
%%%%%%%%%%%%%%%%%%%%%%%%%%%%%%%%%%%%%%%%%%%%%%%%%%%%%%%%%%%%%%%%%%%%%%%%%%%%%%%%%%%%%%%%%%%%%%%%%%%%%%%%%%%%%%%%%%%%%%%%%%%%%%%%%%%%%%%%%%%%%%%%%%%%%%%%%%%%%%%%%%%

\documentclass[12pt]{article} 
\usepackage{alphalph}
\usepackage[utf8]{inputenc}
\usepackage[russian,english]{babel}
\usepackage{titling}
\usepackage{amsmath}
\usepackage{graphicx}
\usepackage{enumitem}
\usepackage{amssymb}
\usepackage[super]{nth}
\usepackage{everysel}
\usepackage{ragged2e}
\usepackage{geometry}
\usepackage{multicol}
\usepackage{fancyhdr}
\usepackage{cancel}
\usepackage{siunitx}
\usepackage{physics}
\usepackage{tikz}
\usepackage{mathdots}
\usepackage{yhmath}
\usepackage{cancel}
\usepackage{color}
\usepackage{array}
\usepackage{multirow}
\usepackage{gensymb}
\usepackage{tabularx}
\usepackage{extarrows}
\usepackage{booktabs}
\usepackage{lastpage}
\usetikzlibrary{fadings}
\usetikzlibrary{patterns}
\usetikzlibrary{shadows.blur}
\usetikzlibrary{shapes}

\geometry{top=1.0in,bottom=1.0in,left=1.0in,right=1.0in}
\newcommand{\subtitle}[1]{%
  \posttitle{%
    \par\end{center}
    \begin{center}\large#1\end{center}
    \vskip0.5em}%

}
\usepackage{hyperref}
\hypersetup{
colorlinks=true,
linkcolor=blue,
filecolor=magenta,      
urlcolor=blue,
citecolor=blue,
}


\title{Gibbs Free Energy and Chemical Reactions}
\date{\today}
\author{Michael Brodskiy\\ \small Professor: A. Stepanyants}

\begin{document}

\maketitle

\begin{itemize}

  \item The Gibbs free energy is a function of $\tau$, $P$, and $N\to G(\tau,P,N)$

  \item If $S$ is brought in thermal and mechanical contact with large $R$ (at constant $P$ and $\tau$), then $G_s$ will decrease and $S$ will come to an equilibrium state in which $G_s$ is minimal

    \begin{enumerate}

      \item Maximum effective work done by the system in a reversible process is equal to -$\Delta G$

      \item $G=U+PV-\tau\sigma$

        $$dG=dU+P\,dV+V\,dP-\tau\,d\sigma-\sigma\,d\tau$$

        \begin{itemize}

          \item From the first law for a reversible process:

            $$dG=\mu\,dN+v\,dP-\sigma\,d\tau$$

        \end{itemize}

      \item From this, we can find:

        $$G(N,P,\tau)\to\left\{\begin{array}{l} \left( \dfrac{\partial G}{\partial N} \right)_{P,\tau}=\mu\\\left( \dfrac{\partial G}{\partial P} \right)_{N,\tau}=V\\ \left( \dfrac{\partial G}{\partial \tau}\right)_{N,P}=-\sigma\end{array}$$

        \item $U=U(\sigma,V,N)=Nf\left( \frac{\sigma}{N},\frac{V}{N} \right)$ — this is an extensive function

          $$G(N,P,\tau)=N\mu(P,\tau)$$

          \begin{itemize}

            \item We can see that $\mu$ is the Gibbs free energy per particle

          \end{itemize}

        \item For an ideal gas ($S=0$, monatomic) we know that $\mu\tau\ln\left( n/n_Q \right)$; this can be rewritten as:

          $$\mu(P,\tau)=\tau\ln\left( \frac{P}{\tau n_Q} \right)$$
          $$G(N,P,\tau)=N\tau\ln\left( \frac{P}{\tau n_Q} \right)$$

    \end{enumerate}

  \item Chemical reactions at $\tau,P$ — constant

    \begin{center}
      \ce{v_1A_1 + v_2A_2 + $\ldots$ + v_lA_l=0}
    \end{center}

    \begin{itemize}

      \item Where $v_i$ are the reaction coefficients

      \item $A_i$ are the reaction species

    \end{itemize}

  \item Example:

    \begin{center}
      \ce{2H2 + O2 <-> 2H2O}
      \ce{2H2 + O2 - 2H2O = 0}
    \end{center}

    \begin{itemize}

      \item We can find $v_1=2$, $v_2=1$, $v_3=-2$ and $A_1=\ce{H2}$, $A_2=\ce{O2}$, $A_3=\ce{H2O}$

    \end{itemize}

  \item In equilibrium, $G(N,P,\tau)$ will be at its minimum and $dG=0$

    $$dG=\sum_{i=1}^l\mu_i\,dN_i=0$$
    $$dN_i=-\Delta N v_i=0\quad\text{in equilibrium}$$
    $$\Delta G=-\Delta N\sum_{i=1}^l\mu_iv_i$$

  \item Moving away from equilibrium:

    $$\Delta G=-\Delta N\sum\mu_iv_i<0\quad\text{(\nth{2} law)}$$

  \item If $\sum\mu_iv_i>0$, then $\Delta N>0$, and the reaction will move to the right

  \item Ideal Gas with internal degrees of freedom

    \begin{itemize}

      \item Spin ($S$)

      \item Vibrations

      \item Rotations

    \end{itemize}

\end{itemize}

\end{document}




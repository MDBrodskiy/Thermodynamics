%%%%%%%%%%%%%%%%%%%%%%%%%%%%%%%%%%%%%%%%%%%%%%%%%%%%%%%%%%%%%%%%%%%%%%%%%%%%%%%%%%%%%%%%%%%%%%%%%%%%%%%%%%%%%%%%%%%%%%%%%%%%%%%%%%%%%%%%%%%%%%%%%%%%%%%%%%%%%%%%%%%
% Written By Michael Brodskiy
% Class: Thermodynamics & Statistical Mechanics
% Professor: A. Stepanyants
%%%%%%%%%%%%%%%%%%%%%%%%%%%%%%%%%%%%%%%%%%%%%%%%%%%%%%%%%%%%%%%%%%%%%%%%%%%%%%%%%%%%%%%%%%%%%%%%%%%%%%%%%%%%%%%%%%%%%%%%%%%%%%%%%%%%%%%%%%%%%%%%%%%%%%%%%%%%%%%%%%%

\documentclass[12pt]{article} 
\usepackage{alphalph}
\usepackage[utf8]{inputenc}
\usepackage[russian,english]{babel}
\usepackage{titling}
\usepackage{amsmath}
\usepackage{graphicx}
\usepackage{enumitem}
\usepackage{amssymb}
\usepackage[super]{nth}
\usepackage{everysel}
\usepackage{ragged2e}
\usepackage{geometry}
\usepackage{multicol}
\usepackage{fancyhdr}
\usepackage{cancel}
\usepackage{siunitx}
\usepackage{physics}
\usepackage{tikz}
\usepackage{mathdots}
\usepackage{yhmath}
\usepackage{cancel}
\usepackage{color}
\usepackage{array}
\usepackage{multirow}
\usepackage{gensymb}
\usepackage{tabularx}
\usepackage{extarrows}
\usepackage{booktabs}
\usepackage{lastpage}
\usetikzlibrary{fadings}
\usetikzlibrary{patterns}
\usetikzlibrary{shadows.blur}
\usetikzlibrary{shapes}

\geometry{top=1.0in,bottom=1.0in,left=1.0in,right=1.0in}
\newcommand{\subtitle}[1]{%
  \posttitle{%
    \par\end{center}
    \begin{center}\large#1\end{center}
    \vskip0.5em}%

}
\usepackage{hyperref}
\hypersetup{
colorlinks=true,
linkcolor=blue,
filecolor=magenta,      
urlcolor=blue,
citecolor=blue,
}


\title{Boltzmann Distribution and Helmholtz Free Energy}
\date{\today}
\author{Michael Brodskiy\\ \small Professor: A. Stepanyants}

\begin{document}

\maketitle

\begin{itemize}

  \item For a system ($S$) in reservoir ($\mathcal{R}$), we can assume:

    \begin{itemize}

      \item $S+\mathcal{R}$ is closed

      \item $S$ and $\mathcal{R}$ are in thermal equilibrium; that is:

        $$\tau_S=\tau_{\mathcal{R}}\cong \tau$$

      \item $U_s+U_{\mathcal{R}}\cong U_o$ (the total energy)

      \item $U_s << U_{\mathcal{R}}, U_o$

      \item $S$ is in a quantum state, $s$, of energy $\varepsilon_s$

        $$U_{\mathcal{R}}=\langle \varepsilon_s\rangle$$

      \item The probability $P(\varepsilon_s)$ to observe $S$ in a quantum state $s$ is:

        $$g_{\mathcal{R}+S}=\sum_{\varepsilon_s}g_{\mathcal{R}}(U_o-\varepsilon_s,V_{\mathcal{R}})g_S(\varepsilon_s,V_s)=g_{\mathcal{R}}(U_o-\varepsilon_s,V_R)$$
        $$P(\varepsilon_s)\approx g_R(U_o-\varepsilon_s)=e^{\ln(g_{\mathcal{R}}(U_o-\varepsilon_s))}=e^{\sigma_{\mathcal{R}}(U_o-\varepsilon_s)}$$

        Using expansion we can transform this to:

        $$e^{\sigma_{\mathcal{R}}(U_o-\varepsilon_s)}\approx e^{\overbrace{\sigma_{\mathcal{R}}(U_o)}^{\text{constant}}-\frac{\partial \sigma_{\mathcal{R}}}{\partial U_{\mathcal{R}}}\varepsilon_s}\approx e^{-\frac{\partial \sigma_{\mathcal{R}}}{\partial U_{\mathcal{R}}}\varepsilon_s}=e^{-\frac{\varepsilon_s}{\tau}}$$
        $$\boxed{P(\varepsilon_s)\approx e^{-\frac{\varepsilon_s}{\tau}}}$$
        $$\boxed{P(\varepsilon_s)= \frac{1}{z}e^{-\frac{\varepsilon_s}{\tau}}}$$

        This is known as the Boltzmann factor; the $z$ is known as the partition function. Given statistical definitions, we know:

        $$\sum_s P(\varepsilon_s)=\frac{1}{z}\sum_s e^{-\frac{\varepsilon_s}{\tau}}=1$$

        This gets us:

        $$\boxed{z=\sum_se^{-\frac{\varepsilon_s}{\tau}}}$$

      \item Ensemble or thermal average energy of $S$, $U$:

        $$U=\langle \varepsilon_s\rangle=\sum_s\varepsilon_sP(\varepsilon_s)=\sum_s \varepsilon_s\frac{e^{-\frac{\varepsilon_s}{\tau}}}{z}$$
        $$\frac{1}{z}\frac{\partial z}{\partial \tau}=\frac{\partial \ln(z)}{\partial \tau}=\frac{1}{\tau^2}\sum_s\varepsilon_s\frac{e^{-\frac{\varepsilon_s}{\tau}}}{z}\rightarrow U=\tau^2\frac{\partial \ln(z)}{\partial \tau}$$

        Thus, the formula is finalized to:

        $$\boxed{U(\tau, V)=\tau^2\left( \frac{\partial \ln(z(\tau,V))}{\partial \tau} \right)_V}$$

    \end{itemize}
    
  \item Pressure

    \begin{itemize}

      \item Consider once more a similar configuration of a reservoir and system

      \item We can change the volume such that $S:\, V\to V-\Delta V$; compression is done slowly, and $S$ remains in the same quantum state, $s$; therefore, $\sigma(\varepsilon_s)$ remains unchanged. This means it is an isentropic, reversible process

      \item The work done on $S$:

        $$W=\underbrace{P}_{\text{pressure}}\cdot \underbrace{A}_{\text{area}}\cdot \underbrace{\Delta x}_{\text{displacement}}=P\Delta V$$

        From conservation of energy:

        $$W_{on}=U(V-\Delta V)-U(V)$$
        $$P\Delta V=-\frac{\partial U}{\partial V}\Delta V$$
        $$P=-\left( \frac{\partial U}{\partial V} \right)_{\sigma}$$

    \end{itemize}

  \item Thermodynamic Identity

    $$\sigma=\sigma(U,V)$$
    $$d\sigma=\frac{\partial \sigma}{\partial U}dU+\frac{\partial \sigma}{\partial V}dV$$
    $$\text{If }d\sigma=0\text{ (isentropic) }\to\left(\frac{\partial U}{\partial V}\right)_{\sigma}=-\frac{\frac{\partial \sigma}{\partial V}}{\frac{\partial \sigma}{\partial U}}$$

  \item Reviewing some important formulas:

    \begin{itemize}

      \item Entropy:

        $$\sigma(u,v)=\ln(g(u,v))$$

      \item Fundamental Temperature:

        $$\frac{1}{\tau}=\left( \frac{\partial \sigma}{\partial U} \right)_V$$

      \item Probability:

        $$P(\varepsilon_s)=\frac{e^{-\frac{\varepsilon_s}{\tau}}}{z}$$

      \item Partition Function:

        $$z(\tau,V)=\sum_s e^{-\frac{\varepsilon_s}{\tau}}$$

      \item Energy:

        $$\tau^2\left( \frac{\partial\ln(z)}{\partial\tau} \right)_V$$

      \item Pressure:

        $$P=-\left( \frac{\partial U}{\partial V} \right)_{\sigma}$$

      \item Note: there may be other ways to derive some of these quantities; however, these are the critical functions we have learned so far.

    \end{itemize}

  \item Back to the total differential, we get:

    $$d\sigma=\frac{\partial \sigma}{\partial U}dU+\frac{\partial \sigma}{\partial V}dV$$

    \begin{itemize}

      \item For $\sigma=c$, a constant, $d\sigma=0$, known as an isentropic process

        $$\frac{\partial U}{\partial V}_{\sigma}=-\frac{\left(\frac{\partial \sigma\right)_U}{\partial V}}{\left(\frac{\partial\sigma}{\partial U}\right)_V}=-\tau\left( \frac{\partial \sigma}{\partial V} \right)_U$$

        Thus, we see:

        $$P=\tau\left( \frac{\partial \sigma}{\partial V} \right)_U\footnote{Note: the negative sign is dropped because entropy can only increase, and pressure is always positive}$$

    \end{itemize}

  \item Thermodynamic Identity
        
    $$d\sigma=\frac{1}{\tau}dU+\frac{P}{\tau}dV$$
    $$\boxed{\underbrace{\tau}_{\text{heat added\footnote{to $S$ in a reversible process}}}\hspace{-12pt}d\sigma=\hspace{-12pt}\underbrace{dU}_{\Delta\text{internal energy}}\hspace{-6pt}+P\hspace{-10pt}\underbrace{dV}_{\text{work done by gas}}}$$
    $$sQ=dU=sW$$

    \begin{itemize}

      \item This is the first law

      \item Helmholtz Free Energy, $F(\tau,V)$

        \begin{itemize}

          \item For a Reservoir ($\mathcal{R}$) and System ($S$) placed in thermal contact:

            $$U_{\mathcal{R}}+U_S=U_o$$
            $$\sigma_{\mathcal{R}+S}=\sigma_{\mathcal{R}}\left( U_{\mathcal{R}} \right)+\sigma_S\left( U_S \right)$$
            $$\sigma_{\mathcal{R}+S}=\sigma_{\mathcal{R}}\left( U_o -U_S \right)+\sigma_S\left( U_S \right)$$

          \item Using Taylor Expansion:

            $$\sigma_{\mathcal{R}}-\frac{1}{\tau_{\mathcal{R}}}\left( U_s-\tau_{\mathcal{R}}\cdot\sigma_S \right)$$

          \item As $\sigma_{\mathcal{R}+s}$ increases,

            $$F_S\equiv U_s-\tau_{\mathcal{R}}\sigma_S$$

            decreases, assuming that $\tau_{\mathcal{R}},V_s,N_s$ are constant

        \end{itemize}

      \item To summarize this process, we can say: If $S$ is brought into thermal contact with $\mathcal{R}$ at $\tau_{\mathcal{R}},V_s,N_s$ constant, then $F_S$ will decrease and $S$ will come into a new equilibrium state in which $F_S$ is minimal, $F_S^{min}$

        \begin{itemize}

          \item $F_S-F_S^{min}$ — The maximum amount of work $S$ can perform

          \item This is because, at the same time as the system is minimizing the free energy, it is also trying to maximize entropy (as seen in the subtracted term of the formula)

        \end{itemize}

      \item Identities Related to Helmholtz Free Energy

        $$F(\tau,V)=U-\tau\sigma$$
        $$dF=dU-\sigma d\tau-\tau d\sigma$$
        $$=\cancel{\tau d\sigma}- PdV-\sigma d\tau-\cancel{\tau d\sigma}$$
        $$=-\sigma d\tau-PdV$$

        \begin{itemize}

          \item This gives us:

        \end{itemize}

        $$\sigma=-\left( \frac{\partial F}{\partial \tau} \right)_v\quad\quad P=-\left( \frac{\partial F}{\partial V} \right)_{\tau}$$

      \item Maxwell Relations

        \begin{itemize}

          \item Consider $f(x,y)$ which is twice continuously differentiable in $x$ and $y$

            $$df=\left(\frac{\partial f}{\partial x}\right)_ydx+\left( \frac{\partial f}{\partial y} \right)_xdy$$
            $$\frac{\partial}{\partial x}\left( \frac{\partial f}{\partial y} \right)_x=\frac{\partial}{\partial y}\left( \frac{\partial f}{\partial x} \right)_y$$

          \item Applying this to the above, we get:

            $$\frac{\partial}{\partial\tau}\left( \frac{\partial F}{\partial V} \right)=\frac{\partial}{\partial \tau}(-P)$$
            $$\frac{\partial}{\partial V}\left( \frac{\partial F}{\partial \tau} \right)=\frac{\partial}{\partial V}(-\sigma)$$
            $$\left( \frac{\partial P}{\partial \tau} \right)_V=\left(\frac{\partial\sigma}{\partial V}\right)_{\tau}$$

        \end{itemize}

      \item Calculation of $F$ from $z$:

        $$F=-\tau\ln(z)$$
        $$z=e^{-\frac{F}{\tau}}$$
        $$P(\varepsilon_s)=e^{-\frac{F\varepsilon_s}{\tau}}$$

      \item Ideal Gas:

        \begin{itemize}

          \item A gas of non-interacting quantum particles in a classical regime

          \item Considering a single quantum particle in a box:

            $$-\frac{\hbar^2}{2M}\left( \frac{\partial^2}{\partial x^2}+\frac{\partial^2}{\partial y^2}+\frac{\partial^2}{\partial z^2} \right)\psi(x,y,z)=\varepsilon\psi(x,y,z)$$
            $$\varepsilon_{n_x,n_y,n_z}=\frac{\hbar^2\pi^2}{2ML^2}\left( n_x^2+n_y^2+n_z^2 \right),\quad n_x,n_y,n_z=1,2,3,\ldots$$
            $$\psi_{n_x,n_y,n_z}(x,y,z)=A\sin\left(\frac{\pi n_x}{L}x\right)\sin\left(\frac{\pi n_y}{L}y\right)\sin\left(\frac{\pi n_z}{L}z\right)$$
            $$z_1=\sum_{n_xn_yn_z=1}^{\infty}e^{-\frac{\hbar^2\pi^2}{2ML^2\tau}\left( n_x^2+n_y^2+n_z^2 \right)}$$
            $$\iiint_0^{\infty}dn_xdn_ydn_z e^{-\alpha^2(n_x^2+n_y^2+n_z^2)}\quad\quad\alpha\equiv\sqrt{\frac{\hbar^2\pi^2}{2ML^2\tau}} << 1$$

            \begin{itemize}

              \item These become Gaussian integrals:

                $$\int_{-\infty}^{\infty}e^{-x^2}\,dx\equiv\sqrt{\pi}$$
                $$\int_{0}^{\infty}e^{-x^2}\,dx\equiv\frac{\sqrt{\pi}}{2}$$
                $$\frac{1}{2}\int_{0}^{\infty}e^{-\alpha^2n_x^2}\,dn_x\equiv\frac{\sqrt{\pi}}{2\alpha}$$

              \item Thus, we get:

                $$z_1=\left(  \frac{\sqrt{\pi}}{2\alpha}\right)^3=\left( \frac{\sqrt{\pi}}{2\sqrt{\frac{\hbar^2\pi^2}{2ML^2\tau}}} \right)^3=\left(\frac{M\tau}{2\pi\hbar^2}\right)^\frac{3}{2}V$$

              \item The following value is known as the quantum concentration

                $$n_Q\equiv\left( \frac{M\tau}{2\pi\hbar^2} \right)^{\frac{3}{2}}$$
                $$n_Q=\frac{1}{\lambda_D^3}$$

                where $\lambda_D$ is the de Broglie wavelength

              \item Finally, we are left with:

                $$z_1=n_QV$$
                $$n_Q<<n$$
                $$\text{Classical regime:}\quad n_Q>>n=\frac{N}{V}$$

            \end{itemize}

        \end{itemize}

    \end{itemize}

\end{itemize}

\end{document}


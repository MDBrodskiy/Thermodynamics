%%%%%%%%%%%%%%%%%%%%%%%%%%%%%%%%%%%%%%%%%%%%%%%%%%%%%%%%%%%%%%%%%%%%%%%%%%%%%%%%%%%%%%%%%%%%%%%%%%%%%%%%%%%%%%%%%%%%%%%%%%%%%%%%%%%%%%%%%%%%%%%%%%%%%%%%%%%%%%%%%%%
% Written By Michael Brodskiy
% Class: Thermodynamics & Statistical Mechanics
% Professor: A. Stepanyants
%%%%%%%%%%%%%%%%%%%%%%%%%%%%%%%%%%%%%%%%%%%%%%%%%%%%%%%%%%%%%%%%%%%%%%%%%%%%%%%%%%%%%%%%%%%%%%%%%%%%%%%%%%%%%%%%%%%%%%%%%%%%%%%%%%%%%%%%%%%%%%%%%%%%%%%%%%%%%%%%%%%

\documentclass[12pt]{article} 
\usepackage{alphalph}
\usepackage[utf8]{inputenc}
\usepackage[russian,english]{babel}
\usepackage{titling}
\usepackage{amsmath}
\usepackage{graphicx}
\usepackage{enumitem}
\usepackage{amssymb}
\usepackage[super]{nth}
\usepackage{everysel}
\usepackage{ragged2e}
\usepackage{geometry}
\usepackage{multicol}
\usepackage{fancyhdr}
\usepackage{cancel}
\usepackage{siunitx}
\usepackage{physics}
\usepackage{tikz}
\usepackage{mathdots}
\usepackage{yhmath}
\usepackage{cancel}
\usepackage{color}
\usepackage{array}
\usepackage{multirow}
\usepackage{gensymb}
\usepackage{tabularx}
\usepackage{extarrows}
\usepackage{booktabs}
\usepackage{lastpage}
\usetikzlibrary{fadings}
\usetikzlibrary{patterns}
\usetikzlibrary{shadows.blur}
\usetikzlibrary{shapes}

\geometry{top=1.0in,bottom=1.0in,left=1.0in,right=1.0in}
\newcommand{\subtitle}[1]{%
  \posttitle{%
    \par\end{center}
    \begin{center}\large#1\end{center}
    \vskip0.5em}%

}
\usepackage{hyperref}
\hypersetup{
colorlinks=true,
linkcolor=blue,
filecolor=magenta,      
urlcolor=blue,
citecolor=blue,
}


\title{Boltzmann Distribution and Helmholtz Free Energy}
\date{\today}
\author{Michael Brodskiy\\ \small Professor: A. Stepanyants}

\begin{document}

\maketitle

\begin{itemize}

  \item For a system ($S$) in reservoir ($\mathcal{R}$), we can assume:

    \begin{itemize}

      \item $S+\mathcal{R}$ is closed

      \item $S$ and $\mathcal{R}$ are in thermal equilibrium; that is:

        $$\tau_S=\tau_{\mathcal{R}}\cong \tau$$

      \item $U_s+U_{\mathcal{R}}\cong U_o$ (the total energy)

      \item $U_s << U_{\mathcal{R}}, U_o$

      \item $S$ is in a quantum state, $s$, of energy $\varepsilon_s$

        $$U_{\mathcal{R}}=\langle \varepsilon_s\rangle$$

      \item The probability $P(\varepsilon_s)$ to observe $S$ in a quantum state $s$ is:

        $$g_{\mathcal{R}+S}=\sum_{\varepsilon_s}g_{\mathcal{R}}(U_o-\varepsilon_s,V_{\mathcal{R}})g_S(\varepsilon_s,V_s)=g_{\mathcal{R}}(U_o-\varepsilon_s,V_R)$$
        $$P(\varepsilon_s)\approx g_R(U_o-\varepsilon_s)=e^{\ln(g_{\mathcal{R}}(U_o-\varepsilon_s))}=e^{\sigma_{\mathcal{R}}(U_o-\varepsilon_s)}$$

        Using expansion we can transform this to:

        $$e^{\sigma_{\mathcal{R}}(U_o-\varepsilon_s)}\approx e^{\overbrace{\sigma_{\mathcal{R}}(U_o)}^{\text{constant}}-\frac{\partial \sigma_{\mathcal{R}}}{\partial U_{\mathcal{R}}}\varepsilon_s}\approx e^{-\frac{\partial \sigma_{\mathcal{R}}}{\partial U_{\mathcal{R}}}\varepsilon_s}=e^{-\frac{\varepsilon_s}{\tau}}$$
        $$\boxed{P(\varepsilon_s)\approx e^{-\frac{\varepsilon_s}{\tau}}}$$
        $$\boxed{P(\varepsilon_s)= \frac{1}{z}e^{-\frac{\varepsilon_s}{\tau}}}$$

        This is known as the Boltzmann factor; the $z$ is known as the partition function. Given statistical definitions, we know:

        $$\sum_s P(\varepsilon_s)=\frac{1}{z}\sum_s e^{-\frac{\varepsilon_s}{\tau}}=1$$

        This gets us:

        $$\boxed{z=\sum_se^{-\frac{\varepsilon_s}{\tau}}}$$

      \item Ensemble or thermal average energy of $S$, $U$:

        $$U=\langle \varepsilon_s\rangle=\sum_s\varepsilon_sP(\varepsilon_s)=\sum_s \varepsilon_s\frac{e^{-\frac{\varepsilon_s}{\tau}}}{z}$$
        $$\frac{1}{z}\frac{\partial z}{\partial \tau}=\frac{\partial \ln(z)}{\partial \tau}=\frac{1}{\tau^2}\sum_s\varepsilon_s\frac{e^{-\frac{\varepsilon_s}{\tau}}}{z}\rightarrow U=\tau^2\frac{\partial \ln(z)}{\partial \tau}$$

        Thus, the formula is finalized to:

        $$\boxed{U(\tau, V)=\tau^2\left( \frac{\partial \ln(z(\tau,V))}{\partial \tau} \right)_V}$$

    \end{itemize}
    
  \item Pressure

    \begin{itemize}

      \item Consider once more a similar configuration of a reservoir and system

      \item We can change the volume such that $S:\, V\to V-\Delta V$; compression is done slowly, and $S$ remains in the same quantum state, $s$; therefore, $\sigma(\varepsilon_s)$ remains unchanged. This means it is an isentropic, reversible process

      \item The work done on $S$:

        $$W=\underbrace{P}_{\text{pressure}}\cdot \underbrace{A}_{\text{area}}\cdot \underbrace{\Delta x}_{\text{displacement}}=P\Delta V$$

        From conservation of energy:

        $$W_{on}=U(V-\Delta V)-U(V)$$
        $$P\Delta V=-\frac{\partial U}{\partial V}\Delta V$$
        $$P=-\left( \frac{\partial U}{\partial V} \right)_{\sigma}$$

    \end{itemize}

  \item Thermodynamic Identity

    $$\sigma=\sigma(U,V)$$
    $$d\sigma=\frac{\partial \sigma}{\partial U}dU+\frac{\partial \sigma}{\partial V}dV$$
    $$\text{If }d\sigma=0\text{ (isentropic) }\to\left(\frac{\partial U}{\partial V}\right)_{\sigma}=-\frac{\frac{\partial \sigma}{\partial V}}{\frac{\partial \sigma}{\partial U}}$$

\end{itemize}

\end{document}


%%%%%%%%%%%%%%%%%%%%%%%%%%%%%%%%%%%%%%%%%%%%%%%%%%%%%%%%%%%%%%%%%%%%%%%%%%%%%%%%%%%%%%%%%%%%%%%%%%%%%%%%%%%%%%%%%%%%%%%%%%%%%%%%%%%%%%%%%%%%%%%%%%%%%%%%%%%%%%%%%%%
% Written By Michael Brodskiy
% Class: Thermodynamics & Statistical Mechanics
% Professor: A. Stepanyants
%%%%%%%%%%%%%%%%%%%%%%%%%%%%%%%%%%%%%%%%%%%%%%%%%%%%%%%%%%%%%%%%%%%%%%%%%%%%%%%%%%%%%%%%%%%%%%%%%%%%%%%%%%%%%%%%%%%%%%%%%%%%%%%%%%%%%%%%%%%%%%%%%%%%%%%%%%%%%%%%%%%

\include{Includes.tex}

\title{Phase Transitions}
\date{\today}
\author{Michael Brodskiy\\ \small Professor: A. Stepanyants}

\begin{document}

\maketitle

\begin{itemize}

  \item First-Order Phase Transitions

    \begin{itemize}

      \item Three Phases: Solid, Liquid, Gas (Vapor)

      \item Liquid and Gas/Vapor are known as fluids

      \item From liquid to gas, solid to gas, and solid to liquid, the process would be isothermal

      \item Phase Changes:

        \begin{itemize}

          \item Solid to Liquid $\to$ Melting

          \item Liquid to Solid $\to$ Crystalization

          \item Solid to Gas $\to$ Sublimation

          \item Gas to Solid $\to$ Deposition

          \item Liquid to Gas $\to$ Boiling

          \item Gas to Liquid $\to$ Condensation

        \end{itemize}

    \end{itemize}

  \item Coexistence of two phases ($l$ and $g$)

    $$G=G_l+G_g=N_l\mu_l(P,\tau)+N_g\mu_g(P,\tau)$$
    $$N_g+N_l=N\quad\text{(constant)}$$

    \begin{itemize}

      \item This expression may be rewritten as:

        $$G=N_l(\mu_l-\mu_g)+N\mu_g$$

      \item If $l$ and $g$ coexist, then:

        \begin{itemize}

          \item $\tau_l=\tau_g$ (thermal equilibrium)

          \item $P_l=P_g$ (mechanical equilibrium)

          \item $\mu_l=\mu_g$ (chemical equilibrium)

        \end{itemize}

      \item Thus, for coexistence of $l$ and $g$, we may write:

        $$\mu_l(\tau,P)=\mu_g(\tau,P)$$

      \item The pressure differential with respect to $\tau$ may be written as:

        $$\frac{dP}{d\tau}=\frac{\left( \frac{\partial\mu_l}{\partial\tau} \right)_P-\left( \frac{\partial\mu_g}{\partial\tau} \right)_P}{\left( \frac{\partial\mu_g}{\partial\tau} \right)_\tau-\left( \frac{\partial\mu_l}{\partial\tau} \right)_\tau}$$

      \item Furthermore, implementing pressure and volume per molecules ($s$ and $v$), we may write:

        $$\frac{dP}{dt}=\frac{s_g-s_l}{v_g-v_l}$$

    \end{itemize}

  \item We can recall that, in a reversible process:

    $$\delta Q=\tau\,d\sigma$$
    $$\Delta Q=\tau\Delta\sigma$$
    $$\Delta Q=\tau(s_g-s_l)$$

    \begin{itemize}

      \item Thus, we may described the heat gained from moving 1 molecule from $g$ to $l$

      \item This $\Delta Q$ term is referred to as $L$, or the latent heat of vaporization per molecule

    \end{itemize}

  \item This leads us to the Clausius-Clapeyron Equation:

    $$\frac{dP}{d\tau}=\frac{L}{\tau(v_g-v_l)}$$

    \begin{itemize}

      \item This is used in liquid-gas coexistence

      \item We assume the gas is ideal

      \item We also assume $v_l<<v_g$

      \item Also, $L$ is assumed to be constant

      \item In assuming this, we may write:

        $$\frac{dP}{d\tau}=\frac{LP}{\tau^2}$$

      \item Solving this, we get:

        $$P(\tau)=P_oe^{-\frac{L}{\tau}}$$

    \end{itemize}

  \item Van der Waals Equation of state

    \begin{itemize}

      \item As a reminder, the free energy for an ideal gas is:

        $$F_{ideal}=-N\tau\left( \ln\left( \frac{n_Q}{n} \right)+1 \right)$$
        $$P=-\left( \frac{\partial F}{\partial V} \right)_{\tau,N}\to PV=N\tau$$

      \item In the above, $V$ must be replaced with $V-bN$, where $b$ is the volume of a single molecule, and $N$ is the total number of molecules

      \item To find the interaction between particles, we use a mean field approximation

        $$\Delta U=\frac{1}{2}\sum_{i\neq j=1}^N\phi(|\bar{r}_i-\bar{r}_j|)=\frac{1}{2}(N^2-N)\langle\phi(r)\rangle$$

      \item Assume that the molecules are uniformly distributed in the accessible volume, $V-bn$, and average overall the positions of all molecules

      \item We can approximate it to:

        $$\frac{1}{2}(N^2-N)\int_V\frac{\phi(r)}{V-bN}\,d^3r$$
        $$\Delta U=-\frac{N^2a}{V},\quad a>0$$

      \item from here, we can write the free energy:

        $$F_{VdW}=-N\tau\left( \ln\left( \frac{n_Q(V-bN)}{N} \right)+1 \right)-\frac{N^2a}{V},\quad a,b>0$$
        $$P=\frac{\tau N}{V-bN}-\frac{N^2a}{V^2}$$

      \item The equation of state may be simplified as:

        $$\left( P+\frac{N^2a}{V} \right)(V-bN)=N\tau$$

        \begin{itemize}

          \item The term dependent on $a$ is the reduction in pressure due to attractive interactions between molecules

          \item The term dependent on $b$ is the excluded volume due to molecule volume

        \end{itemize}

      \item Critical points may be defined as:

        $$\tau_c=\frac{8a}{27b}$$
        $$P_c=\frac{a}{27b^2}$$
        $$V_c=3Nb$$

      \item The Gibbs Free Energy for this Van der Waals gas is:

        $$G=F+PV=F+P(V-bN)$$
        $$G=-N\tau\ln\left( \frac{n_Q(V-bN)}{N} \right)-\frac{2N^2a}{V}$$

      \item We find the following relations:

        $$\left( \frac{\partial G}{\partial P} \right)_{\tau,N}=\frac{V}{N}$$
        $$\left( \frac{\partial G}{\partial \tau} \right)_{P,N}=-\frac{\sigma}{N}=-S$$

    \end{itemize}

  \item Ferromagnetism

    \begin{itemize}

      \item There are $N$ magnetic moments $\mu$ in a binary-spin system. From this, we get:

        $$U=-\muB$$

      \item Assume that $M$ is the magnetization of the volume — net magnetic moment per volume

        $$B_E=\lambda M$$

      \item Is the effective magnetic field due to $M$ (mean field approximation)

    \end{itemize}

\end{itemize}

\end{document}




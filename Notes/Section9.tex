%%%%%%%%%%%%%%%%%%%%%%%%%%%%%%%%%%%%%%%%%%%%%%%%%%%%%%%%%%%%%%%%%%%%%%%%%%%%%%%%%%%%%%%%%%%%%%%%%%%%%%%%%%%%%%%%%%%%%%%%%%%%%%%%%%%%%%%%%%%%%%%%%%%%%%%%%%%%%%%%%%%
% Written By Michael Brodskiy
% Class: Thermodynamics & Statistical Mechanics
% Professor: A. Stepanyants
%%%%%%%%%%%%%%%%%%%%%%%%%%%%%%%%%%%%%%%%%%%%%%%%%%%%%%%%%%%%%%%%%%%%%%%%%%%%%%%%%%%%%%%%%%%%%%%%%%%%%%%%%%%%%%%%%%%%%%%%%%%%%%%%%%%%%%%%%%%%%%%%%%%%%%%%%%%%%%%%%%%

\documentclass[12pt]{article} 
\usepackage{alphalph}
\usepackage[utf8]{inputenc}
\usepackage[russian,english]{babel}
\usepackage{titling}
\usepackage{amsmath}
\usepackage{graphicx}
\usepackage{enumitem}
\usepackage{amssymb}
\usepackage[super]{nth}
\usepackage{everysel}
\usepackage{ragged2e}
\usepackage{geometry}
\usepackage{multicol}
\usepackage{fancyhdr}
\usepackage{cancel}
\usepackage{siunitx}
\usepackage{physics}
\usepackage{tikz}
\usepackage{mathdots}
\usepackage{yhmath}
\usepackage{cancel}
\usepackage{color}
\usepackage{array}
\usepackage{multirow}
\usepackage{gensymb}
\usepackage{tabularx}
\usepackage{extarrows}
\usepackage{booktabs}
\usepackage{lastpage}
\usetikzlibrary{fadings}
\usetikzlibrary{patterns}
\usetikzlibrary{shadows.blur}
\usetikzlibrary{shapes}

\geometry{top=1.0in,bottom=1.0in,left=1.0in,right=1.0in}
\newcommand{\subtitle}[1]{%
  \posttitle{%
    \par\end{center}
    \begin{center}\large#1\end{center}
    \vskip0.5em}%

}
\usepackage{hyperref}
\hypersetup{
colorlinks=true,
linkcolor=blue,
filecolor=magenta,      
urlcolor=blue,
citecolor=blue,
}


\title{Phase Transitions}
\date{\today}
\author{Michael Brodskiy\\ \small Professor: A. Stepanyants}

\begin{document}

\maketitle

\begin{itemize}

  \item First-Order Phase Transitions

    \begin{itemize}

      \item Three Phases: Solid, Liquid, Gas (Vapor)

      \item Liquid and Gas/Vapor are known as fluids

      \item From liquid to gas, solid to gas, and solid to liquid, the process would be isothermal

      \item Phase Changes:

        \begin{itemize}

          \item Solid to Liquid $\to$ Melting

          \item Liquid to Solid $\to$ Crystalization

          \item Solid to Gas $\to$ Sublimation

          \item Gas to Solid $\to$ Deposition

          \item Liquid to Gas $\to$ Boiling

          \item Gas to Liquid $\to$ Condensation

        \end{itemize}

    \end{itemize}

  \item Coexistence of two phases ($l$ and $g$)

    $$G=G_l+G_g=N_l\mu_l(P,\tau)+N_g\mu_g(P,\tau)$$
    $$N_g+N_l=N\quad\text{(constant)}$$

    \begin{itemize}

      \item This expression may be rewritten as:

        $$G=N_l(\mu_l-\mu_g)+N\mu_g$$

      \item If $l$ and $g$ coexist, then:

        \begin{itemize}

          \item $\tau_l=\tau_g$ (thermal equilibrium)

          \item $P_l=P_g$ (mechanical equilibrium)

          \item $\mu_l=\mu_g$ (chemical equilibrium)

        \end{itemize}

      \item Thus, for coexistence of $l$ and $g$, we may write:

        $$\mu_l(\tau,P)=\mu_g(\tau,P)$$

      \item The pressure differential with respect to $\tau$ may be written as:

        $$\frac{dP}{d\tau}=\frac{\left( \frac{\partial\mu_l}{\partial\tau} \right)_P-\left( \frac{\partial\mu_g}{\partial\tau} \right)_P}{\left( \frac{\partial\mu_g}{\partial\tau} \right)_\tau-\left( \frac{\partial\mu_l}{\partial\tau} \right)_\tau}$$

      \item Furthermore, implementing pressure and volume per molecules ($s$ and $v$), we may write:

        $$\frac{dP}{dt}=\frac{s_g-s_l}{v_g-v_l}$$

    \end{itemize}

  \item We can recall that, in a reversible process:

    $$\delta Q=\tau\,d\sigma$$
    $$\Delta Q=\tau\Delta\sigma$$
    $$\Delta Q=\tau(s_g-s_l)$$

    \begin{itemize}

      \item Thus, we may described the heat gained from moving 1 molecule from $g$ to $l$

      \item This $\Delta Q$ term is referred to as $L$, or the latent heat of vaporization per molecule

    \end{itemize}

  \item This leads us to the Clausius-Clapeyron Equation:

    $$\frac{dP}{d\tau}=\frac{L}{\tau(v_g-v_l)}$$

    \begin{itemize}

      \item This is used in liquid-gas coexistence

      \item We assume the gas is ideal

      \item We also assume $v_l<<v_g$

      \item Also, $L$ is assumed to be constant

      \item In assuming this, we may write:

        $$\frac{dP}{d\tau}=\frac{LP}{\tau^2}$$

      \item Solving this, we get:

        $$P(\tau)=P_oe^{-\frac{L}{\tau}}$$

    \end{itemize}

\end{itemize}

\end{document}




%%%%%%%%%%%%%%%%%%%%%%%%%%%%%%%%%%%%%%%%%%%%%%%%%%%%%%%%%%%%%%%%%%%%%%%%%%%%%%%%%%%%%%%%%%%%%%%%%%%%%%%%%%%%%%%%%%%%%%%%%%%%%%%%%%%%%%%%%%%%%%%%%%%%%%%%%%%%%%%%%%%
% Written By Michael Brodskiy
% Class: Thermodynamics & Statistical Mechanics
% Professor: A. Stepanyants
%%%%%%%%%%%%%%%%%%%%%%%%%%%%%%%%%%%%%%%%%%%%%%%%%%%%%%%%%%%%%%%%%%%%%%%%%%%%%%%%%%%%%%%%%%%%%%%%%%%%%%%%%%%%%%%%%%%%%%%%%%%%%%%%%%%%%%%%%%%%%%%%%%%%%%%%%%%%%%%%%%%

\documentclass[12pt]{article} 
\usepackage{alphalph}
\usepackage[utf8]{inputenc}
\usepackage[russian,english]{babel}
\usepackage{titling}
\usepackage{amsmath}
\usepackage{graphicx}
\usepackage{enumitem}
\usepackage{amssymb}
\usepackage[super]{nth}
\usepackage{everysel}
\usepackage{ragged2e}
\usepackage{geometry}
\usepackage{multicol}
\usepackage{fancyhdr}
\usepackage{cancel}
\usepackage{siunitx}
\usepackage{physics}
\usepackage{tikz}
\usepackage{mathdots}
\usepackage{yhmath}
\usepackage{cancel}
\usepackage{color}
\usepackage{array}
\usepackage{multirow}
\usepackage{gensymb}
\usepackage{tabularx}
\usepackage{extarrows}
\usepackage{booktabs}
\usepackage{lastpage}
\usetikzlibrary{fadings}
\usetikzlibrary{patterns}
\usetikzlibrary{shadows.blur}
\usetikzlibrary{shapes}

\geometry{top=1.0in,bottom=1.0in,left=1.0in,right=1.0in}
\newcommand{\subtitle}[1]{%
  \posttitle{%
    \par\end{center}
    \begin{center}\large#1\end{center}
    \vskip0.5em}%

}
\usepackage{hyperref}
\hypersetup{
colorlinks=true,
linkcolor=blue,
filecolor=magenta,      
urlcolor=blue,
citecolor=blue,
}


\title{States of a Model System}
\date{\today}
\author{Michael Brodskiy\\ \small Professor: A. Stepanyants}

\begin{document}

\maketitle

\begin{itemize}

  \item Consider a system of $N$ quantum particles in a stationary quantum state ($U$, $V$, $N$\ldots are independent of time)

  \item Multiplicity of degeneracy of an energy level $\varepsilon_n$ is the number of quantum states, $g_n$, corresponding to $\varepsilon_n$

  \item Hydrogen Atom:

    \begin{itemize}

      \item One proton, one electron

      \item $\varepsilon_n=-\dfrac{13.6[\si{\eV}]}{n^2}$

      \item $\psi_{n,l,m,s}=R_{n,l}(r)\Theta_{l,m}(\theta)\Phi_m(\phi)(s)$

        \begin{itemize}

          \item $n,l,m,s$ are quantum numbers

          \item $n$ is called the principle quantum number

          \item $l$ is the angular momentum quantum number

            \begin{itemize}

              \item $0\leq l\leq n-1$

            \end{itemize}

          \item $m$ is the magnetic quantum number

            \begin{itemize}

              \item $-l\leq m\leq l$

            \end{itemize}

          \item $s$ is the spin-magnetic quantum number

            \begin{itemize}

              \item $s=\pm\frac{1}{2}$

            \end{itemize}

          \item $g_n=2n^2$

        \end{itemize}

    \end{itemize}

  \item Quantum Particle in a Box ($L\times L\times L$)

    \begin{itemize}

      \item We find $\varepsilon_{n_x,n_y,n_z}=\frac{\hbar^2\pi^2}{2mL^2}(n_x^2+n_y^2+n_z^2)$

      \item $\psi_{n_x,n_y,n_z}(x,y,z)=\ldots$

      \item $1\leq n_x,n_y,n_z\leq \infty$

        \begin{center}
          \begin{tabular}[h]{c | c | c | c}
            $n_x$ & $n_y$ & $n_z$ & $\varepsilon_{n_x,n_y,n_z}/A$\\
            \hline
            1 & 1 & 1 & 3\\
            \hline
            2 & 1 & 1 & 6\\
            1 & 2 & 1 & 6\\
            1 & 1 & 2 & 6\\
            \hline
            2 & 2 & 1 & 9\\
            2 & 1 & 2 & 9\\
            1 & 2 & 2 & 9\\
            \hline
          \end{tabular}
        \end{center}

      \item We can see that the ``6'' energy level is degenerate, with a multiplicity of $3$, just like ``9''

    \end{itemize}

  \item Binary Model System

    \begin{itemize}

      \item Energy of the system, $\varepsilon=-MB$

        \begin{itemize}

            \item $M$ is total magnetic moment: $M=$ (spins up - spins down)$m\rightarrow M=\left( N_{\uparrow}-N_{\downarrow} \right)m$

            \item $(N_{\uparrow}-N_{\downarrow})=2s$ — spin excess

            \item Thus, $\varepsilon=-2mBs$, meaning it is dependent on spin excess

            \item $N=3$ example:

              \begin{itemize}

                \item $\uparrow\uparrow\uparrow$ — $2s=N_{\uparrow}-N_{\downarrow}=3\Rightarrow g=1$

                \item $\uparrow\uparrow\downarrow$, $\uparrow\downarrow\uparrow$, $\downarrow\uparrow\uparrow$ — $2s=N_{\uparrow}-N_{\downarrow}=1\Rightarrow g=3$

                \item $\uparrow\downarrow\downarrow$, $\downarrow\downarrow\uparrow$, $\downarrow\uparrow\downarrow$ — $2s=N_{\uparrow}-N_{\downarrow}=-1\Rightarrow g=3$

                \item $\downarrow\downarrow\downarrow$ — $2s=N_{\uparrow}-N_{\downarrow}=-3\Rightarrow g=1$

              \end{itemize}

            \item In general, there are $N+1$ values of $2s$ (or $M$ or $\varepsilon$) and $2^N$ states of the system in total $\Rightarrow$ same energy levels have very high multiplicity

        \end{itemize}

      \item Calculation of $g(N,s)$

        $$\left\{\begin{array}{l} N_{\uparrow}-N_{\downarrow}=2s\\N_{\uparrow}+N_{\downarrow}=N\end{array}\quad\quad\left\{\begin{array}{l} N_{\uparrow}=\frac{N}{2}+s\\N_{\downarrow}=\frac{N}{2}-s\end{array}$$

          \item $g(N,s)=\dfrac{N!}{N_{\uparrow}!N_{\downarrow}!}$

          \item Drawing from combinatorics below, an approximation of $g(N,s)$ for $N>>1$ and $s<<N$, we can use the Stirling formula:

            $$N!\approx\sqrt{2\pi N}\left( \frac{N}{e} \right)^N$$

          \item $\ln(N!)\approx N\ln(N)-N+\frac{1}{2}\ln(N)+\frac{1}{2}\ln(2\pi)+0\left( \frac{1}{N} \right)$

          \item $\ln(1+x)\approx x-\frac{1}{2}x^2+0x^3,\quad -1\leq x\leq 1$

          \item Thus, $g(N,s)\approx 2^N\sqrt{\frac{2}{\pi N}}e^{-\frac{2s^2}{N}}$

    \end{itemize}

  \item Important Combinatorics

    \begin{itemize}

      \item $(x+y^n)=\displaystyle\sum_{k=0}^n \left( ^n_k \right)x^ky^{n-k}$ — Binomial Expansion

      \item \large$\left( ^n_k \right)=\frac{n!}{k!(n-k!)}$ \normalsize — Binomial Coefficient

    \end{itemize}

  \item Gaussian Probability Density Function (PDF)

    $$G(x)=\frac{1}{\sqrt{2\pi}\sigma}e^{-\frac{x^2}{2\sigma^2}}$$

    \begin{itemize}

      \item $\sigma$ represents the standard deviation of $G$

    \end{itemize}

  \item Macroscopic properties of a large system are well defined (\textit{i}.\textit{e}.\ fluctuations about the mean values are small $\approx O(\sqrt{N})$)

  \item Some Important Definitions:

    \begin{itemize}

      \item ``Closed System'' — A system is said to be closed if it has constant energy, $U$, number of particles, $N$, volume, $V$, and other physical properties; a closed system can interact with constant external fields

      \item ``Accessible Quantum State'' — A quantum state is accessible to a system if it is compatible with all physical constraints on the system

      \item Fundamental assumption of thermodynamics — A closed system is equally likely to be in any quantum state accessible to it

    \end{itemize}

  \item Probabilities

    \begin{itemize}

      \item If the \# of accessible quantum states is $g$, then the probability to find the system in a given state is:

        $$P(s)=\left\{\begin{array}{l} 0,\,\quad\text{if $s$ is not accessible}\\\displaystyle\frac{1}{g},\,\quad\text{if $s$ is accessible}\end{array}$$

          We also know:

          $$\underbrace{\sum P(s)}_{\text{all quantum states}}=\underbrace{\sum_{i=1}^g\frac{1}{g}}_{\text{accessible only}}=1$$

        \item The average of an observable property of the system, $X$:

          $$\langle X\rangle=\underbrace{\sum X(s)P(s)}_{\text{all quantum $s$}}=\underbrace{\sum_{i=1}^g X(s)\frac{1}{g}}_{\text{accessible $s$}}=\frac{1}{g}\sum_{i=1}^g X(s)$$

        \item $\langle\rangle$ defines the average over all quantum states on ensemble average

          \begin{itemize}

            \item An ensemble is a set of replicas of the original system, each replica is in one of the quantum states accessible to the system

              $$\boxed{\uparrow\uparrow\downarrow}\quad\boxed{\uparrow\downarrow\uparrow}\quad\boxed{\downarrow\uparrow\uparrow}$$

              shows an ensemble of states of a binary model system with $N=3,\,2s=1$

          \end{itemize}

    \end{itemize}

  \item Logarithmic Derivatives

    $$\frac{1}{y(x)}\frac{d\,y(x)}{dx}=\frac{d\,\ln(y(x))}{x}$$

    \begin{itemize}

      \item The most likely configuration of two binary model systems in thermal contact thus can be found with:

        $$\frac{\partial\,\ln(g(N_1,s_1))}{\partial s_1}=\frac{\partial\,\ln(g(N_2,s_2))}{\partial s_2}$$

      \item Using the definition of $g$ from above, we get:

        $$\ln\left(2^N\sqrt{\frac{2}{\pi N}}\right)\left( -\frac{2s^2}{N} \right)$$
        $$-\frac{4s_1}{N_1}=-\frac{4s_2}{N_2}$$
        $$\boxed{\frac{\hat{s_1}}{N_1}=\frac{\hat{s_2}}{N_2}}$$

      \item Using $\hat{s_1}+\hat{s_2}=s$, we obtain:

        $$\hat{s_1}+\hat{s_2}=s\Rightarrow\left\{\begin{array}{l} \hat{s_1}=\displaystyle\frac{sN_1}{N_1+N_2}\\\\\hat{s_2}=\displaystyle\frac{sN_2}{N_1+N_2}\end{array}$$

    \end{itemize}

  \item How sharp is the maximum ($\hat{s_1}\hat{s_2}$)?

    $$\underbrace{g(N_1,\hat{s_1}+s)g(N_2,\hat{s_2}-s)}_{\text{\# of q-states in perturbed config}}=\underbrace{g(N_1,\hat{s_1})g(N_2,\hat{s_2})e^{-2s^2\left( \frac{1}{N_1}+\frac{1}{N_2} \right)}}_{\text{\# of q-states in most likely config}}$$

    \begin{itemize}

      \item Ex. Consider $N_1=N_2=10^{22}$, and $s=10^{12}$


        $$e^{-2s\left( \frac{1}{N_1}+\frac{1}{N_2} \right)}=e^{-400}\approx 10^{-174}<<<< 1$$

      \item In summary, the fluctuations of physical properties of large systems about those observed for the most likely configuration are small; average properties of systems in thermal contact are accurately observed by the most likely (thermal equilibrium) configuration

    \end{itemize}

  \item Thermal equilibrium (most likely) configuration of 2 systems in contact

    \begin{itemize}

      \item Given two systems, one with $N_1$ and $U_{1_o}$, and another with $N_2$ and $U_{2_o}$, the two are put into thermal contact. The energies convert to $U_1$ and $U_2$, respectively. This yields a system with $N_1+N_2=N$ and $U_1+U_2=U_{1_o}+U_{2_o}=U$

      \item The multiplicity (or number of quantum states) of 1 and 2 combined generates:

        $$g(N,U)=\sum_{U_1}g(N_1,U_1)g(N_2,U-U_1)$$

      \item To find the most likely configuration $(U_1)$, we need to find the largest term of the sum:

        $$\frac{\partial}{\partial U_1}\left( g(N_1,U_1)g(N_2,U-U_1) \right)\Rightarrow$$
        $$\frac{\partial g(N_1,U_1)}{\partial U_1}g(N_2,U-U_1)+g(N_1,U_1)\frac{\partial g(N_2,U-U_1)}{U_2}(-1)=0$$
        $$\frac{1}{g(N_1,U_1)}\frac{\partial g(N_1,U_1)}{\partial U_1}=\frac{1}{g(N_2,U_2)}\frac{\partial g(N_2,U_2)}{\partial U_2}$$
        $$\frac{\partial \ln(g(N_1,U_1))}{\partial U_1}=\frac{\partial \ln(g(N_2,U_2))}{\partial U_2}$$

      \item Entropy, $\sigma$, is denoted by: $\sigma(N,U)\equiv\ln(g(N,U))$

      \item Fundamental Temperature, $\tau$, is denoted by: $\frac{1}{\tau}\equiv\frac{\partial \sigma(N,U)}{\partial U}\equiv\left( \frac{\partial \sigma(N,U)}{\partial U}\right)_N$\footnote{The $N$ subscript is used to denote constant, invariable quantities}

      \item Thus, we can simplify the above to:

        $$\frac{1}{\tau_1}=\frac{1}{\tau_2}$$

      \item When systems 1 and 2, together, are in the most likely configuration, $\tau_1=\tau_2$, then it is said that $1+2$ is in thermal equilibrium

        $$\boxed{T=\frac{\tau}{k_B},\quad\quad\text{Kelvin (absolute) temperature}}$$
        $$\boxed{S=k_B\sigma,\quad\quad\text{Conventional entropy}}$$
        $$\boxed{k_B=1.381\cdot10^{-23}\left[ \frac{\si{\joule}}{\si{\kelvin}} \right],\quad\quad\text{Boltzmann constant}}$$

    \end{itemize}

  \item Second Law of Thermodynamics (Law of Increase of Entropy):

    \begin{itemize}

      \item The final entropy will always be greater than or equal to the intial entropy ($\sigma_f\geq\sigma_i$)

      \item The entropy of a closed system tends to increase or remain constant when the constraint internal to the system is removed

      \item Entropy increase is an irreversible process

    \end{itemize}

  \item The \nth{0} Law of Thermodynamics:

    \begin{itemize}

      \item If system 1 is in thermal equilibrium with system 3, and 2 is in thermal equilibrium with system 3, then 1 is in thermal equilibrium with 2; that is:

        $$\tau_1=\tau_3\text{ and }\tau_2=\tau_3\Rightarrow \tau_1=\tau_2$$

    \end{itemize}

  \item The \nth{1} Law of Thermodynamics:

    \begin{itemize}

      \item Represents conservation of energy — $\delta Q$ is the heat added to the system, $dU$ is the change in a systems internal energy, and $\delta W$ is the work done by the system

        $$\delta Q=dU+\delta W$$

    \end{itemize}

  \item The \nth{2} Law of Thermodynamics:

    \begin{itemize}

      \item Law of increase of entropy

    \end{itemize}

  \item The \nth{3} Law of Thermodynamics:

    \begin{itemize}

      \item As the temperature goes to zero, the entropy goes to a constant ($U_g$ represents the ground state energy)

        $$\tau\to0\Rightarrow\sigma\to C$$
        $$\sigma=\ln(g(N,U_g))$$

    \end{itemize}

  \item Flow of energy for 2 systems in thermal contact

    \begin{itemize}

      \item Given two systems with $N_1$, $U_1$ and $N_2$, $U_2$, respectively, we find:

        $$\sigma=\sigma_1+\sigma_2$$
        $$\Delta \sigma=\Delta\sigma_1+\Delta\sigma_2=\frac{\partial\sigma_1}{\partial U_1}\Delta U_1+\frac{\partial\sigma_2}{\partial U_2}\Delta U_2\Rightarrow$$
        $$\frac{1}{\tau_1}\Delta U_1+\frac{1}{\tau_2}\Delta U_2=\frac{1}{\tau_1}(-\Delta U)+\frac{1}{\tau_2}\Delta U\Rightarrow$$
        $$\Delta U\left( \frac{1}{\tau_2}-\frac{1}{\tau_1} \right)\geq 0$$

        \begin{itemize}

          \item Thus, we expect $\tau_1\geq\tau_2\Rightarrow\Delta U\geq0$

        \end{itemize}

      \item Energy flows from hot (high energy) to cold (low energy)

    \end{itemize}

\end{itemize}

\end{document}

